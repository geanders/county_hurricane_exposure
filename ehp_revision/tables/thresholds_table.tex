\begin{longtable}{lp{35em}}
\caption{Reasons behind the choices of thresholds for binary exposure classifications, as well as discussion of some other reasonable choices. These are provided for the three exposure metrics for which our database includes continuous data, and so a threshold is selected to determine binary exposure based on the metric. This table provides reasoning for the choice of threshold used in this paper as well as guidance on other thresholds that could be considered, depending on the hypothesized pathways for an epidemiological study.} \\
\label{tab:thresholds} \\
\hline
Metric & Threshold choice\\
\hline
\textbf{Distance} & Distance-based exposure was determined based on whether the storm track came within 100 km of county population mean center. This threshold has been used in prior research as a proxy for exposure to hazards from the storm (e.g., \cite{grabich2015measuring}). Tropical cyclones vary dramatically in size: US tropical cyclones have been observed with radii to maximum winds as small as 20 km and as large as 200 km \parencite{mallin2006, quiring2011variations}, and dangerous winds can extend beyond these maximum winds. One study assessed county-level risk and exposure based on a three-tiered definition, with primary counties being those closest to the storm track on either side, secondary counties being adjacent to primary counties, and tertiary counties adjacent to secondary counties, which resulted in an average distance radius of 120 km on either side of the storm track \parencite{czajkowski2011}. Other distance thresholds that could be considered include 60 km and 30 km, both of which have been used in previous research \parencite{grabich2015measuring, grabich2015, currie2013}. However, based on the results presented in the main manuscript, hazard-based metrics should often be used directly rather than a distance-based proxy.\\

\textbf{Rain} & Rain-based exposure was determined based on whether the county had cumulative rainfall of $\ge75$ mm over the period from two days before to one day after the storm’s closest approach and the storm track came within 500 km of the county population mean center. One recent study has highlighted that a two-year rainfall value (which is the median annual maximum rainfall) for a location can provide a useful threshold in identifying rainfall events with potential for societal impacts \parencite{bosma2020intuitive}. For some of the more northern, inland communities included in our study area, the two-year rainfall value for two-, three- and four-day windows is in the 65--85 mm range. For example, Pittsburgh, PA, has two-year rainfall values of 69 mm for a two-day window, 74 mm for a three-day window, and 78 mm for a four-day window \parencite{noaaatlas}, consistent with the 75 mm threshold we selected for exposure classification in this paper. However, other thresholds, particularly higher thresholds, would be reasonable in some cases. For example, the two-year rainfall values tend to be much higher in counties of the study area that are further south and close to the coast. The two-year rainfall value for Miami for a three-day window, for example is 180 mm \parencite{noaaatlas}. A variety of definitions have been used previously to identify both extreme or heavy rain (whether associated with a tropical cyclone or not) and tropical cyclone--associated rain. In defining precipitation associated with tropical cyclones, studies have used thresholds of 12.5 mm per day as a metric of regions of ``moderately heavy'' rainfall \parencite{zhou2017spatial} and, as a lower threshold, a lower limit of 10 mm of total storm precipitation---\parfillskip=0pt\tabularnewline

&in conjunction with proximity to the storm's center---in identifying tropical cyclone precipitation events at a location \parencite{feldmann2019estimation}. Other definitions of extreme rain events---including but not limited to tropical cyclone--associated rainfall---are higher than the threshold we use here---for example, one paper defined extreme rain events as cases in which a gauge reported 125 mm or more of rain in 24 hours \parencite{schumacher2006characteristics}. Studies have also used definitions that are relative to the norms for a given location (e.g., 24 hour rainfall totals over the 50-year return value for the location, which the part of the US east of the Rocky Mountains range from 3.5 in [89 mm] to 13 in [330 mm]) \parencite{schumacher2006characteristics, schumacher2005organization, stevenson201410}.\\

\textbf{Wind} & Wind-based exposure was determined based on whether modeled storm-associated peak sustained surface wind was $\ge34$ kts at the county’s population mean center. This threshold is being applied to local winds for each county, and it represents the threshold for gale-force winds on the Beaufort wind scale. This limit is used as the outer limit in measuring storm size through the US National Hurricane Center's wind radii for tropical cyclone forecasts \parencite{cangialosi2016examination}. Other thresholds could be selected based on other points on the Beaufort scale---for example, $\ge48$ kts for capturing storm-force winds or $\ge64$ kts for capturing hurricane-force winds. As a note, hurricane-force winds will be rarely experienced for counties, as it will likely only be observed for very severe storms and even for those, only for counties near the storm's landfall. Many presentations of the Beaufort wind scale include descriptions of the conditions that winds in each category would produce both over land and at sea.\\
\hline
\end{longtable}