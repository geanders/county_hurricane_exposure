\begin{table}

\caption{\label{tab:highprecipcorr}Precipitation correlation during all versus high-precipitation events.
                     The same sample of counties is shown as in Figure ... of the main text.
                     Events are cases where a tropical cyclone came within ... km of each of the 
                     listed counties. The number of total events gives the sum of all points
                     shown on the main plot for the county in Figure ... of the main text. 
                     The Spearman correlation for all events is the same as that shown in Figure ...
                     of the main text. High-precipitation events are those for which storm-associated
                     precipitation was 75 mm or higher based on at least one of the two measures
                     considered in this comparison (NLDAS and monitors). The Spearman correlation
                     for these high-precipitation events is given in the last column of the 
                     table.}
\centering
\begin{tabular}[t]{lcccc}
\toprule
\multicolumn{1}{c}{ } & \multicolumn{2}{c}{All events} & \multicolumn{2}{c}{High-precipitation events} \\
\cmidrule(l{3pt}r{3pt}){2-3} \cmidrule(l{3pt}r{3pt}){4-5}
County & \makecell[c]{Number\\of events} & \makecell[c]{Spearman\\correlation} & \makecell[c]{Number\\of events} & \makecell[c]{Spearman\\correlation}\\
\midrule
Miami-Dade, FL & 65 & 0.94 & 18 & 0.49\\
Harris, FL & 38 & 0.93 & 10 & 0.84\\
Mobile, AL & 50 & 0.95 & 20 & 0.57\\
Orleans, LA & 55 & 0.89 & 13 & 0.95\\
Fulton, GA & 48 & 0.95 & 12 & 0.69\\
\addlinespace
Charleston, SC & 73 & 0.94 & 17 & 0.65\\
Wake, NC & 61 & 0.98 & 12 & 0.84\\
Baltimore, MD & 33 & 0.92 & 5 & 0.70\\
Philadelphia, PA & 52 & 0.96 & 6 & 0.77\\
\bottomrule
\end{tabular}
\end{table}
