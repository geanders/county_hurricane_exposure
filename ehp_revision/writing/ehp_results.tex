\section*{Results}

\subsection*{Exposure assessment data, data validation, and software}

We created and published this tropical cyclone exposure data as an R
package~\parencite{hurricaneexposuredata}.  The package size exceeded the
recommended maximum size for \ac{CRAN}, the standard repository for publishing
R packages. Therefore, we used the \textit{drat} framework to set up our own
repository to host the package~\parencite{anderson2017hosting}. These data
include continuous county-level measurements for the closest distance of each
storm, cumulative rainfall, and peak sustained surface wind. These continuous
metrics can be used for classifying counties as exposed or unexposed---using
thresholds selected by the user---or can be extracted directly as continuous
metrics. The dataset also includes binary data on flood and tornado events
associated with each storm in each county. To accompany these data, we
published an additional R package with software tools to explore and map the
data and to integrate it with human health
datasets~\parencite{hurricaneexposure}.

We explored potential limitations in these data by comparing them with data
from other available sources.  For estimates of storm-associated rainfall, we
compared data in the open-source package with ground-based
observations in nine sample counties (Figure~\ref{fig:raincomparison}). Within
these counties, storm-related rainfall measurements were well-correlated
between the two data sources, with rank correlations (bottom right of each
graph in Figure~\ref{fig:raincomparison}) between 0.87 and 0.98. 
There was some evidence that our primary rainfall metric may tend to
underestimate rainfall totals in storms with extremely high rainfall, based on
a few heavy-rainfall storms in Harris County, TX, Mobile County, AL, Charleston
County, SC, and Wake County, NC (Figure~\ref{fig:raincomparison}). However,
even in these cases, it was rare for a storm to be classified differently
(under the precipitation threshold of 75~\si{\milli\metre} we used for binary
exposure classification for later analysis) based on the source of
precipitation data. Horizontal and vertical lines in each small plot in
Figure~\ref{fig:raincomparison} show the threshold of 75~\si{\milli\metre}, so
storms in the lower left and upper right quadrants would be classified the same
(``exposed'' or ``unexposed'') regardless of the precipitation data source,
while storms in the upper left and lower right quadrants would be classified
differently. Such cases were rare.

For peak sustained surface wind estimates, we found that the primary wind
exposure metric in the open-source data generally agreed well with the wind
radii reported in \ac{HURDAT2}. For most storms, $\>$90\% of counties were
assigned the same category of wind speed ($<$34~kt; 34\,--\,49.9~kt;
50\,--\,63.9~kt; $\ge$64~kt) by both data sources
(Figure~\ref{fig:windcomparison}).  Disagreement was limited to a few storms
(e.g., Hurricanes Sandy in 2012 and Ike in 2008). For these two storms, the
modeled wind speed in the open-source data somewhat overestimated the severity
of the storm's winds at landfall but then underestimated, particularly for
Hurricane Sandy, how far 34\,--\,49.9 kt winds extended from the storm's
central track further inland (Figure~S2). For epidemiology researchers who
would like to conduct sensitivity analysis using both sets of wind data,
we have included estimates from the \ac{HURDAT2} wind radii as a secondary
measure of county-level peak sustained wind in the open-source dataset
\parencite{hurricaneexposuredata}.

For flood data, we compared the flood status values included in the open-source
data to streamflow measurements at \ac{US} Geological Survey gages within nine
sample counties (Figure~\ref{fig:floodcomparison}). Across the sampled
counties, streamgage data generally indicated higher discharge during periods
identified as flood events based on the NOAA Storm Events database
(Figure~\ref{fig:floodcomparison}). There were some cases, however, where the
two flooding data sources were somewhat inconsistent.  For example, there were
one or two tropical cyclones in several of the counties (Mobile County, AL,
Escambia County, FL, Fairfield County, CT, and Fulton County, GA) with
associated flood event listings but for which the total discharge across county
streamflow gages was relatively low. For storms without a flood listing for the
county, in most cases the total streamflow discharge in the county was
relatively low, and, in all but two cases all streamgage flows were
below the flooding threshold. The exceptions were for Hurricane Ida in 2009 in
Fulton County, GA, and Hurricane Isaac in 2012 in Mobile County, AL. 

\subsection*{Patterns in tropical cyclone exposures}

We used this storm exposure data to classify counties as exposed or unexposed
to four different hazards for each tropical cyclone and then explored patterns
over years with available data. Across the four storm hazards considered, there
was wide variation in the average number of county exposures per year
(Table~\ref{tab:exposuresummaries}). For tropical cyclone tornadoes,
there were on average about 40 county exposures per year within our study.
County exposures were more frequent for tropical cyclone wind exposures ($>$160
/ yr on average), even more frequent for tropical cyclone flood exposures
($>$190 / yr on average), and most frequent for tropical cyclone rain exposure
($>$290 / yr on average). For every hazard except tornadoes, we identified at
least one tropical cyclone that exposed over~250 counties
(Table~\ref{tab:exposuresummaries}).  However, the largest-extent tropical
cyclone varied across hazards: Frances in~2004 exposed the most counties based
on rain, Michael in~2018 based on wind, and Ivan in 2004 based on flooding and
tornadoes (Table~\ref{tab:exposuresummaries}).

When we calculated and mapped the average number of exposures per decade in
each county for single-hazard exposures (Figure~\ref{fig:averageexposure}),
strong geographical patterns were clear. Peak sustained wind exposure had a
strong coastal pattern, with almost all exposures in counties within about~200
\si{\kilo\metre} (124~mi) of the coastline. While tropical cyclone rain
exposure was also more frequent in coastal areas compared to inland areas,
there were also inland rain exposures in counties that were rarely or never
classified as exposed based on wind. Flood-based exposures were frequently in
the Mid-Atlantic region, with a pattern that skewed north compared with other
exposures. Rain and, to some extent, flood exposures were characterized by a
pattern defined by the Appalachian Mountains, with fewer exposures west of the
mountain range than to the east. Almost all tornado-based exposures were in
coastal states, with many in Florida, and almost none north of Maryland.
Patterns were similar when analysis was restricted to years with exposure data
for all four hazards available (1996\,--\,2011; Figure~S3). 

\subsection*{Agreement across exposure metrics}

Finally, we assessed within-storm agreement between exposure classifications
for each pair of hazards, as well as with a proxy measurement based on distance
from the storm's track. We found that these exposure classifications typically
did not agree strongly between pairs of metrics---the set of counties
identified as ``exposed'' based on one metric often overlapped little with the
set identified as ``exposed'' by another metric, as storms frequently brought
different hazards to different locations. 

Figure~\ref{fig:ivanexposure} shows as an example Hurricane Ivan in 2004.
For the distance-based metric, the counties assessed as exposed follow the
tropical cyclone's track. For the wind-based metric,
only counties near the tropical cyclone's first landfall were assessed as
exposed. For rain- and flood-based metrics, however, exposure extended to the
left of the track, including counties as far north as New York and Connecticut,
while for the tornado metric, exposed counties tended to be to the right of the
track and included several counties in central North Carolina, South Carolina,
and Georgia that were not identified as exposed to Ivan based on any other
metric. Figure~S4 provides similar maps for three other example tropical
cyclones (selected because they exposed many \ac{US}  counties based on at
least one metric).

We drew similar conclusions when we investigated all~46 tropical cyclones
between~1996 and~2011 (when data for all five metrics were available) for
which~100 or more counties were exposed based on at least one metric
(Figure~\ref{fig:jaccard}; \textbf{Tables S[x]--S[x] provide the underlying
numbers for the most extensive of these, the storms for which 250 of more
counties were exposed based on at least one metric}). In this figure, each row
provides results for one tropical cyclone, and each box in that row shows the
Jaccard coefficient for a pair of metrics. For all pairs of metrics, agreement
in exposure assessment was, at best, moderate for all but a few storms. When
comparing distance- and wind-based exposure assessment, only about 10\% of
storms had Jaccard indices higher than 0.6 (i.e., out of the counties assessed
as exposed by at least one of the two metrics in the pair,~60\si{\percent} or
more were assessed the same under both metrics). For comparisons of assessments
based on other combinations of distance-, wind-, rain-, and flood-based
metrics, fewer than 5\% of storms had Jaccard indices above 0.6.  The
tornado-based metric had universally poor agreement with other metrics in
county-level classification across the tropical cyclones considered.  

There were a few exceptions---tropical cyclones in which exposure assessment
agreed well across several of the metrics considered.  For Floyd in~1999
(Figure~S4) and Irene in~2011, for example, county-level classification agreed
moderately to well for all pairs of exposure metrics except those involving the
tornado-based metric (Figure~\ref{fig:jaccard}). For another set of tropical
cyclones (e.g., Ernesto in~2006, and Bertha in~1996, Isabel in~2003), there was
moderate to good agreement for pairwise combinations of distance, rain, and
wind, but poor agreement for other combinations of metrics, while for another
set of storms (e.g., Matthew in~2004 and Katrina in~2005), there was moderate
to good agreement between distance and rain.  



