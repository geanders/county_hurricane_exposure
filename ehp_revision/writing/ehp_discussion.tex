\section*{Discussion}

Epidemiologic studies can help characterize which health risks are elevated
during disasters, to what degree, and for
whom~\parencite{ibrahim2005unfortunate, noji2005disasters}.  As a result, these
studies help improve disaster preparedness and
response~\parencite{noji2005disasters}.  However, tropical cyclones are
multi-hazard events, making it complicated to measure exposure and so to
conduct multi-community, multi-year studies leveraging large administrative
datasets.  Here, we provide open-source county-level data for several tropical
cyclone exposures and explore limitations in that data.  
Further, we explore patterns in storm exposure classifications based on
different metrics, and we find that county-level tropical cyclone exposure
assessments vary substantially when using different metrics.
Our results can inform exposure assessment for future county-level studies of
the health risk and impacts associated with tropical cyclones exposure as
well as provide insights to inform epidemiologic study design.  

\subsection*{Exposure assessment data and software}

The open-source data generally correspond well with data from other potential
sources (Figures~\ref{fig:raincomparison}\,--\,\ref{fig:floodcomparison}), but
there are caveats. The rainfall data are generally well-correlated with
ground-based observations, but may sometimes underestimate very high rainfall
values (Figure~\ref{fig:raincomparison}), \textbf{and in some counties the
correlation was substantially lower when considering only tropical cyclones
with cumulative local rainfall of $\ge75$ mm (Supplemental Table 2).} When
rainfall data is used to create binary exposure classifications, this
disagreement is unlikely to influence results, as both data sources agree in
identifying these as storms with high rainfall, but would be important to
consider for cases that include rainfall as a continuous measurement. 

The peak sustained wind estimates are based on modeled, rather than observed,
values, and while the modeled wind data generally agree well
(Figure~\ref{fig:windcomparison}) with post-analysis maximum wind radii
\parencite{landsea2013}, there were a few storms with some discrepancies. These
storms---for example, Hurricane Sandy in 2012 and Hurricane Ike in 2008---were
unusually large systems for which high winds persisted well inland from
landfall (Figures~\ref{fig:windcomparison} and~S2). 

For the flooding data, we found that flood event status, as determined based on
the NOAA Storm Events listings, typically agreed with measurements from
\ac{USGS} streamgages, with a flood event more likely to be listed if a storm
elevated streamflow at streamgages across the county
(Figure~\ref{fig:floodcomparison}).  However, there are differences between the
two flooding datasets, and these highlight both the difficulty of measuring
flood exposure at the county level and inherent challenges in using data from a
storm event database for epidemiologic exposure assessment.  For example, there
was one storm in Fulton County, GA, for which there was high streamflow but not
an associated flood event listing (Hurricane Ida, 2009).  This storm occurred
in November 2009, following a month with historic rainfall and flooding in
Georgia~\parencite{shepherd2011overview}.  In this case, the flooding
associated with Ida was incorporated into an ongoing flood event listing, with
a start date well before the five-day window we used to temporally match storm
event listings with tropical cyclone tracks for our data. 

This disagreement highlights the difficulty of large-scale pairing of storm
tracks with storm event listings---without criteria for temporally matching
event start dates to storm dates, many false positives would be
captured, for which the occurrence of a storm in the midst of an ongoing 
event might be improperly attributed to the storm.  However, distance
and time restrictions, like those we used in matching NOAA Storm Event listings
with tropical cyclone locations and dates, do cause 
occasional false negatives, as for Hurricane Ida in Fulton County, GA, where a
storm contributes meaningfully to an ongoing event, but the event is not
captured for the storm in the exposure data because its start date is not
close in time to the storm's date.  

There are further limitations for the flood and tornado data---these data came
from the \ac{NOAA} Storm Events database, which, while a widely-used database
of events maintained by \ac{NOAA}, is based on reports, and so may be prone to
underreporting~\parencite{Ashley2008flood, Curran2000}, especially in
sparsely-populated areas~\parencite{Witt1998, Ashley2007}, as well as to other
reporting errors. 

While these are important caveats for the data, we selected these data sources
as among the best currently available for measuring each of these hazards
consistently and comprehensively at a multi-county, multi-year scale. In
addition to providing tropical cyclone exposure metrics for individual hazards,
this dataset and its associated software allow users not only to access
measurements for single hazards, but also to create tropical cyclone exposure
profiles based on multiple hazards or to craft exposure indices that combine
hazard metrics~\parencite{chakraborty2005population, peduzzi2009assessing}.
This functionality can be critical, as different hazards of tropical cyclones can
act synergistically in causing impacts~\parencite{smith2009}.  

\textbf{This dataset is limited to the contiguous US, and while expansion to
global coverage would be useful and feasible, there would be some challenges.
For precipitation data, the re-analysis product used here (NLDAS-2) only covers
the contiguous US, although other re-analysis products, as well as other types
of precipitation datasets, have global coverage \parencite{sun2018review}. The
wind data are generated based on a model that is currently US-focused
\parencite{stormwindmodel}, but could be extended to other areas, although this
would require adding new land/sea masks within the associated software, as well
as accounting for differences across storm basins in wind averaging periods
\parencite{harper2010guidelines} and in the direction of cyclonic winds in the
Northern versus Southern Hemisphere. Further, since the core of the wind model
was developed based on data from Atlantic-basin storms
\parencite{willoughby2006parametric}, an extension to other areas should
include separate validation and calibration to ensure it performs appropriately
in those settings. Alternatively, other wind field modeling software is
available that provides a global coverage, like Geoscience Australia's Tropical
Cyclone Risk Model (http://geoscienceaustralia.github.io/tcrm/). For tornado
and flood events, the data described in this paper drew on a US-focused storm
events database, and so international extension of data on these hazards would
require access to similar databases covering other countries. Finally, the
relevant geopolitical boundaries to use for aggregation would vary by country
(e.g., municipalities in Mexico, districts in India).}

\textbf{The dataset we present focuses on the physical hazards of a tropical
cyclone. However, health impacts will often come through indirect pathways,
including through damage to property and infrastructure. In future work, it
would be useful to expand this dataset to add data related to these pathways,
to allow research exploring the role of indirect pathways from tropical cyclone
phyiscal hazards to health risk. This expansion could include adding data on
normalized storm-associated damages, or proxy measurements of such damages,
from sources like the US NOAA's Storm Events database and the US Federal
Emergency Management Agency's county-level disaster declarations.}

\subsection*{Patterns in tropical cyclone exposures}

We found average exposures to different tropical cyclone hazards differed
geographically (Figure~\ref{fig:averageexposure}). These patterns were not
unexpected based on what is known about tropical cyclone hazards, but still
highlight variations that are critical to consider in designing studies and
statistical analysis for tropical cyclone epidemiology. Further, they
demonstrate the need for multi-hazard exposure datasets for tropical cyclone
epidemiology, especially in capturing inland risks. 

Tropical cyclone wind exposures had a strong coastal pattern, which is
consistent with the dramatic decrease in wind intensity that typically
characterizes the landfall of tropical cyclones.  Tropical cyclone rain
exposures tended to extend further inland compared to wind exposures, up to
the Appalachian mountains. This agrees with previous research indicating that
the Appalachian mountains' topography both enhances precipitation during
tropical cyclones and provides hydrological conditions for severe
flooding~\parencite{rees2001}.  Almost all tropical cyclone tornado exposures
were in southern coastal states, consistent with previous evidence that
tropical cyclone-related tornadoes typically occur to the right of tropical
cyclone tracks in Atlantic-basin \ac{US} storms~\parencite{moore2012}.  It is
important to note, however, that the exposure averages we calculated may be
limited as estimates of long-term frequencies, as tropical cyclones follow
decadal patterns~\parencite{kossin2007more} that may not adequately captured in
the available data. 

\subsection*{Agreement between exposure metrics}

We found that tropical cyclones tended to bring different hazards to different
counties, and so agreement was typically low between distance-based
tropical cyclone exposure assessment and each of the single hazard-specific
exposures, as well as between pairs of hazard-specific metrics
(Figures~\ref{fig:ivanexposure}\,--\,\ref{fig:jaccard} and S4).  These
findings align with previous results from atmospheric science and related
fields on the characteristics of tropical cyclones. While tropical cyclone
rainfall and windspeed can be well-correlated when the tropical cyclone is over
water~\parencite{cerveny2000}, this relationship often weakens once the
hurricane has made landfall~\parencite{jiang2008}.  Fast-moving tropical
cyclones heighten risk of dangerous winds inland~\parencite{kruk2010},
while slow-moving tropical cyclones are likely to bring more
rain~\parencite{rappaport2000} and may cause more damage because of sustained
hazardous conditions~\parencite{rezapour2014}. Further, while the likelihood
and extent of flooding during a tropical cyclone is related to the tropical
cyclone's rainfall, it is also driven by factors like top soil saturation and
the structure of the water basin's drainage network~\parencite{chen2015,
rees2001}. 

Based on our results, the use of a distance-based metric to assess exposure to
any of these hazards, or the use of measurements from one hazard as a proxy for
exposure to any of the other hazards considered, would often introduce exposure
misclassification \textbf{(Figure \ref{fig:jaccard}, Tables S3--S6)}. This
conclusion reinforces similar findings from a study of Florida's 2004 storm
season~\parencite{grabich2015measuring}.  For some studies, such exposure
misclassification might plausibly be differential.  For example, tropical
cyclone wind exposures tend to be concentrated in counties near the coast,
while tropical cyclone rain exposures sometimes extend well inland.  If the
etiologically-relevant exposure for a health outcome is extreme rainfall but
exposure is classified based on measurements of wind, the probability of being
misclassified as unexposed would be higher in inland counties, while the
probability of being misclassified as exposed would be higher in coastal
counties. If coastal counties differ from inland counties in either the outcome
of interest or in factors associated with risk of that outcome, exposure
misclassification would be differential~\parencite{savitz2016interpreting}.
Such differential exposure misclassification could bias estimates of tropical
cyclone effects either towards the null (estimating a lower or null association
compared to the true association that exists) or away from the null (estimating
a larger association than actually exists)~\parencite{savitz2016interpreting,
armstrong1998effect}.  

We did find a small set of tropical cyclones for which for which agreement was
high across several single-hazard exposure assessments (e.g., Floyd in~1999,
Irene in 2011, Hannah in~2008, Bertha in~1996; Ernesto in~2006
(Figure~\ref{fig:jaccard})).  Hurricanes Floyd in~1999 and Irene in~2011 both
made their first \ac{US} landfall in North Carolina at minor hurricane
intensity (Category~2 and~1, respectively) and then skimmed the eastern coast
of the \ac{US} north through New England, bringing substantial rainfall to much
of the eastern coast from North Carolina north and causing extensive inland
flooding in North Carolina (Floyd) and New England
(Irene)~\parencite{avila2013atlantic, lawrence2000atlantic}.  Hurricanes Hannah
in~2008, Bertha in~1996, and Ernesto in~2006 also followed the eastern
coastline. For these storms, the tropical cyclones' persistent proximity to
water may have helped maintain wind speeds in similar patterns to rain and
distance-based exposures, resulting in more similarities across exposure
assessments compared to other tropical cyclones.  For these storms, it may be
possible to assess exposure to multiple hazards of the storm using a single
metric, perhaps even a proxy like the distance between the county and the
storm's track. \textbf{With the dataset we describe in this paper, however,
there is little need to limit analysis based on exposure to a single hazard or
proxy, although multi-hazard studies of storms with high agreement among hazard
exposures should look out for modeling issues from multicollinearity.} Further,
for these storms it may be difficult to untangle the contribution of each
hazard to the overall effect of the storm, given that several hazards have
similar geographical patterns. 

\subsection*{Limitations}

The dataset presented here does have several limitations, in addition to the
caveats previously discussed. First, the dataset is not comprehensive of all
tropical cyclone hazards. For example, coastal counties can experience
dangerous storm surge, which is not specifically covered in this dataset
(although some resulting coastal flooding is captured). We are exploring ways
to include this in future versions of the dataset, but to date we have focused
on exposures that could affect any county, whether inland or coastal.  

Second, these data are aggregated to the county level. \textbf{This spatial
scale allows for easy integration with health outcome data aggregated at the
county level. Such data is often used for disaster epidemiology, as aggregated
data may be easier to access than individual-level data, especially at a scale
that covers many locations and years and so allows higher statistical power and
includes a broader range of exposure levels
\parencite{wakefield2008overcoming}.}

\textbf{Such ecological exposure assessment, however, sets a common exposure
level throughout the county, ignoring within-county variability, even though
such variability exists. For some hazards, this within-county variation could
be stark. For example, tornadoes cause very localized damage, directly along
the tornado's path---a tornado can destroy homes on one side of a street while
leaving those on the opposite side untouched. Levels of other hazards, like
storm-associated winds and rain, will also vary within a county, but typically
with smoother variation. In particular, it will be unlikely that a county will
have one area that is exposed to extremes of these hazards while other parts of
the county are completely unexposed, as both the wind fields and rain fields of
tropical cyclones tend to be large in comparison to the size of a county.}

\textbf{Aggregated data can be used to infer contextual-level
associations---for example, the association between county-level exposure to a
storm hazard and county-wide rates of a health outcome. However, ecological
data is also sometimes used to infer individual-level associations (e.g., the
association between personal exposure to a storm hazard and personal risk of
experiencing the health outcome). Individual-level inference from
ecological/aggregated data is susceptible to ecological bias
\parencite{greenland1994invited, portnov2007ecological, idrovo2011three}.
Researchers who use the data provided here for ecological studies---with the
aim of making individual-level inferences---should be aware of this potential
and could explore approaches for minimizing risk of ecological bias (e.g.,
\cite{wakefield2008overcoming}).}

Further, while many health outcome datasets are aggregated by county, 
some may be aggregated at a finer spatial resolution (e.g., census tract or ZIP
code) or unaggregated (i.e., point locations for each outcome). We have
published the wind model used to create this dataset as its own open-source R
package \parencite{stormwindmodel}, and it can be used to model
storm-associated winds at a finer spatial resolution; however, measurements of
other hazards cannot similarly be re-scaled through tools we provide. 

Next, we provide these data and associated software tools through R packages,
and so some experience in the R programming language is required to make full
use of them. However, R is currently a popular programming language for
environmental epidemiology, allowing the data to reach a large audience, and we
are exploring options to create a web application using the Shiny platform to
allow broader web-based access of the data \parencite{shiny2019}.  

Finally, to assess patterns and agreement for binary exposure classifications,
we have chosen one set of sensible thresholds for binary classifications based
on continuous metrics (rainfall, maximum sustained surface wind, and distance
from the storm's track), \textbf{but other thresholds would be reasonable
depending on hypothesized pathways for a given epidemiological study (Table
S1)}.  Results and conclusions would differ somewhat with other threshold
choices. We have published code for this analysis online
(\url{https://github.com/geanders/county_hurricane_exposure}), allowing other
researchers to explore other threshold choices for these analyses.

\subsection*{Conclusions}

To conduct tropical cyclone epidemiological studies that span multiple
communities and storms, it is critical to have consistent and comprehensive
measurements of exposure to storm hazards. Here we have created and shared a
dataset that provides these data for counties in the United States over
multiple years. Despite some limitations in these data, they provide a powerful
tool for expanding tropical cyclone epidemiology studies to more extensively
leverage existing administrative health data, allowing researchers to
investigate how these storms affect county-wide health risk.  Further, this
dataset provides hazard measurements that are comparable across communities and
storms, allowing epidemiological researchers to design studies to explore
how health risks are modified by characteristics of both the storms and the
communities that are hit. The data are given in an open-source format, along
with associated software tools, which allows them to be freely used and for
others to explore all associated code and to contribute additions through
platforms like GitHub.

Based on our analysis in this paper, these data are typically in agreement with
measurements from other sources of data available to characterize
storm-associated hazards (e.g., ground-based monitors, streamgages, post-storm
wind radii).  However, researchers who are planning to use the data
should explore the analyses presented in this paper to understand the strengths
and weaknesses of the data.  Further, our results indicate that county-level storm
exposure is not well-characterized by the closest distance that a storm's
central track came to a county, and that exposure to one storm hazard within a
county (e.g., severe winds) does not imply exposure to other hazards (e.g.,
excessive rainfall, flooding). As a result, it is critical that researchers
consider which storm hazards are likely on the causal pathway for the outcomes
they are studying, and to characterize storm exposure in a way that captures
those specific hazards, to avoid exposure misclassification. 
