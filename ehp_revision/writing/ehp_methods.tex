\section*{Methods and Materials}

For our study domain, we used counties in the eastern half of the \ac{US}
(Figure~\ref{fig:hurrtracks}).  We included all tropical cyclones between 1988
and 2018 that were tracked in \ac{HURDAT2} \parencite{landsea2013} and came
within~250 \si{\kilo\metre} of at least one eastern \ac{US} county
(Figure~\ref{fig:hurrtracks}). We therefore including all land-falling or
near-land Atlantic basin tropical cyclones while excluding storms that never
neared the \ac{US}.  \subsection*{Distance-based exposure metric}

We first measured how close each storm's central track came to each county. We
used tracking data from \ac{HURDAT2}, which records the storm center's position
at four standardized times for weather data collection (synoptic times),
6:00~am, 12:00~pm, 6:00~pm, and 12:00~am \ac{UTC}.  We interpolated this
position to~15-\si{\minute} intervals using natural cubic spline
interpolation~\parencite{hurricaneexposure}. At each~15-\si{\minute} interval,
we measured the distance between the storm's center and each county's
population mean center, **as of the 2010 US Decennial Census**~\parencite{countycenters}. We took the minimum distance
for each county as a measure of how close the storm came to the county over its
lifetime.  We also recorded the time when this occurred for each county so we
could later link observed data on precipitation, flooding, and tornadoes.  To
allow matching with data recorded in local time (e.g., health data), we
converted these times from \ac{UTC} to local time~\parencite{countytimezones}.

\subsection*{Rain-based exposure metric}

We based storm rainfall measurements on precipitation data files from the
\ac{NLDAS-2} re-analysis dataset, which is available for the continental
\ac{US}~\parencite{rui2013nldas}. We used data that were previously aggregated
from this re-analysis dataset to the county level by two of the coauthors
(Al-Hamdan and Crosson) for the Centers for Disease Control and Prevention's
\ac{WONDER} database \parencite{cdcwonder, alhamdan2014environmental}. To
create these aggregated data, these coauthors took hourly precipitation
measurements in the \ac{NLDAS-2} precipitation files, which are given
at~1/8\si{\degree} grid points across the continental \ac{US}, and summed them
by day for each grid point.  This created a daily precipitation total for each
grid point; these were then averaged for all grid points within a county's~1990
\ac{US} Census boundaries~\parencite{alhamdan2014environmental, cdcwonder}.
This process generated daily county-level precipitation estimates for each
continental \ac{US} county for 1988\,--\,2011 \parencite{cdcwonder}.  In this
study, we matched these county-level daily measurements by date with storm
tracks, using the date when the storm was closest to each county. Given the
location of storm-affected counties and the typical timing of tropical
cyclones, these precipitation measures primarily represent rainfall, although
occasionally they may represent snowfall or other types of precipitation.  

In the open-source data~\parencite{hurricaneexposuredata}, we provide
precipitation values at a daily resolution for each county for the period from
five days before to three days after each storm's closest approach to the
county.  In the software that we published in association with this
data~\parencite{hurricaneexposure}, we provide functionality to aggregate these
daily values to create cumulative precipitation estimates for custom windows
around the date of the storm's closest approach to the county. For example, a
user could determine precipitation only on the day the storm was closest to
each county, but could also determine the cumulative rainfall for a more
extended period (e.g., two days before to two days after the storm's
closest approach). For the analysis of binary exposure classifications in this
study, we calculated storm-associated rainfall as the sum of precipitation from
two days before the storm's closest approach to the county to one day after.  

To validate these precipitation estimates, we compared them with ground-based
observations in a subset of study counties. We selected nine sample counties
geographically spread across storm-prone regions of the eastern \ac{US} and for
which precipitation data were available from multiple ground-based stations
in the Global Historical Climatology Network throughout the study
period~\parencite{menne2012overview, rnoaa, countyweather}. These sample
counties were: Miami-Dade, FL; Harris County, TX; Mobile County, AL; Orleans
Parish, LA; Fulton County, GA; Charleston County, SC; Wake County, NC;
Baltimore County, MD; and Philadelphia County, PA. We summed daily
station-specific measurements from two days before to one day after each
storm's closest approach and then averaged these cumulative station-based
precipitation totals for each county to create a county-wide estimate of
cumulative storm-related precipitation based on ground-based monitoring. We
measured the rank correlation (Spearman's~$\rho$;~\textcite{spearman1904proof})
between storm-specific cumulative precipitation estimates for the two data
sources (\ac{NLDAS-2} re-analysis data versus ground-based monitoring) within
each sample county.

\subsection*{Wind-based exposure metric}

We created a dataset of county-level peak sustained surface wind during each
storm. To do this, we first modeled each storm's wind field to each county's
population mean center~\parencite{countycenters} at 15-minute intervals
throughout the course of the storm. We used a double exponential wind model
based on results from Willoughby~\parencite{willoughby2006parametric} to model
1-minute surface wind (10 meters above ground) at each county center, based on
inputs of the storm's forward speed, direction, and maximum wind
speed~\parencite{stormwindmodel}.  Our model incorporated the asymmetry in wind
speeds around the tropical cyclone center that results from the storm's forward
movement~\parencite{phadke2003modeling}.  From these 15-\si{\minute} interval
estimates, we identified the peak sustained surface wind in each county over
the course of each storm.

To validate these modeled county-level peak wind estimates, we compared them
to the  post-season wind radii, which have been routinely published
since 2004 \parencite{knaff2016using}. These estimate the maximum
distance from the storm's center that winds of a certain intensity extend. They
give separate estimates for four quadrants of the storm to capture asymmetry in
the storm's wind field and are based on a post-season re-analysis that
incorporates all available data (e.g., satellite data, aircraft reconnaissance, and
ground-based data if available) \parencite{knaff2016using}. Wind radii are
estimated for three thresholds of peak sustained surface wind: 64,~50, and~34
kt.  They therefore allow for the classification of counties into four
categories of peak sustained surface wind: $<$34~kt; 34\,--\,49.9~kt;
50\,--\,63.9~kt; and $\ge$64~kt. 

We interpolated the wind radii data to~15-minute increments using linear
interpolation \parencite{hurricaneexposuredata} and classified a county as
exposed to winds in a given wind speed category if its population mean center
was within~85\% of the maximum radius for that wind speed in that quadrant of
the storm. The~85\% adjustment is based on previous research that found that
this ratio helps capture average wind extents within a quadrant based on these
maximum wind radii \parencite{knaff2016using}.  We compared these wind
radii--based estimates of county-level peak sustained surface wind during a
storm with the modeled wind estimates, comparing all study storms since 2004
for which at least one study county experienced peak sustained winds of~$\ge$34
kt (based on the post-season wind radii). For each of these storms, we
calculated the percent of study counties that were classified in the same wind
category based on both data sources.

\subsection*{Flood- and tornado-based exposure metrics}

To identify flood- and tornado-based tropical cyclone exposures in \ac{US}
counties, we matched storm tracks with event listings from the National Oceanic
and Atmospheric Administration (NOAA)'s Storm Event
Database~\parencite{stormevents}. While this database has recorded storm data,
particularly tornadoes, since~1950, its coverage changed substantially in~1996
to cover more types of storm events, including flood
events~\parencite{stormevents}. We therefore only considered flood metrics of
tropical cyclone exposure for storms in~1996 and later.

For each tropical cyclone, we identified all events with event types related to
flooding (``Flood,'' ``Flash Flood,'' ``Coastal Flood'') or tornadoes
(``Tornado'') with a start date within a five-day window centered on the date
of the tropical cyclone's closest approach to the
county~\parencite{hurricaneexposuredata}. To exclude events that started near
in time to the storm but far from the storm's track, and so were likely
unrelated to the storm, we excluded any events that occurred~$\ge$500
\si{\kilo\metre} from the tropical cyclone's track. ``Flood,'' ``Flash Flood,''
and ``Tornado'' events in this database were reported by county \ac{FIPS} code
and so could be directly linked to counties.  ``Coastal Flood'' events were
reported by forecast zone; for these, the event was matched to the appropriate
county if possible using regular expression matching of listed county
names~\parencite{noaastormevents}. 

The tornado observations from this dataset form a traditional tornado event
database for the \ac{US}, and so we did not further validate
the tornado event data. It is difficult to characterize flooding at the county
level because flooding can be very localized and can be triggered by a variety
of causes. To investigate the extent to which the NOAA flood event listings
capture extremes that might be identified with other flooding data sources, we
investigated a sample of study counties, comparing the flood event data during
tropical storms with streamflow measurements at \ac{US} Geological Survey
county streamflow gages \parencite{usgsgages, countyfloods, dataRetrieval}.  
We considered nine study counties, selecting counties geographically spread
through storm-prone areas of the eastern \ac{US} and with at least one streamflow
gage reporting data during all events. The sample counties were: Baltimore County,
MD; Bergen County, NJ; Escambia County, FL; Fairfield County, CT; Fulton
County, GA; Harris County, TX; Mobile County, AL; Montgomery County, PA; and
Wake County, NC. 

For each county, we first identified all streamflow gages in the county with
complete data for Jan.~1~1996\,--\,Dec.~31,~2018. If a storm did not come
within~500~\si{\kilo\metre} of a county, it was excluded from this analysis,
but all other study storms were considered.  For each storm and county, we
summed streamflow measurements across all county gages to generate daily totals
for the five-day window around the storm's closest approach. We took the
maximum of these daily streamflow totals as a measure of the county's maximum
daily streamflow during that storm. We also calculated the percent of
streamflow gages in the county with a daily streamflow that exceeded a
threshold of flooding (the streamgage's median value for annual peak flow) on
any day during the five-day window. We investigated how these measurements
varied between storms with associated flood events listings versus storms
without an event listing for the county, to explore if storms with flood event
listings tended to be associated with higher streamflow at gages within the
county.

\subsection*{Binary storm exposure classifications}

In our open-source data, we provide continuous measurements of some of the
exposure metrics: closest distance of each storm to each county, peak sustained
surface wind at the county center, and cumulative rainfall. However,
epidemiologic studies of tropical cyclones often use a binary exposure
classification (``exposed'' versus ``unexposed'') to assess storm-related
health risks (e.g., \textcite{grabich2015, mckinney2011,
caillouet2008increase}), and so we explored patterns in storm exposure based on
binary classifications of these exposure metrics. 

Two of the exposure metrics (flood- and tornado-based) were inherently binary
in our data, since these metrics were based on whether an event was listed in
the NOAA Storm Events database.  For other exposure metrics, each county was
classified as exposed to a tropical cyclone based on whether the exposure
metric exceeded a certain threshold (Table~\ref{tab:exposuremetrics}). We
picked reasonable thresholds (e.g., the threshold for gale-force wind for 
wind exposure), but others could be used with the open-source data and its
associated software. 

For the rainfall metric, a distance constraint was also necessary, to ensure
that rainfall unrelated and far from the storm track was not misattributed to a
storm. Through exploratory analysis, we set this distance metric at~500
\si{\kilo\metre} (i.e., for a county to be classified as exposed based on
rainfall, the cumulative rainfall in the county had to be over the
75~\si{\milli\metre} threshold and the storm must have passed
within~500 \si{\kilo\metre} of the county; Table~\ref{tab:exposuremetrics}).
This distance constraint was typically large enough to capture storm-related
rain.  However, data users should note that in rare cases---for example,
exceptionally large storms (e.g., Hurricane Ike in 2008) or storms for which
storm tracking was stopped at extratropical transition (e.g., Tropical Storm
Lee in 2011)---some storm-related rain exposures may be missed because of this
distance constraint (Figure~S1). This distance constraint can be customized
using the software published in association with the open-source
data~\parencite{hurricaneexposure}.

We characterized patterns in county-level exposure in the eastern \ac{US} for
exposure assessment based on each of the measured tropical cyclone hazards
(cumulative precipitation, peak sustained wind, flooding, and tornadoes).
Depending on available exposure data, this assessment included some or all of
the period from~1988 to~2018 (Table 1). For each binary metric of exposure to a
tropical cyclone hazard, we first summed the total number of county-level
exposures over available years and mapped patterns in
these exposures. 

Finally, we investigated how well exposure assessment agreed across 
these metrics. For each storm we investigated the degree to which the 
set of counties assessed as exposed based on one metric overlapped with the set
assessed as exposed based on each other metric, including a metric that used
distance of a county from the storm track as a proxy for exposure to storm hazards. 
We calculated the within-storm
Jaccard index ($J_s$)~\parencite{jaccard1901distribution, jaccard1908nouvelles} between
each pair of exposure metrics. This measures similarity
between two metrics ($X_{1,s}$ and~$X_{2,s}$) for tropical cyclone~$s$ as the
proportion of counties in which both of the metrics classify the county as
exposed out of all counties classified as exposed by at least one of the
metrics:

\begin{equation} 
J_s = \frac{X_{1,s} \cap X_{2,s}}{X_{1,s} \cup X_{2,s}}
\end{equation}

\noindent This metric can range from~0, in the case of no overlap between the
counties classified as exposed based on the two metrics, to~1, if
the two metrics classify an identical set of counties as exposed to the
tropical cyclone. We calculated these values for all study storms that affected
$\ge$100 study counties (based on at least one exposure metric) during the
years when all exposure data were available (1996\,--\,2011).

\subsection*{Data and code access}

We have posted most study data as an open-source R package
\parencite{hurricaneexposuredata}. We have posted remaining data and code at
\url{https://github.com/geanders/county_hurricane_exposure}.
