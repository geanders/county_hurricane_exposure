Storms are displayed within
clusters that have similar patterns in county-level exposure agreement for
metric pairs, based on hierarchical clustering using the complete link
method~\parencite{murtagh2012algorithms} (i.e., storms in the same cluster tend
to have similar patterns for the pairwise strength of agreement among metrics);
columns are also ordered based on hierarchical clustering. 
Maps are
available showing the counties identified as exposed under each of five metrics
for the widest-extent storm in each cluster: Hurricane Ivan in~2004
(Figure~\ref{fig:ivanexposure}) and Hurricanes Floyd in~1999, Lee in~2011, Cindy
in~2005, and Katrina in~2005 (Figure~S4).

For most tropical cyclones,
distance-based county-level exposure assessment was, at best, in moderate
agreement with exposure assessments based on the four hazard-based metrics.

\subsection*{Case study}

We conducted a case study comparing estimates of physical exposure to tropical
cyclones based on each exposure metric.  Figure~\ref{fig:topelecdependexposure}
shows the study counties with the highest expected rates of physical exposure
to tropical cyclones among electricity-dependent Medicare beneficiaries based
on each metric. A few counties (Miami-Dade County, FL; Harris County, TX) were
ranked in the top three in expected yearly rate of physical exposure under
almost all exposure metrics (Figure~\ref{fig:topelecdependexposure}). However,
there were key differences across exposure metrics. For example, highly-ranked
based on wind- and tornado-based metrics were in Florida, while over half of
the highly-ranked counties based on the flood-based metric were in Mid-Atlantic
states (PA, NY).


