Much of the
epidemiological research to date on the health impacts of tropical cyclones has, like these studies,
focused on a single storm (e.g., [refs]), or has relied on death and injury
counts for which a link to the storm was identifiable through mention on a
death certificate or other information about the death [refs], an approach that
may miss many of the health impacts of storms, particularly from non-accidental
causes, as evidenced with Hurricane Maria [ref for this?]. 

To date, epidemiological research has largely focuses on single storms, as with
these studies, or has leveraged fatality counts from diaster surveillance
activities. Single-storm studies clarify the health impacts of a specific
storm, but are limited in extrapolating what risks might be expected under
future storms or which characteristics of a storm modify its likely health
impacts.  Studies based on disaster surveillance have helped in identifying
large-scale patterns in the health impacts of tropical cyclones, especially for
accidental deaths and injuries. However, disaster surveillance can vary from
storm to storm or across agencies estimating fatality and injury tolls, and
this inconsistency makes counts from disaster surveillance difficult to
compare. Further, disaster surveillance activities are typically not 
designed in a way that captures increased risks for non-accidental outcomes. 
For example, details on the circumstances of death are often needed for 
a coroner or medical examiner to note a connection with a disaster on a 
death certificate, but the types of deaths that trigger death scene investigations
are largely accidental causes (e.g., drowning, carbon monoxide poisoning, 
motor vehicle accident, electrocution).    

A number of studies have investigated how tropical cyclones affect human
health. These have largely taken two paths. First, many studies focus on a
specific storm, often one that was severe with clear human impacts. For
example, studies have explored how rates of ... changed during and after
Hurricane Sandy, which hit New York and New Jersey in 2012. 
Other studies have explored the health impacts of Hurricane Katrina in 
Mississippi and Louisiana in 2005. These studies show that tropical cyclones
can cause important health impacts from a variety of causes, but they say 
little about how less severe but more frequent tropical cyclone exposures 
might impact human health.  

There is a second main path for tropical cyclone epidemiology studies to date.
This has been to leverage data from disaster surveillance activities, in which
[state officials], aid organizations, or [FEMA] estimate the death and injury
tolls from a disaster. These estimated health impacts are recorded across many
tropical cyclones (and other storm or disaster events) across databases like
the NOAA Storm Events database, [EM-DAT], and [SHELDUS]. Studies that use this
data can show large-scale patterns over long time scales. For example, studies
in this vein have shown that average death tolls have decreased dramatically
over the last century for the community where a storm makes landfall, and that
storms can cause numerous drowning deaths in communities well inland from
landfall. However, these studies have two key limitations. First, disaster
surveillance does not have clear standards for counting deaths and injuries
during a disaster---in fact, one recent review found that estimates of death
tolls can vary widely for the same storm across the agencies and organizations
attempting to count deaths. This review found that the estimated death tolls
for Hurricane Ike, for example, included an estimate of [x] deaths from FEMA,
[x] from ..., and [x] as recorded in NOAA's Storm Events database. Similar
inconsistency is likely in estimates made for different storms, particularly
ones that are far apart in time or space. This inconsistency complicates
attempts to link storm characteristics with the type and scale of health
impacts that the storm is likely to cause. The second limitation is that
studies that are based on disaster surveillance counts are likely to miss many
of the impacts from non-accidental causes. For deaths, for example, a medical
examiner or coroner may need to have details on the circumstances and
environment of the death to be able to add a note linking a death to a disaster
on a death certificate, and this death certificate information is a key source
in generating disaster surveillance death toll estimates. However, death scene
investigations are typically only triggered for certain kinds of deaths, mostly
those from accidental causes like drowning, motor vehicle accidents,
electrocution, carbon monoxide poisonings, suicides, and homicides. If this
risk of death from common, non-accidental causes is elevated, this association
is likely not fully captured by disaster surveillance activities. 

Given these limitations in the existing literature on tropical cyclone
epidemiology, much can be learned by investigating how the risk of health
outcomes change in the period surrounding a storm's landfall through examining
patterns in administrative data, like all-cause mortality counts across a
community, or insurance claims.  This analysis can help estimate how storms
impact non-accidental mortality and morbidity and, for accidental causes, can
provide an estimate of impacts that may be more consistent across time and
space that estimated tolls from disaster surveillance. Conversely, multi-storm
studies using administrative data will lack the ability to explore the details
of specific deaths, since they can estimate the community-wide change in risk
but not idea which specific deaths were the ones contributing to that increased
risk. Studies based on community-wide administrative data should be considered
a complement to continuing research based on thorough studies of single storms
and on studies that leverage data from disaster surveillance activities. In
fact, one study examined how [heat waves]. 

For other types of ambient exposures, administrative
data or vital data records have allowed the exploration of changing rates in a
variety of health outcomes, accidental and non-accidental, in association with
changes in ambient exposures.

However, tropical cyclones are multi-hazards events, and a storm's hazards are
not perfectly aligned with distance from the storm's track. Recent storms have
demonstrated the wide variation possible in these hazards; Hurricanes
Michael and Maria brought extreme storm-related winds, for
example, while Hurricanes Florence and Harvey were more notable for extreme
rain and flooding [refs]. Depending on the health outcome considered, some
storm hazards might be more plausible as a risk factor than others. Respiratory
outcomes in the months after a storm, for example, might be increased through a
pathway of heightened mold exposure from flooding, while heat stroke and other
heat-related outcomes in the days immediately after a storm might follow from a
pathway of extreme winds causing power outages, resulting in loss of air
conditioning [refs]. 

Geographical patterns can differ for these different
storm hazards, and so exposure assessment based on one hazard could create
exposure misclassification if the pathway for health risks is through a
different storm hazard.   

Tropical cyclones, including hurricanes, can have disastrous impacts in the
\ac{US}. A tropical cyclone's high winds can bring health risks
and property damage through structural damage of houses and other buildings,
falling trees, and wind-borne debris~\parencite{rappaport2000} and cause power
outages~\parencite{liu2005, han2009}, which introduce a number of threats to human
and ecological health, including water quality risks if the outage affects
wastewater treatment plants~\parencite{mallin2006}.  Other risks can exist without
severe wind; for example, one study found that most of the direct
hurricane-related deaths in the \ac{US} between~1970 and~1999 occurred in cases
when wind was below hurricane strength, including for Tropical Storms Charley
in~1998 and Alberto in~1994~\parencite{rappaport2000}.  Tropical cyclones can
produce excessive rain, especially in certain topographies (e.g., near
mountains), so counties well inland sometimes experience more extreme rain than
coastal counties. Flood risks from tropical cyclones can result from this rain,
although the two risks are not perfectly correlated~\parencite{chen2015}. In the
\ac{US}, over half of hurricane-related direct deaths from~1970 to~1999 from
Atlantic basin storms were a result of freshwater flooding~\parencite{rappaport2000}. 
Flooding can also degrade water quality~\parencite{mallin2006}, which can threaten both 
human and ecological health.

To measure tropical cyclone exposure for either investigations of exposure
patterns or for estimation of tropical cyclone risks and impacts, including
those related to human health, studies have varied widely in how they determine
tropical cyclone exposure (some examples shown in Figure~S1). While some
studies have measured exposure based on specific hazards of the tropical
cyclone (e.g., wind, rain), many have used distance from the tropical cyclone's
central track as a surrogate in identifying exposure to the hazards of a
tropical cyclone (e.g.,~\textcite{czajkowski2011, tansel2010, kinney2008,
caillouet2008increase}).  

Distance from the tropical cyclone's central track can be easily measured using
widely-available hurricane tracking data. There are, however, a number of
limitations to using distance to assess exposure to tropical cyclones, and

Geographical patterns likely differ for these different storm hazards, and so
exposure assessment based on one hazard could create exposure misclassification
if the pathway for health risks is through a different storm hazard.  

There are reasons to suspect that
exposure classifications based on tropical cyclone hazards like wind and rain
may disagree with distance-based classifications, based only on distance from
the storm's path. Further, an assessment based on one storm hazard (e.g., wind)
might disagree with assessments based on other hazards (e.g., rain or
flooding).  

[Previous studies,] which found tropical cyclones can increase risk not only
for accidental health outcomes (e.g., drowning, crushing deaths and injuries,
electrocution, carbon monoxide poisoning [refs]) but also for non-accidental,
non-communicable diseases [refs].  

, including Sandy in 2012 (e.g., \parencite{swerdel2014}) and Katrina
in 2005 (e.g., \parencite{burton2009health}), 

A recent review highlights the need to supplement these single-storm studies
with multi-year, multi-storm studies, to identify patterns that are common
across multiple storms [ref]. While multi-year studies of hurriance mortality
impacts have been conducted using disaster surveillance data, studies based on
administrative data could identify risks for non-accidental mortality and
morbidity that those studies likely undercount [ref]. Multi-storm studies with
administrative data require exposure assessment that is consistent across time
and space, allowing the administrative data to be linked to disaster exposure
[ref]. Key to expanding tropical cyclone epidemiology are therefore exposure
assessment methods that appropriately capture hazards of the storm that are
relevant to human health, methods that can be applied consistently across
multiple storms, locations, and years. 

If a study misclassifies exposure to the hazard or hazards that cause the
health risk being studied, the study will generate biased estimates of tropical
cyclone risks and impacts. Further, when different studies use different
methods to assess exposure to tropical cyclones, their results are difficult to
meaningfully compare and aggregate. If different methods identify
similar sets of communities as ``exposed'',  these concerns are less serious.
However, if different methods differ substantially in which communities they
identify as ``exposed'', it makes it very important that epidemiological
studies are thoughtful in how they assess exposure.

 and
measured how well exposure classification agreed across these five metrics in
terms of classifying specific counties as exposed to a tropical cyclone.  

