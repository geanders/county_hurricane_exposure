\documentclass[fleqn,10pt,lineno]{olplainarticle}
% Use option lineno for line numbers

\graphicspath{{../figures/}}

\usepackage[implicit=false]{hyperref}
\usepackage{acronym}

\title{Assessing United States county-level exposure for research on tropical cyclones and human health}

\author[a,1]{G. Brooke Anderson} 
\author[a,b]{Joshua Ferreri}
\author[c]{Mohammad Al-Hamdan} 
\author[c]{William Crosson} 
\author[d]{Andrea Schumacher} 
\author[e]{Seth Guikema} 
\author[f]{Steven Quiring} 
\author[g]{Dirk Eddelbuettel} 
\author[a]{Meilin Yan} 
\author[h]{Roger D. Peng}

\affil[a]{Department of Environmental \& Radiological Health Sciences, Colorado 
  State University, Fort Collins, CO, 80523} 
\affil[b]{University of Colorado Denver School of Medicine, Aurora, CO, 80045} 
\affil[c]{Universities Space Research Association, NASA Marshall Space Flight 
  Center, Huntsville, AL, 35805}
\affil[d]{Cooperative Institute for Research in the Atmosphere, Colorado State
  University, Fort Collins, CO, 80523} 
\affil[e]{Department of Industrial and Operations Engineering, University of 
  Michigan, Ann Arbor, MI, 48109}
\affil[f]{Department of Geography, Ohio State University, Columbus, OH, 43210}
\affil[g]{Debian and R Projects; Department of Statistics, University of
  Illinois at Urbana-Champaign, Champaign, IL, 61820} 
\affil[h]{Department of Biostatistics, Johns Hopkins Bloomberg School of Public 
  Health, Baltimore, MD, 21205}

\keywords{Hurricanes, Tropical cyclones, Disaster impacts, Exposure assessment}

\begin{abstract} 
  \begin{abstract}

\noindent \textbf{Background:} 
        Tropical cyclone epidemiology can be 
	advanced through county-level exposure assessment methods that are comprehensive and 
	consistent across space and time, facilitating multi-year, multi-storm studies. 
	Further, an understanding of 
	patterns in and between exposure metrics that are based on specific hazards of the storm can help in 
	designing tropical cyclone epidemiological studies.\\ 
\textbf{Objectives:} (1)~Provide an open-source dataset for tropical
	cyclone exposure assessment for epidemiological research and
	(2)~investigate patterns and agreement between county-level assessments of tropical cyclone
	exposure based on different storm hazards. \\ 
\textbf{Methods:} We created an open-source dataset with county-level data on
	exposure to four tropical cyclone hazards: peak sustained wind,
	rainfall, flooding, and tornadoes. The data cover all eastern \ac{US} counties 
	for all land-falling or near-land
	Atlantic basin storms, covering~1996\,--\,2011 for all
	metrics and up to~1988\,--\,2015 for specific metrics. 	
	We validated 
	measurements against other data sources and investigated patterns
	and agreement among binary exposure classifications based on these
	metrics, as well as compared them to use of distance from the storm's track as a proxy for
	exposure.\\ 
\textbf{Results:} We describe how we developed this dataset and explore how its
	measurements compare to other data that could be used to characterize
	tropical cyclone hazards for epidemiological research. Measurements in
	the dataset created here typically agreed with data from other sources,
	and we present and discuss areas of disagreement and other caveats for
	the data. Tropical cyclones typically brought different hazards to different
	counties, and so agreement was usually low between distance-based tropical
	cyclone exposure assessment and each hazard-specific metric, as well as 
	between pairs of hazard-specific metrics. \\
\textbf{Discussion:} Our results provide a multi-hazard dataset that can be
	leveraged for epidemiologic research on tropical cyclones, as well as 
	insights that can inform the design of tropical cyclone epidemiological 
	studies.
\end{abstract}

\end{abstract}

\begin{document}

\flushbottom 
\maketitle 
\thispagestyle{empty}

\section*{Introduction}

\acresetall

Recently, several major tropical cyclones have hit \ac{US} cities, including
Hurricanes Harvey, Irma, and Maria in 2017 \parencite{blake20182017} and
Hurricanes Florence and Michael in 2018 \parencite{avila20192018}. Researchers
are exploring how these storms impacted human health (e.g.,
\parencite{santos2018use, rivera2018estimating, santos2018differential,
grineski2019impact, issa2018deaths, tanz2019notes, paul2019brief}), including
through projects funded through \ac{NIH} Rapid Response grants
\parencite{nihreporter}. These projects will add to evidence from previous
hurricanes, including Sandy in 2012 (e.g., \parencite{swerdel2014}) and Katrina
in 2005 (e.g., \parencite{burton2009health}), characterizing the health impacts
of hurricanes. 

A recent review highlights the need to supplement these single-storm studies
with multi-year, multi-storm studies, to identify patterns that are common
across multiple storms [ref]. While multi-year studies of hurriance mortality
impacts have been conducted using disaster surveillance data, studies based on
administrative data could identify risks for non-accidental mortality and
morbidity that those studies likely undercount [ref]. Multi-storm studies with
administrative data require exposure assessment that is consistent across time
and space, allowing the administrative data to be linked to disaster exposure
[ref]. Key to expanding tropical cyclone epidemiology are therefore exposure
assessment methods that appropriately capture hazards of the storm that are
relevant to human health, methods that can be applied consistently across
multiple storms, locations, and years. 

The \ac{NHC} publishes a ``Best Tracks'' dataset that is considered the gold
standard for Atlantic-basin tropical cyclone tracks.  It records the central
position of a tropical cyclone every six hours, as well as the storm's
\underline{minimum pressure} and \underline{central maximum sustained winds}. This data is openly
available through the \ac{HURDAT2}, a post-storm assessment conducted
by the \ac{US} \ac{NHC} that incorporates data from a variety of sources,
including satellite data and, when available, aircraft reconnaissance
data~\parencite{landsea2013, jarvinen1988}. With this data, it is
straightforward to measure whether a community was in the direct path, or
within a certain distance of the path, of a storm. Some studies have indeed
used this data to assess tropical cyclone exposure based on how near the
storm's central track came to the county, either to assess average exposure
patterns to tropical storms [refs] or to characterize exposure for health
studies [refs].

When epidemiological studies assess exposure to a tropical cyclone based on 
how close the storm came to a community, however, they may misclassify exposure. 
Tropical cyclones vary dramatically in size: \ac{US} tropical
cyclones have been observed with radii to maximum winds as small
as~20~\si{\kilo\metre} and as large
as~200~\si{\kilo\metre}~\parencite{mallin2006, quiring2011variations}.  While a
number of tropical cyclone hazards are strongly associated with distance from
the tropical cyclone's center (e.g., wind and, at the coast, storm surge and
waves~\parencite{rappaport2000, kruk2010}), other hazards like heavy rainfall,
floods, and tornadoes can occur well away from the tropical cyclone's central
track~\parencite{rappaport2000, atallah2007, moore2012}.  For example, fatal
tropical cyclone tornadoes, which were linked to over~300 deaths in the \ac{US}
between~1995 and~2009, most often occur~200\,--\,500~\si{\kilo\metre} from the
tropical cyclone's center~\parencite{moore2012}.  

Further, when studies use distance from the storm's track to assess exposure,
they often use an equal buffer distance on each side of the tropical cyclone
track (e.g.,~\cite{czajkowski2011, grabich2015, grabich2016, zandbergen2009,
tansel2010}). However, the forces of a tropical cyclone tend to be distributed
around the center in a non-symmetrical way. Extreme winds are more common to
the right of the track, where counter-clockwise cyclonic winds move in concert
with the tropical cyclone's forward motion~\parencite{halverson2015}, and the
fatal tornadoes associated with \ac{US} tropical cyclones between~1995 and~2009
occurred almost exclusively to the right of the tropical cyclone's track,
mostly in the right front quadrant of the tropical
cyclone~\parencite{moore2012}. Rain, conversely, is often heaviest to the left
of the tropical cyclone's track, especially when the tropical cyclone interacts
with other weather systems~\parencite{atallah2003, atallah2007,
zhu2013variations} or undergoes an extratropical
transition~\parencite{elsberry2002}.

If a study misclassifies exposure to the hazard or hazards that cause the
health risk being studied, the study will generate biased estimates of tropical
cyclone risks and impacts. Further, when different studies use different
methods to assess exposure to tropical cyclones, their results are difficult to
meaningfully compare and aggregate. If different methods identify
similar sets of communities as ``exposed'',  these concerns are less serious.
However, if different methods differ substantially in which communities they
identify as ``exposed'', it makes it very important that epidemiological
studies are thoughtful in how they assess exposure.

To provide open-source data on tropical cyclone exposures to other researchers,
we developed a dataset of county-level exposure to tropical cyclones in all
eastern \ac{US} counties for five different metrics
(Table~\ref{tab:exposuremetrics}): (1)~distance to tropical cyclone track;
(2)~maximum sustained wind speed; (3)~cumulative rainfall; (4)~flood events;
and (5)~tornado events~\parencite{hurricaneexposure}. This data is provided at
the county level since data on many potential impacts are available at
county-level aggregations (e.g., direct hurricane-related
deaths~\parencite{czajkowski2011}; birth outcomes~\parencite{grabich2015,
grabich2016}; autism prevalence~\parencite{kinney2008}) and since decisions and
policies to prepare for, and respond to, tropical cyclones are often undertaken
at the county level~\parencite{zandbergen2009, rappaport2000}. Further, we
explored patterns in these exposure assessments across storm hazards and
measured how well exposure classification agreed across these five metrics in
terms of classifying specific counties as exposed to a tropical cyclone.  


\begin{table}%[tbhp] 
\centering 
\caption{Exposure metrics considered to assess county-level exposure to 
tropical cyclones}
\begin{tabular}{p{0.9cm}p{2.5cm}p{9cm}} 
Metric & Available & Criteria for exposure \\ \midrule 
Distance & 1988--2015 & County population mean center within 100 kilometers of storm track \\ 
Rain & 1988--2011 & County cumulative rainfall of 75 millimeters or more over the period from two days before to one day after the storm's closest approach and county population mean center within 500 kilometers of the storm track \\ 
Wind & 1988--2015 & Modeled maximum sustained wind speed at the county's population mean center 15 meters per second or higher during the storm\\ 
Flood & 1996--2015 & Flood event listed in the National Oceanic and Atmospheric Administration (NOAA) Storm Events database for the county with a start date within two days of the storm's closest approach and county population mean center within 500 kilometers of the storm track \\
Tornado & 1996--2015 & Tornado event listed NOAA Storm Events database for the county with a start date within two days of the storm's closest approach and county population mean center within 500 kilometers of the storm track\\
\bottomrule 
\end{tabular} 
\label{tab:exposuremetrics} 
\end{table}

\section*{Methods and Materials}

This study covered all counties in the eastern half of the \ac{US}
(Figure~\ref{fig:hurrtracks}), as of the~2010 \ac{US} Decennial Census. The
study covered all tracked land-falling or near-land Atlantic-basin tropical
cyclones between~1988 and~2015, considering all 136 storms in \ac{HURDAT2}
\parencite{landsea2013} that came within~250 \si{\kilo\metre} of at least one
eastern \ac{US} county in this period (Figure~\ref{fig:hurrtracks}). 

\begin{comment}
These data typically give measurements of
the tropical cyclone center's location [and central pressure / maximum
windspeed?] at~6-\si{\hour} intervals at synoptic times (i.e., 6:00~am,
12:00~pm, 6:00~pm, and 12:00~am \ac{UTC}); some landfalling tropical cyclones
have an additional observation at the time of landfall~\parencite{landsea2013}.
\end{comment}

\subsection*{Distance-based exposure metric}

We first measured how close each study storm came to each study county. We
started with tracking data from \ac{HURDAT2}, which records the storm's
position every~6 \si{\hour}, and interpolated to~15-\si{\minute} intervals
using natural cubic splines~\parencite{hurricaneexposure}. We then measured the
distance between the storm's position at each~15-\si{\minute} interval and each
county's population mean center, as assessed by the 2010 Decennial
Census~\parencite{countycenters}, using the Great Circle (WGS84 ellipsoid)
method~\parencite{bivand2013applied}. These measurements captured the distance
between the storm's central track and each study county throughout the period
when the storm was tracked. The closest distance between the storm and each
county was recorded as the distance of the storm's closest approach to the
county. We also recorded the time storm came closest to each county, for use in
assessing rain-, flood-, and tornado-based metrics where observed data on the
hazards needed to be matched by time to storm events.  To allow matching with
data based on local time, times were converted from \ac{UTC} to local
time~\parencite{countytimezones}.

\subsection*{Rain-based exposure metric}

To estimate storm-associated rainfall, we used precipitation data from a
re-analysis dataset, as this data was available for every county and every day
through a continuous period of the study
(1988\,--\,2011)~\parencite{alhamdan2014environmental, cdcwonder}.  By
comparison, observations from ground-based monitoring networks had missing
values spatially (i.e., for some study counties), temporally (for some days),
or both.  

We estimated daily rainfall for each study county by aggregating
hourly~1/8\si{\degree} gridded reanalysis data from the \ac{NLDAS-2}
precipitation data files~\parencite{rui2013nldas}. These data integrate
satellite-based and land-based monitoring, applying a land-surface model to
create a reanalysis dataset that is spatially and temporally complete across
the continental \ac{US}~\parencite{rui2013nldas, alhamdan2014environmental}. To
aggregate to county-level, the hourly data at each grid point were summed to
create a daily rainfall total, and these grid point rainfall totals were then
averaged across all grid points within a county's~1990 \ac{US} Census
boundaries~\parencite{alhamdan2014environmental, cdcwonder}. These daily
county-level estimates were then matched temporally with storm tracks, using
the date when each storm was closest to each county. The cumulative
storm-associated rainfall was then calculated as the sum of rainfall from two
days before the storm's closest approach to a county to one day after, as 
storm-related rains often precede the passing of the storm's center. 

To validate these rainfall estimates, we investigated a subset of counties that
also had available ground-based precipitation observations. We selected nine
sample counties geographically spread across storm-prone regions of the eastern
\ac{US} and for which precipitation data were available from multiple stations
throughout the study period: Miami-Dade, FL; Harris County, TX; Mobile County,
AL; Orleans Parish, LA; Fulton County, GA; Charleston County, SC; Wake County,
NC; Baltimore County, MD; and Philadelphia County, PA. We collected
ground-based observations of precipitation data from all stations in the county
with available daily data in the Global Historical Climatology Network
throughout 1988\,--\,2011~\parencite{menne2012overview, rnoaa, countyweather}.
We summed daily station-specific measurments for two days before to one day
after each storm's closest approach and then averaged these cumulative
station-based precipitation totals for each county as a county-wide estimate of
cumulative storm-related precipitation. We measured the rank correlation
(Spearman's~$\rho$~\parencite{spearman1904proof}) between storm-specific
cumulative precipitation estimates for the two data sources within each sample
county.

\subsection*{Wind-based exposure metric}

Ground-based observations of wind speed are problematic as a metric of exposure
to tropical cyclones, as instruments often fail at high wind speed, while
reanalysis data, often available at hourly or higher resolution, can lack a
fine enough temporal resolution to capture wind extremes associated with a
tropical cyclone. Therefore, to create a dataset of county-level sustained
winds during historical tropical cyclones, we modeled maximum sustained wind
speeds at each county's population mean center~\parencite{countycenters}, using
a wind speed model based on results from
Willoughby~\parencite{willoughby2006parametric}. For each storm, we used the
interpolated storm tracks generated for the distance metric and, for
each~15-\si{\minute} increment, we modeled maximum ground-level sustained wind
speed at each county center using a double exponential wind speed model with
inputs for the storm's forward speed, direction, and central wind
speed~\parencite{willoughby2006parametric, stormwindmodel}. This model
incorporated asymmetry in wind speeds around the tropical cyclone center that
results from the storm's forward movement~\parencite{phadke2003modeling,
stormwindmodel}. We then determined the maximum value of this estimate over the
storm tracking period for each study county to determine the maximum sustained
surface wind speed in each county during each storm.

To validate these modeled county-level wind speed estimates, we compared them
to county-level maximum sustained surface wind speeds based on the wind radii
estimates in \ac{HURDAT2}, which are based on \ldots. The wind radii values
have been included in \ac{HURDAT2} since 2004 [?] and estimate the distance
from the storm's center to the furthest point with winds at or above winds of a
certain speed in each four quadrants of the storm. They give estimates of these
maximum radii for three thresholds of maximum surface wind speeds: 64,~50,
and~34 \si{\knot}.  They therefore allow for the classification of counties
into four categories of maximum sustained wind speeds: $<$34~\si{\knot};
34\,--\,49.9~\si{\knot}; 50\,--\,63.9~\si{\knot}; $\ge$64~\si{\knot}. We
interpolated this data to~15-\si{\minute} increments and classified a county as
exposed to winds in a given wind speed category if its population mean center
was within~85\% of the maximum radius for that wind speed in its quadrant of
the storm [clarify why 85\%]. We compared these wind radii-based estimates with
the modeled wind speed estimates for the study storms since 2004 for which at
least one study county had a sustained wind of~$\ge$34 \si{\knot} (based on the
\ac{HURDAT2} wind radii). For each of these storms, we calculated the percent
of study counties that were classified in the same wind speed category.

\subsection*{Flood- and tornado-based exposure metrics}

To identify flood- and tornado-based tropical cyclone exposures in \ac{US}
counties, we used event listings from the National Oceanic and Atmospheric
Administration (NOAA)'s Storm Event Database~\parencite{stormevents}. For each
tropical cyclone, we identified all events with event types related to flooding
(``Flood", ``Flash Flood", ``Coastal Flood") and tornadoes (``Tornado") and
that occurred in a county within~500 \si{\kilo\metre} of the tropical cyclone's
track and with a start date within a five-day window centered on the date of
the tropical cyclone's closest approach to the
county~\parencite{hurricaneexposuredata}. ``Flood", ``Flash Flood" and
``Tornado" events in this database were reported by county \ac{FIPS} code and
so could be directly linked to counties.  ``Coastal Flood" events were reported
by forecast zone; for these, the event was matched to the appropriate county if
possible using regular expression matching of listed county
names~\parencite{noaastormevents}. While this database has recorded storm data
since~1950, its coverage changed substantially in~1996 to cover a wider variety
of storm events~\parencite{stormevents}. We therefore only considered these
metrics of hurricane exposure for tropical cyclones in~1996 and later.

The tornado observations from this dataset form the traditional tornado event
database for the \ac{US}, and so we did not conduct analyses to validate the
tornado event data. It is difficult to characterize flooding at the county
level because flooding can be very localized and can be triggered by a variety
of causes. To investigate the extent to which the NOAA flood event data
captures extremes that might be identified with other flooding data sources, we
investigated a sample of study counties, comparing the flood event data during
tropical storms with streamflow measurements at \ac{US} Geological Survey
county streamflow gages \parencite{usgsgages, countyfloods, dataRetrieval}.  

We considered nine study counties, selecting counties geographically spread
through storm-prone areas of the eastern \ac{US} and with multiple streamflow
gages reporting data during events (Baltimore County, MD; Bergen County, NJ;
Escambia County, FL; Fairfield County, CT; Fulton County, GA; Harris County,
TX; Mobile County, AL; Montgomery County, PL; and Wake County, NC). For each
county, we first identified all streamflow gages in the county that had
complete data for Jan.~1~1996\,--\,Dec.~31,~2015. If a storm did not come
within~500~\si{\kilo\metre} of a county, it was excluded from this analysis,
but all other study storms were considered. For each storm and county, we
summed daily total streamflow across all streamgages for each day in the
five-day window around the storm's closest approach. We took the maximum daily
streamflow total during those five days as a measure of the county's maximum
streamflow during that storm. We also calculated the percent of streamflow
gages in the county with a daily streamflow that exceeded a threshold of
flooding (the streamgage's median value for annual peak
flow~\parencite{countyfloods}) on any day during the five-day window. We
investigate how these measurements varied between storms with associated flood
events in the NOAA Storm Events data versus storms without an event listing, to
explore if storms with flood event listings tended to be  associated with
higher streamflows at gages within the county.

\subsection*{Binary storm exposure classifications based on different metrics}

In our open-source dataset, we provide the closest distance of each storm to
each county, as well as county-level storm-related winds and rainfall, as
continuous metrics. However, epidemiologic studies of tropical cyclones often
compare ``exposed'' versus ``unexposed'' communities to assess storm-related
health risks [refs], and so we also used these continuous metrics to classify
counties as ``exposed'' or ``unexposed'' and explored patterns in storm
exposure based on these binary clasifications. 

Two of the exposure metrics (flood- and tornado-based) were inherently binary,
since these metrics were based on whether an event was listed in the NOAA Storm
Events database.  For the other exposure metrics, each county was classified as
exposed to a tropical cyclone based on whether the exposure metric exceeded a
certain threshold (Table~\ref{tab:exposuremetrics}). For the rainfall metric, a
distance constraint was also necessary, to ensure that rains unrelated and far
from the storm track were not misattributed to a storm. Through exploratory
analysis, we set this distance metric at~500 \si{\kilo\metre} (i.e., for a
county to be classified as exposed based on rainfall, the cumulative rainfall
had to be over 75 \si{\milli\metre} and the storm must have passed within~500
\si{\kilo\metre} of the county; Table~\ref{tab:exposuremetrics}). We set this
distance constraint at a value that was typically large enough to capture
storm-related rain. However, data users should note that in rare examples of
exceptionally large storms (e.g., Hurricane Ike in 2008) or storms for which
storm tracking was stopped at extratropical transition (e.g., Tropical Storm
Lee in 2011), some storm-related rains may be missed because of this distance
constraint (Figure~S7). 

We assessed patterns in county-level tropical cyclone exposure in the eastern
\ac{US} based on each exposure metric. Depending on available exposure data,
this assessment included some or all of the period from~1988 to~2015 (Table 1).
For each binary metric of tropical cyclone exposure (Table 1), we first summed
the total number of county-level exposures over available years for each
exposure metric and mapped patterns in these exposures. We next investigated
agreement between storm exposure classifications based on different exposure
metrics. We calculated the within-storm Jaccard
index~\parencite{jaccard1901distribution, jaccard1908nouvelles} between each
pair of exposure metrics. The Jaccard index~($J_s$) measures similarity between
two metrics ($X_{1,s}$ and~$X_{2,s}$) for tropical cyclone~$s$ as the
proportion of counties in which both of the metrics classify the county as
exposed out of all counties classified as exposed by at least one of the
metrics:

\begin{equation} 
J_s = \frac{X_{1,s} \cap X_{2,s}}{X_{1,s} \cup X_{2,s}}
\end{equation}

\noindent This metric can range from~0, in the case of no overlap between the
counties classified as exposed based on the two metrics, to~1, in the case that
the two metrics classify exactly the same set of counties as exposed to the
tropical cyclone. We measured these values for all study storms that affected a
large number of eastern \ac{US} counties (250 or more of the study counties
classified as exposed by at least one of the exposure metrics) during the years
when all exposure data were available (1996\,--\,2011).

\subsection*{Case study: Physical exposure of electricity-dependent Medicare
beneficiaries to tropical cyclones}

Disasters' societal impacts depend both on the geophysical forces of the
disaster and on the vulnerability of those living in the affected geographical
areas~\parencite{chakraborty2005population, anderson2003community,
cutter1996vulnerability}. As a case study, we explored how differences in
tropical cyclone exposure assessments across different exposure metrics might
influence estimates of physical exposures of a susceptible population to
tropical cyclones. We calculated expected average physical exposures among
electricity-dependent Medicare beneficiaries in eastern \ac{US} counties to
tropical cyclones based on each exposure metric. This subpopulation was
selected since it may be particularly susceptible to health impacts from
tropical cyclone exposure, especially through the pathway of storm-associated
power outages and evacuations. 

We collected data on the number of electricity-dependent Medicare beneficiaries
in each study county from the \ac{US} Department of Health \& Human Service's
emPower Map~2.0~\parencite{empower}. We then calculated the physical exposure
of the electricity-dependent Medicare population in each county, based on
tropical cyclone assessments using each hazard metric,
following~\parencite{peduzzi2009assessing}:

\begin{equation}
E_c = F_c * P_c
\end{equation}

\noindent where~$E_c$ is the average yearly physical exposure among
electricity-dependent Medicare beneficiaries in county~$c$ to tropical cyclone
exposures based on a given metric,~$F_c$ is the estimated yearly expected
frequency of tropical cyclone exposures in county~$c$ based on that metric,
and~$P_c$ is the electricity-dependent Medicare population in
county~$c$, as of July~2017. 

\begin{comment}
When combined with estimates of vulnerability of a
population to a natural hazard, such measurements of physical exposure can be
used to calculate risk of human losses from the
hazard~\parencite{peduzzi2009assessing}.
\end{comment}




\begin{figure*}%[tbhp] 
\centering 
\includegraphics[width=16cm]{ivanonly}
\caption{Counties classified as exposed to Hurricane Ivan (2004) under each
exposure metric (Table 1). The red line shows the track of Hurricane Ivan 
based on the revised Atlantic hurricane database (HURDAT2 \citep{landsea2013}).
Similar maps for other large-extent storms are given in Fig. S5.}
\label{fig:ivanexposure} 
\end{figure*}

% latex table generated in R 3.6.2 by xtable 1.8-4 package
% Wed Feb 19 10:41:15 2020
\begin{table}[ht]
\centering
\caption{Summary statistics for the number of county tropical cyclone exposures under each metric.} 
\label{tab:exposuresummaries}
\begin{tabular}{p{4.5cm}p{4.5cm}p{4.5cm}}
  \toprule
Metric (years available) & Mean (interquartile range) of county exposures per year & Tropical cyclone with most counties exposures (\# exposed counties) \\ 
  \midrule
Distance (1988--2018) & 399 (78, 503) & Beryl, 1994 (344) \\ 
  Rain (1988--2018) & 298 (48, 441) & Frances, 2004 (464) \\ 
  Wind (1988--2011) & 162 (55, 258) & Michael, 2018 (260) \\ 
  Flood (1996--2018) & 197 (76, 241) & Ivan, 2004 (317) \\ 
  Tornado (1988--2018) & 38 (8, 42) & Ivan, 2004 (91) \\ 
   \bottomrule
\end{tabular}
\end{table}


\begin{figure}%[tbhp] 
\centering 
\includegraphics[width = 0.7\linewidth]{jaccard_heatmap} 
\caption{Heatmap of Jaccard index values for
specific exposure metric pairs within storms. Only storms between 1996 and 2011,
and for which at least 250 counties were exposed based on at least one metric,
are included. The color of each cell within the main heatmap indicates the value
of the Jaccard index (proportion of counties classified as exposed by both
metrics out of storms classified as exposed by either metric) for a given pair
of metrics for a given storm. Storms are displayed within clusters that have
similar patterns in county-level exposure agreement for metric pairs, based on
hierarchical clustering using the complete link method
\citep{murtagh2012algorithms} (i.e., storms in the same cluster tend to have
similar patterns for the pairwise strength of agreement among metrics); columns
are also ordered based on hierarchical clustering. The colors to the right of
the main heatmap for each storm indicate the total number of counties classified
as exposed to the storm by any of the five metrics, providing an estimate of
storm extent. Maps are available showing the counties identified as exposed
under each of five metrics for the widest-extent storm in each cluster:
Hurricane Ivan (2004) (Fig. \ref{fig:ivanexposure}) and Hurricanes Floyd (1999),
Lee (2011), Cindy (2005), and Katrina (2005) (Fig. S5).} 
\label{fig:jaccard}
\end{figure}

\begin{figure*}%[tbhp] 
\centering
\includegraphics[width=16cm]{averageexposureonly} 
\caption{Average number of storm exposures per decade in U.S. counties for 
each exposure metric. The criteria behind each of the five metrics is given 
in Table \ref{tab:exposuremetrics}. The years used to estimate these averages 
are based on years of available exposure data (distance and wind: 1988--2015; 
rain: 1988--2011; flood and tornado: 1996--2015). Similar patterns persist when
analysis is restricted to years with all exposure data available (1996--2011;
Fig. S3).} 
\label{fig:averageexposure} 
\end{figure*}

\section*{Results}

\subsection*{Exposure assessment data, data validation, and software}

\paragraph{Open-source dataset}

We created and published open source data with multi-year, county-level
tropical cyclone exposure data for eastern \ac{US}
counties~\parencite{hurricaneexposuredata}. These data include continuous
county-level measurements for the closest distance of each storm, cumulative
rainfall, and maximum sustained surface wind speed. These continuous metrics
can be used for classifying counties as exposed or unexposed, using thresholds
selected by the user, or can be used as continuous metrics. The dataset also
include binary data on flood and tornado events associated with each storm in
each county. With these data, we published software tools to explore and map
the data and to integrate it with human health
datasets~\parencite{hurricaneexposure}. 

We explored potential limitations in these data by comparing them with data
from other available sources. 

\paragraph{Rainfall data validation}

For estimates of storm-associated rainfall, we compared estimates from NLDAS-2
re-analysis precipitation data (included as the rainfall estimates in the
open-source data) with ground-based observations in nine sample counties
(Figure~\ref{fig:raincomparison}). Within these counties, storm-related
rainfall measurements were well-correlated between the two data sources, with
rank correlations (bottom right of each graph in
Figure~\ref{fig:raincomparison}) between 0.87 and 0.98. There was some evidence
that the reanalysis data may tend to underestimate rainfall totals in storms
with extremely high rainfall, based a few heavy-rainfall storms in Harris
County, TX, Mobile County, AL, Charleston County, SC, and Wake County, NC
(Figure~\ref{fig:raincomparison}). However, even in these cases, it was rare
for a storm to be classified differently (based on the precipitation threshold
of 75~\si{\milli\metre} we used for exposure classification for later analysis)
based on the source of precipitation data---horizontal and vertical lines in
each small plot in Figure~\ref{fig:raincomparison} show the threshold of
75~\si{\milli\metre}, so storms in the lower left and upper right quadrants
would be classified the same (``exposed'' or ``unexposed'') regardless of the
precipitation data source, while storms in the upper left and lower right
quadrants would be classified differently.

\paragraph{Wind speed data validation}

For maximum sustained surface wind speed estimates, we found that the modeled
estimates, which we included as the primary wind exposure metric in the open
source data and used for further analysis in this study, generally agreed well
with estimates based on the wind radii reported in \ac{HURDAT2}. For most
storms, $\ge$95\% of counties were assigned the same category of wind speed
($<$34~\si{\knot}; 34\,--\,49.9~\si{\knot}; 50\,--\,63.9~\si{\knot};
$\ge$64~\si{\knot}) by both data sources (Figure~\ref{fig:windcomparison}).
Disagreement was limited to large storms that were characterized by high winds
well inland, like Hurricanes Sandy in 2012 and Ike in 2008. For these two
storms, the modeled wind speed values somewhat overestimated the extent of the
storm's most severe winds at landfall but then underestimated the extent of
34\,--\,49.9 \si{\knot} storm winds further inland (Figure~S6). For researchers
who would like to conduct sensitivity analysis between these two sets of wind
data, we have included estimates from the \ac{HURDAT2} wind radii as a
secondary measure of wind speeds in the open-source dataset
\parencite{hurricaneexposuredata}.

\paragraph{Flood data validation}

For flood data, we compared the flood status measurements, based on event
listings in the NOAA Storm Events database, to streamflow measurements at
\ac{US} Geological Survey gages within nine sample counties
(Figure~\ref{fig:floodcomparison}). Each small plot in
Figure~\ref{fig:floodcomparison} shows results for one of the sample counties.
Each point represents a single tropical storm, and the point's position along
the x-axis shows the highest daily total streamflow (cubic feet per second),
summed across all identified streamgages in the county, for a five-day window
centered on the day of the storm's closest approach to the county. The y-axis
separates storms for which a flood event was reported in NOAA's Storm Events
database for the county with a start date within the five-day window of the
storm's closest approach to the county. The color of each point gives the
percent of streamflow gages in the county with a daily streamflow that exceeded
a threshold of flooding on any day during the five-day window.  Stream gage
data generally indicated higher discharge in these counties on dates identified
as flood events, particularly in counties with more reporting stream gages
(e.g., Harris County, TX, with~10 reporting stream gages;
Figure~\ref{fig:floodcomparison}). There were some cases, however, where the
two flooding data sources were not completely consistent. 
For example, there
were one or two tropical cyclones in several of the counties (Mobile County,
AL, Escambia County, FL, and Montgomery County, PA) which were classified as
flood events based on NOAA storm event listings but for which the total
discharge across county streamflow gages was below the 25\textsuperscript{th}
percentile of measurements during storms without a flood event listing. 
For tropical cyclones that did not have an associated flood listing in 
the NOAA Storm Events database, most had lower total streamflow discharge
across county gages compareed to storms with flood event listings, and in all
cases but one none of the gages were over the flooding threshold. The exception
was \ldots in Fulton County, GA. [More on which storm and why---I think there
was already a flood event that had started earlier, so it missed the cutoff for
the startdate being within the range.]

[For discussion?] Based on these results, the two data sources were generally
in agreement, but the analysis highlights the difficulty of assessing
storm-related flooding at a multi-county, multi-year scale, as flooding can be
very localized within a county or can \ldots .   

\subsection*{Patterns in tropical cyclone exposures in eastern \ac{US} counties}

We used the storm exposure metrics to classify counties as exposed or unexposed
to specific tropical cyclones, and we explored patterns in these exposure
classifications over years with available data
(Table~\ref{tab:exposuresummaries}). Across the five exposure metrics, there
was wide variation in the average number of county exposures per year. For the
tornado metric, there were approximately~50 county exposures per
year, on average, within our study.  For flood and wind metrics, there were
substantially more county exposures ($>$200 / yr on average), and
even more when assessing exposure using the distance and rain metrics ($>$400 /
yr on average). 

We also measured the average number of county exposures per storm under each
exposure metric (limiting this analysis to the tropical cyclones that exposed
at least one county to that hazard; Table~\ref{tab:exposuresummaries}).
Tropical cyclones typically exposed only a few counties, if any, to tornadoes
(median counties exposed per storm:~7), more counties under the flood and wind
metrics (median:~26 and~34, respectively), and even more counties for the
metrics of rain (median:~68) and distance (median:~84).  For every metric
except the tornado-based metric, we identified at least one tropical cyclone
with over~300 counties exposed.  However, the largest-extent tropical cyclone
varied by metric (e.g., Beryl in~1994 exposed the most counties based on
distance, Frances in~2004 based on rain, and Ike in~2008 based on wind).

When we calculated and mapped the average number of exposures per decade in
each county for each exposure metric (Figure~\ref{fig:averageexposure}), strong
geographical patterns were clear. Wind-based exposure had a strong coastal
pattern, with almost all exposures in counties within about~200
\si{\kilo\metre} (124~mi) of the coastline. While exposures by rain- and
distance-based metrics were also more common in coastal areas compared to
inland areas, they also created inland exposures that were not identified by
the wind-based metric. Rain and, to some extent, flood exposures were
characterized by a pattern defined by the Appalachian Mountains, with much
lower exposures west of the mountain range than to the east. Almost all
tornado-based exposures were in coastal states, with most in Florida, Alabama,
South Carolina, and North Carolina and almost none north of Maryland.
Flood-based exposures were highest in North Carolina and along the South
Carolina coast, as well as further north in New Jersey, southeastern
Pennsylvania, and southeastern New York. Notably, some areas of the \ac{US}
that experienced regular exposures to tropical cyclone-related excessive
rainfall and floods were not identified as areas of regular risk based on the
more commonly used tropical cyclone exposure assessment metrics of wind or
distance (Figure~\ref{fig:averageexposure}). Patterns were similar when
analysis was restricted to years with all exposure data available
(1996\,--\,2011; Figure~S5). 

\subsection*{Agreement in county-specific exposure classification between exposure metrics}

Next, we measured the agreement between exposure classifications for each pair
of metrics. We found that these exposure classifications typically did not
strongly agree between pairs of metrics, indicating that the set of counties
identified as exposed based on one metric can have limited overlap with the
set of counties identified as exposed to the storm by another metric. 

Figure~\ref{fig:ivanexposure} shows Hurricane Ivan in 2004 as an example. For
the distance-based metric, the counties assessed as exposed followed the
tropical cyclone's track, including counties well inland. For the wind metric,
only counties near two of the tropical cyclone's landfalls were assessed as
exposed. For rain- and flood-based metrics, however, exposure extended to the
left of the track, including counties as far north as New York and Connecticut,
while for the tornado metric, exposed counties tended to be to the right of the
track and included several counties in central North Carolina, South Carolina,
and Georgia that were not identified as exposed to Ivan based on any other
metric. Figure~S3 provides similar maps for four other example tropical
cyclones (selected because they exposed many \ac{US}  counties based on at
least one metric), and Figure~S4 provides four examples of tropical cyclones in
which the disagreement in exposure assessment between rain- and wind-based
metrics was particularly notable.

We drew similar conclusions when we investigated all~46 tropical cyclones
between~1996 and~2011 (the period for which all five metrics were available)
for which~250 or more counties were exposed based on at least one metric
(Figure~\ref{fig:jaccard}). In this figure, each row provides results for a
specific tropical cyclone, and each box in that row shows the measured Jaccard
coefficient between two of the exposure metrics for that storm.  For most
tropical cyclones, distance-based county-level exposure assessment was at best
in moderate agreement with exposure assessments based on the four hazard-based
metrics. For almost all tropical cyclones, Jaccard index values between
distance and any of the four hazard-based metrics were~$\le0.6$ (i.e., out of
the counties assessed as exposed to the tropical cyclone by at least one of the
two metrics considered,~60\si{\percent} or fewer of the counties had the same
exposure assessment under the two metrics). For many tropical cyclones, the
Jaccard index values were~$\le0.2$, suggesting very poor agreement in exposure
assessment between the metrics.  Among the four hazard-based metrics, the
tornado-based metric showed universally poor agreement with other metrics in
county-level classification across all tropical cyclones considered.  For other
pairs of metrics, there were also generally large differences in which counties
were determined to be exposed, with the Jaccard index values~$<0.4$ for most
metric pairs in most tropical cyclones.

\begin{comment}
Storms are displayed within
clusters that have similar patterns in county-level exposure agreement for
metric pairs, based on hierarchical clustering using the complete link
method~\parencite{murtagh2012algorithms} (i.e., storms in the same cluster tend
to have similar patterns for the pairwise strength of agreement among metrics);
columns are also ordered based on hierarchical clustering. 
Maps are
available showing the counties identified as exposed under each of five metrics
for the widest-extent storm in each cluster: Hurricane Ivan in~2004
(Figure~\ref{fig:ivanexposure}) and Hurricanes Floyd in~1999, Lee in~2011, Cindy
in~2005, and Katrina in~2005 (Figure~S3).
\end{comment}

There were, however, a few exceptions\,---\,tropical cyclones in which similar
counties were determined to be exposed to the tropical cyclone for two or more
of the metrics considered.  For Floyd in~1999 (Figure~S3) and Irene in~2011,
for example, county-level classification agreed moderately to well (Jaccard
index of~$\sim 0.5\mbox{\,--\,}0.8$) for all pairs of exposure metrics except
those including the tornado-based metric. For another set of tropical cyclones
(e.g., Lee in~2011 [Figure~S3], Ernesto in~2006, and Bertha in~1996), there was
moderate to good agreement for all pairwise combinations of distance, rain, and
wind, but poor agreement for all other combinations of metrics. However, these
two sets of tropical cyclones represented the minority of all tropical cyclones
considered; for most tropical cyclones, there was low overlap in the counties
determined to be exposed to the tropical cyclones between most pairings of
metrics.

\subsection*{Case study}

Finally, we conducted a case study to investigate how these differences in
exposure assessment across the exposure metrics could affect estimates of
expected physical exposure among a vulnerable subpopulation,
electricity-dependent Medicare beneficiaries.
Figure~\ref{fig:topelecdependexposure} shows the study counties with the
highest expected physical exposures in this subpopulation based on each
exposure metric. A few counties (Miami-Dade County, FL; Harris County, TX) were
ranked in the top three counties of physical exposure for almost all exposure
metrics (Figure~\ref{fig:topelecdependexposure}). However, there were key
differences when using different exposure metrics. For example, Florida
counties made up most of the counties counties with highest expected physical
exposure based on wind- and tornado-based metrics, while over half of the
counties with highest expected physical exposures for the flood-based metric
were in Mid-Atlantic states (PA, NY).





\section*{Discussion}

Tropical cyclones impact public health in many ways, and epidemiological
studies of the health risk and impacts associated with tropical cyclone
exposures could help improve preparedness for and response to future tropical
cyclones.  However, tropical cyclones are multi-hazard events, making it
complicated to measure exposure. Here, we provide open-source county-level data
on multiple storm exposure metrics, and we find that county-level tropical
cyclone exposure assessments vary substantially when using a distance-based
metric versus four hazard-based metrics, as well as among different
hazard-based metrics. Our results can inform exposure assessment for future
county-level studies of the health risks and impacts associated with tropical
cyclones exposures, and the open-source data and software we have created
provides multi-hazard, county-level exposure data for such studies.

\subsection*{Exposure assessment data and software}

In creating this hurricane exposure dataset, we aimed for data that are
available for all eastern \ac{US} counties. In addition to providing tropical
cyclone exposure metrics for individual hazards, this data and its associated
software allow users not only to access measurements for single hazards, but
also to create tropical cyclone exposure profiles based on multiple hazards or
craft exposure indices that combine hazard
metrics~\parencite{chakraborty2005population, peduzzi2009assessing}. This
ability can be critical, as different hazards of tropical cyclones often act
synergistically in causing impacts~\parencite{smith2009}.  

\begin{comment}
To assist with future tropical cyclone studies, we created and published open
source software with this dataset and accompanying
tools~\parencite{hurricaneexposuredata, hurricaneexposure}.  Many previous
studies have used geographical information system software (e.g., ArcGIS) to
assess exposure to tropical cyclones in the \ac{US}~\parencite{grabich2016,
zandbergen2009, czajkowski2011, kruk2010}.  Here, we offer methods to map and
output historic exposure to tropical cyclones that does not require the use of
proprietary software but instead uses a package written in the R statistical
programming language~\parencite{R}, which is free and open-source.  Further, by
including measurements of different hazard exposures in each county for each
tropical cyclone, this software allows for the development of more complex
exposure indices or models (e.g., random forests, multivariable generalized
linear models) that incorporate multiple tropical cyclone hazard measurements.  
\end{comment}

We investigated how well these data correspond with data from other available
sources (Figures~\ref{fig:windcomparison}\,--\,\ref{fig:floodcomparison}
and~S6\,--\,7). While generally in agreement with data from other sources,
there are a few caveats that should be considered when applying these data in
future studies. The rainfall data are from re-analysis data are generally
well-correlated with observed ground-based station data, but may sometimes
oversmooth very high rainfall values compared to ground-based observations
(Figure~\ref{fig:raincomparison}). When rainfall data is used to create binary
exposure classifications, this disagreement is unlikely to influence results,
as both data sources agree in identifying these as storms with high rainfall,
but it may be important to consider for cases that include rainfall as a
continuous measurement. 

The wind data are based on modeled, rather than observed, values, and while the
modeled wind data generally agree well with post-analysis maximum wind radii
from \ac{HURDAT2}~\parencite{landsea2013} (Figure~\ref{fig:windcomparison}),
there were a few storms where the two datasets did not agree as well ($<$90\%
of counties classified into the same wind categories by the two sources of wind
estimates). These storms were exceptionally large systems for which high winds
persisted well inland from landfall, like Hurricanes Sandy and Ike
(Figures~\ref{fig:windcomparison} and\~S6). Since the wind radii are based on
\ldots, they have [advantages]; their main disadvantages is that they provide
categorical, rather than continuous, estimates of wind speed, and in many cases
researchers might prefer more precise estimates of storm wind speeds. However,
since the wind radii data may be useful in some cases as either a primary
metric or for sensitivity analysis, we have include county-level wind estimates
for each storm based on these wind radii, in addition to the continuous modeled
windspeeds, in the open-source data \cite{hurricaneexposuredata}.

For the flooding events data, we found it often corresponded with measurements
from \ac{US} Geographical Survey streamgages, with flood event listings more
likely during storms which elevated stream flows. However, there are
differences between the two flooding datasets, and these highlight both the
difficulty of measuring flood exposure at the county level and inherent
challenges in using the storm event database for epidemiologic exposure
assessment.  The flood and tornado data came from the \ac{NOAA} Storm Events
database, which, while a widely-used database of events maintained by
\ac{NOAA}, is based on reports, and so may be prone to
underreporting~\parencite{Ashley2008flood, Curran2000}, especially in less
populated areas~\parencite{Witt1998, Ashley2007}, as well as to other reporting
errors. 

\begin{comment}
[For discussion?] Based on these results, the two data sources were generally
in agreement, but the analysis highlights the difficulty of assessing
storm-related flooding at a multi-county, multi-year scale, as flooding can be
very localized within a county or can \ldots .   
\end{comment}

While these are important caveats for the data, we selected these data sources
as among the best currently available for measuring each of these hazards at a
multi-county, multi-year scale.

\subsection*{Patterns in tropical cyclone exposures}

We found geographic patterns in tropical cyclones based on the different
exposure metrics that. These patterns were not unexpected based on what is
known about tropical cyclone hazards, but still highlight variations across
metrics that are critical to consider in designing studies for tropical cyclone
epidemiology, and they demonstrate the need for multi-hazard exposure datasets
for tropical cyclone epidemiology. Wind-based exposures had a strong coastal
pattern, which is consistent with the dramatic decrease in wind intensity that
typically characterizes the landfall of tropical cyclones. Exposures by rain-
and distance-based metrics often extended further inland compared to the
wind-based metric, up to the Appalachian mountains. This agrees with previous
research indicating that the Appalachian mountains' topography both enhances
precipitation during tropical cyclones and provides hydrological conditions for
severe flooding~\parencite{rees2001}.  Almost all tornado-based exposures were
in southern coastal states, echoing previous findings that most noteworthy
tropical cyclone-related tornadoes occur on the right-side of the tropical
cyclone track in Atlantic-basin \ac{US} storms~\parencite{moore2012}.  These
exposure averages are somewhat limited as estimates of long-term frequencies,
as tropical cyclones follow decadal patterns~\parencite{kossin2007more} likely
not adequately captured in the available data. However, exposure assessments
spanning over a century have found similar patterns for distance- and
wind-based metrics~\textcite{zandbergen2009, kruk2010} [double-check that refs
found similar patterns].

\begin{comment}
However, these frequency maps, together with evidence from
specific tropical cyclones (Figures~\ref{fig:ivanexposure}
and~\ref{fig:jaccard}),  do illustrate the potential for strong differences in
spatial patterns in tropical cyclone exposures, depending on which tropical
cyclone hazards are considered.  A few previous studies have sought to
determine county-level exposure to tropical cyclones over multi-year periods,
including~\textcite{zandbergen2009}, which estimated exposure in \ac{US}
counties to all \ac{US} landfalling Atlantic-basin tropical cyclones
between~1851 and~2003, using both a distance-based metric and a metric that
combined distance and windspeed, and~\textcite{kruk2010}, which explored
exposure to hurricane-related winds in the \ac{US}, including inland areas,
for~1900\,--\,2008.  Our results suggest that such exposure assessments may
perform well in capturing some tropical cyclone hazards (e.g., wind), but
likely miss other potentially dangerous tropical cyclone exposures, especially
for hazards that repeatedly threaten northern or inland counties (e.g., rain,
flooding).
\end{comment}

\subsection*{Agreement between exposure metrics}

We found that there was typically only moderate or low agreement between
county-level exposure classifications for a tropical cyclone when comparing
different exposure metrics. This suggests that the use of one of these metrics
as a proxy for all storm-based hazards in epidemiological research would often
be problematic, introducing exposure misclassification.

\begin{comment}
For example, distance from a tropical cyclone's track is relatively easy to
measure and has been used as an operational metric of exposure to tropical
cyclones in previous large-scale studies (examples include ...). Since distance
itself does not constitute a hazard, distance is meant in these cases as a
surrogate to capture exposure to hazards from the tropical cyclone. However,
here we found that in assessing \ac{US} county-level exposure to tropical
cyclones, distance is, at best, a moderate, and often a very poor, surrogate
for exposure to the specific tropical cyclone hazards of high wind, extreme
rainfall, flooding, and tornadoes (Figure~\ref{fig:jaccard}). Therefore, use of
distance to assess tropical cyclone exposure for impact studies could result in
problematic exposure misclassification, which could mask true associations,
even strong associations, between tropical cyclone exposure and outcomes of
interest in impact studies~\parencite{savitz2016interpreting,
armstrong1998effect}.  
\end{comment}

These findings align with previous results from atmospheric science and related
fields on the characteristics of tropical cyclones. While tropical cyclone
rainfall and windspeed are often well-correlated when the tropical cyclone is
over water~\parencite{cerveny2000}, this relationship often does not remain as
strong once the hurricane has made landfall~\parencite{jiang2008}.  Fast-moving
tropical cyclones bring higher risks of dangerous winds
inland~\parencite{kruk2010}, while slow-moving tropical cyclone are likely to
bring more rain~\parencite{rappaport2000} and cause more damage because of
sustained hazardous conditions~\parencite{rezapour2014}. Further, while the
likelihood and extent of flooding during a tropical cyclone is related to the
tropical cyclone's rainfall, it is also driven by factors like top soil
saturation and the structure of the water basin's drainage
network~\parencite{chen2015, rees2001}. 

However, we did find a small set of tropical cyclones for which for which
agreement between metrics were unusually high (e.g., Floyd in~1999, Irene in
2011, Hannah in~2008, Bertha in~1996).  Hurricanes Floyd in~1999 and Irene
in~2011 both made their first \ac{US} landfall in North Carolina at minor
hurricane windspeeds (Category~2 and~1, respectively) and then skimmed the
eastern coast of the \ac{US} north through New England, bringing large rainfall
to much of the eastern coast from North Carolina north and causing extensive
inland flooding in North Carolina (Floyd) and New England
(Irene)~\parencite{avila2013atlantic, lawrence2000atlantic}.  Hurricanes Hannah
in~2008 and Bertha in~1996 followed the eastern coastline north of North
Carolina in a pattern similar to Floyd and Irene. For this set of storms, the
tropical cyclones' persistent proximity to water may have helped maintain wind
speeds in similar patterns to rain and distance exposures, resulting in this
moderate to good agreement among exposure assessments based on different
metrics. 

In some cases, exposure misclassification from using one of these exposure
metrics to capture one or more of the other storm hazards may be differential
(i.e., associated with the outcome of interest or with factors associated with
risk of the outcome of interest). For example, tropical cyclone wind exposures
tend to be concentrated in counties near the coast, since most tropical
cyclones rapidly decrease in their sustained windspeed following landfall.
However, tropical cyclone exposures based on distance can extend well inland,
following the tropical cyclone's tracks, but may not adequately capture all
wind-exposed counties near the coast. In this case, if the
etiologically-relevant exposure is high wind but exposure is classified based
on distance, the probability of being misclassified as unexposed would be
higher in coastal counties, while the probability of being misclassified as
exposed would be higher in inland counties. If coastal counties differ from
inland counties in either the outcome of interest or in factors associated with
risk of that outcome, differential exposure misclassification would
exist~\parencite{savitz2016interpreting}. Such differential exposure
misclassification can bias estimates of tropical cyclone effects either towards
the null (estimating a lower or null association compared to the true
association that exists) or away from the null (estimating a larger association
than actually exists)~\parencite{savitz2016interpreting, armstrong1998effect}.

\begin{comment}
The use of a single hazard-based metric (e.g., wind) could cause similar
problems if the impact is driven, at least in part, by a different hazard or by
multiple hazards of the tropical cyclone.
\end{comment}

\subsection*{Case study}

The conducted case study demonstrates how differences in exposure assessment
across storm exposure metrics can result in differing estimates of expected
physical exposure to tropical cyclones among a population
(Figure~\ref{fig:topelecdependexposure}).  Previous assessments of physical
exposure to tropical cyclones have often focused on solely storm winds and
storm proximity. For example, a global study of exposure and vulnerability to
natural hazards used a combination of windspeed above a certain threshold and
distance from the storm's track to assess
exposure~\parencite{peduzzi2009assessing}. While many storm impacts might be
most strongly linked to wind hazards, there is a growing recognition of the
potential risks of adverse health outcomes and property damage from rain- and
flood-related tropical cyclone hazards in the \ac{US}~\parencite{smith2009}.
Such assessments may underemphasize potentially dangerous exposures to certain
regions of the \ac{US}, particularly inland locations. In fact, based on our
assessment of physical exposures to tropical cyclones among
electricity-dependent Medicare beneficiaries, the highest expected physical
exposure under any metric was measured for Philadelphia County, PA, under the
flood metric (Figure~\ref{fig:topelecdependexposure}).  While this county has a
similar-sized population of electricity-dependent Medicare beneficiaries as
Miami-Dade County, FL (Figure~\ref{fig:topelecdependexposure}), the frequency
of storm-related flood events in this inland county results in a very high
expected physical exposure among this subpopulation, while this county is not
included in the top ten counties for physical exposure based on most of the
other exposure metrics.

\subsection*{Conclusions}

Previous research has highlighted the range of impacts that tropical cyclones
can have in \ac{US} communities. However, studies have varied widely in how they
assess exposure to tropical cyclones for exposure, health impact, and other
impact studies, in some cases using distance to the tropical cyclone's track as
a surrogate metric of exposure to a tropical cyclone's hazards. Here we found
large differences in which counties are exposed to different hazards of
tropical cyclones and that distance is, at best, a moderate, and often a very
poor, surrogate for exposure to the specific tropical cyclone hazards of high
wind, extreme rainfall, flooding, and tornadoes. Use of distance as a surrogate
for any of these hazards could lead to exposure misclassification and, in the
case of tropical cyclone risk and impact studies, including epidemiological
studies, result in biased estimates.  Similarly, if such studies disagree on
their findings, it could result from the often poor agreement between exposure
classifications based on these different metrics, making studies that use
different metrics hard to compare or combine in meta-analyses. Our findings
highlight the importance of clarifying the potential pathway from tropical
cyclone hazards to health impacts when conducting tropical cyclone
epidemiological studies and basing exposure assessment on measurements of these
hazards. To help with this, we provide extensive tropical cyclone
hazard-related measurements for a large collection of historical Atlantic-basin
cyclones, aggregated at the county level to align with the spatial scale at
which much health-related data is collected.


\section*{Acknowledgments}

This work was supported in part by grants from the National Institute of
Environmental Health Sciences (R00ES022631), the National Science Foundation
(1331399), the Department of Energy (Grant No. DE-FG02-08ER64644), and a
National Aeronautics and Space Administration Applied Sciences Program/Public
Health Program Grant (NNX09AV81G). Rainfall data are based on data acquired as
part of the mission of the National Aeronautics and Space Administration's Earth
Science Division and archived and distributed by the Goddard Earth Sciences
(GES) Data and Information Services Center (DISC).


\bibliography{hurr_exposure}

\end{document}
