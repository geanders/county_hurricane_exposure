\section*{Figures}

\listoffigures

\clearpage

\begin{figure}[tbhp!] \centering
\includegraphics[width=0.7\linewidth]{hurrtracks}
\caption{States and storms considered in this study. All counties in the states
        shown in this map were investigated. The lines show the paths
	of the study storms, which included all tracked storms in 1988\,--\,2015
	\ac{HURDAT2}~\parencite{landsea2013} that came within 250~\si{\kilo\metre} 
	of at least one \ac{US} county. Thicker lines show the tracks 
	of storms whose names have been retired, indicating that the storm was particularly 
	severe or had notable impacts~\parencite{retirednames}.
	}
\label{fig:hurrtracks}
\end{figure}

\clearpage

\begin{figure}[tbhp!] \centering
\includegraphics[width=0.9\linewidth]{raincomparison} 
	\caption{Comparison of storm-associated rainfall estimates between 
	a re-analysis dataset (included as the rainfall exposure metric
	in the open-source data and included in further analysis) and ground-based
	observations in nine sample counties. The rainfall estimates  
	include rainfall from two days before to one day after the storm's closest
	approach to the county. Each small graph shows data for one sample county, 
	and each point shows one tropical storm. The number of storms 
	within each county and the average number of stations reporting rainfall during 
	the county's storms are given above each plot. Horizontal and vertical lines 
	in each plot show the threshold of 75~\si{\milli\metre} used to classify a storm 
	as ``exposed'' in further analysis (Table 1). Note that 
	x- and y-axis ranges differ by county.
	} 
\label{fig:raincomparison}
\end{figure}

\begin{figure}[tbhp!]
\centering
\includegraphics[width=0.6\linewidth]{windcomparison}
\caption{Comparison of modeled maximum sustained surface wind speeds 
	(included as the primary wind exposure metric in the open-source
	data and used in further analysis) with estimates from the \ac{HURDAT2}
	wind speed radii. Each point represents a storm, with the x-axis
	giving the percent of counties classified in the same category
	of maximum sustained surface wind speeds ($<$34~\si{\knot}; 
	34\,--\,49.9~\si{\knot}; 50\,--\,63.9~\si{\knot}; 
	$\ge$64~\si{\knot}) by both sets of wind data for that storm. The 
	color of each point indicates the extent of the storm, 
	giving the number of study counties that were exposed to 
	maximum sustained surface wind speeds of at least 34~\si{\knot}
	(based on modeled wind speeds). Estimates are shown for [all?]
	study storms since 2004, the earliest year for which 
	wind speed radii estimates are available in \ac{HURDAT2} [?].
	}
\label{fig:windcomparison}
\end{figure}

\begin{figure}[tbhp!]
\centering
\includegraphics[width=0.8\linewidth]{floodcomparison}
\caption{Comparison for a sample of study counties of NOAA Storm Events 
	listings versus five-day total streamflows at county streamflow 
	gages during tropical cyclones. Each small plot shows results for 
	one of the sample counties. Each point represents a single tropical 
	storm, and the point's position along the x-axis shows the highest 
	daily total streamflow (cubic feet per second), summed across all 
	identified streamgages in the county, for a five-day window centered 
	on the day of the storm's closest approach to the county. The y-axis 
	separates storms for which a flood event was reported in NOAA's Storm Events 
	database for the county with a start date within the five-day window 
	of the storm's closest approach to the county. The color of each point 
	gives the percent of streamflow gages in the county with a daily streamflow 
	that exceeded a threshold of flooding on any day during the five-day window.
	Comparison for a sample of study counties of NOAA Storm Events 
	listings versus five-day total streamflows at county streamflow gages during 
	tropical storms. The number of streamflow gages used in analysis are given in 
	parentheses beside the county's name in the panel title. Each point represents 
	a single tropical storm. Note that the x-axis scales differ by county, 
	depending on the number of streamflow gages and typical flow rates for each 
	gage, and are on a log-10 scale.}
\label{fig:floodcomparison}
\end{figure}

\clearpage

\begin{figure}%[tbhp] 
\centering
\includegraphics[width=15cm]{averageexposureonly.pdf} 
\caption{Average number of storm exposures per decade in U.S. counties for each 
	exposure metric. The criteria behind each of the five metrics is given in 
	Table \ref{tab:exposuremetrics}. The years used to estimate these averages 
	are based on years of available exposure data 
	(distance and wind:~1988\,--\,2015; rain: 1988\,--\,2011; 
	flood and tornado events:~1996\,--\,2015). } 
\label{fig:averageexposure} 
\end{figure}

\clearpage

\begin{figure}%[tbhp]
\centering
\includegraphics[width=16cm]{ivanonly.pdf}
\caption{Counties classified as exposed to Hurricane Ivan in 2004 under each
exposure metric considered (Table~\ref{tab:exposuremetrics}). The red line
shows the track of Hurricane Ivan based on the revised Atlantic hurricane
database \ac{HURDAT2}~\parencite{landsea2013}.  Similar maps for other
large-extent storms are given in Figure~S3.}
\label{fig:ivanexposure} 
\end{figure}

\clearpage

\begin{figure}%[tbhp] 
\centering 
\includegraphics[width = 0.6\linewidth]{jaccard_heatmap.pdf} 
\caption{Agreement between exposure classifications based on different
         exposure metrics for all storms between~1996 and~2011 for which 
	 at least~250 counties were exposed based on at least one metric.
	 Each row shows one storm, and the color of each cell shows the 
	 measured Jaccard index for each pair of exposure metrics 
	 (proportion of counties classified as exposed by both metrics out 
	 of storms classified as exposed by either metric). The colors to the 
	 right of the main heatmap for each storm indicate the total number of 
	 counties classified as exposed to the storm by any of the five metrics, 
	 providing an indication of each storm's extent. Storms are displayed 
	 within clusters that have similar patterns in county-level exposure 
	 agreement for metric pairs, based on hierarchical clustering using the 
	 complete link method~\parencite{murtagh2012algorithms}; columns are also 
	 ordered based on hierarchical clustering. Maps are available showing the 
	 counties identified as exposed under each of five metrics for the widest-extent 
	 storm in each cluster: Hurricane Ivan in~2004 (Figure~\ref{fig:ivanexposure}) 
	 and Hurricanes Floyd in~1999, Lee in~2011, Cindy in~2005, and Katrina 
	 in~2005 (Figure~S3).
} 
\label{fig:jaccard}
\end{figure}

\clearpage

\begin{figure*}%[tbhp]
\centering
\includegraphics[width=15.5cm]{topelecdependexposure}
\caption{Study counties with the highest expected physical exposure per year among
	 electricity-dependent Medicare beneficiaries for each exposure metric. 
	 The color of each bar indicates the number of Medicare beneficiaries in the 
	 county reliant on electricity-dependent medical and assistive equipment as 
	 of July~2017. The length of each bar shows the average expected physical exposure
	 to tropical cyclones among this population based on a given exposure metric, i.e., 
	 the expected number of these electricity-dependent Medicare beneficiaries exposed 
	 to tropical storms per year based on that metric.}
\label{fig:topelecdependexposure}
\end{figure*}


