\section*{Results}

\subsection*{Exposure assessment data, data validation, and software}

\paragraph{Open-source dataset}

We created and published open source data with multi-year, county-level
tropical cyclone exposure data for eastern \ac{US}
counties~\parencite{hurricaneexposuredata}. These data include continuous
county-level measurements for the closest distance of each storm, cumulative
rainfall, and maximum sustained surface wind speed. These continuous metrics
can be used for classifying counties as exposed or unexposed, using thresholds
selected by the user, or can be used as continuous metrics. The dataset also
include binary data on flood and tornado events associated with each storm in
each county. With these data, we published software tools to explore and map
the data and to integrate it with human health
datasets~\parencite{hurricaneexposure}. 

We explored potential limitations in these data by comparing them with data
from other available sources. 

\paragraph{Rainfall data validation}

For estimates of storm-associated rainfall, we compared estimates from NLDAS-2
re-analysis precipitation data (included as the rainfall estimates in the
open-source data) with ground-based observations in nine sample counties
(Figure~\ref{fig:raincomparison}). Within these counties, storm-related
rainfall measurements were well-correlated between the two data sources, with
rank correlations (bottom right of each graph in
Figure~\ref{fig:raincomparison}) between 0.87 and 0.98. There was some evidence
that the reanalysis data may tend to underestimate rainfall totals in storms
with extremely high rainfall, based a few heavy-rainfall storms in Harris
County, TX, Mobile County, AL, Charleston County, SC, and Wake County, NC
(Figure~\ref{fig:raincomparison}). However, even in these cases, it was rare
for a storm to be classified differently (based on the precipitation threshold
of 75~\si{\milli\metre} we used for exposure classification for later analysis)
based on the source of precipitation data---horizontal and vertical lines in
each small plot in Figure~\ref{fig:raincomparison} show the threshold of
75~\si{\milli\metre}, so storms in the lower left and upper right quadrants
would be classified the same (``exposed'' or ``unexposed'') regardless of the
precipitation data source, while storms in the upper left and lower right
quadrants would be classified differently.

\paragraph{Wind speed data validation}

For maximum sustained surface wind speed estimates, we found that the modeled
estimates, which we included as the primary wind exposure metric in the open
source data and used for further analysis in this study, generally agreed well
with estimates based on the wind radii reported in \ac{HURDAT2}. For most
storms, $\ge$95\% of counties were assigned the same category of wind speed
($<$34~\si{\knot}; 34\,--\,49.9~\si{\knot}; 50\,--\,63.9~\si{\knot};
$\ge$64~\si{\knot}) by both data sources (Figure~\ref{fig:windcomparison}).
Disagreement was limited to large storms that were characterized by high winds
well inland, like Hurricanes Sandy in 2012 and Ike in 2008. For these two
storms, the modeled wind speed values somewhat overestimated the extent of the
storm's most severe winds at landfall but then underestimated the extent of
34\,--\,49.9 \si{\knot} storm winds further inland (Figure~S6). For researchers
who would like to conduct sensitivity analysis between these two sets of wind
data, we have included estimates from the \ac{HURDAT2} wind radii as a
secondary measure of wind speeds in the open-source dataset
\parencite{hurricaneexposuredata}.

\paragraph{Flood data analysis}

For flood data, we compared flood status, based on event listings in the NOAA
Storm Events database, to streamflow measurements at \ac{US} Geological Survey
gages within nine sample counties (Figure~\ref{fig:floodcomparison}).   

For flood data, we compared flood status, based on event listings in the NOAA
Storm Events database, to streamflow measurements at \ac{US} Geological Survey
gages within nine sample counties (Figure~\ref{fig:floodcomparison}).  Across
the sampled counties, streamgage data generally indicated higher discharge in
on dates identified as flood events, particularly in counties with more
reporting stream gages (e.g., Harris County, TX, with~10 reporting stream
gages; Figure~\ref{fig:floodcomparison}). There were some cases, however, where
the two flooding data sources were somewhat inconsistent.  For example, there
were one or two tropical cyclones in several of the counties (Mobile County,
AL, Escambia County, FL, and Montgomery County, PA) which were classified as
flood events based on NOAA storm event listings but for which the total
discharge across county streamflow gages was below the 25\textsuperscript{th}
percentile of measurements taken during storms that did not have a flood event
listing. For tropical cyclones that did not have an associated flood listing in
the NOAA Storm Events database, most had lower total streamflow discharge
across county gages compared to storms with flood event listings, and in all
cases but one all gages were below the flooding threshold.  The exception was
one storm in Fulton County, GA (Ida in 2009). [More on which storm and why---I
think there was already a flood event that had started earlier, so it missed
the cutoff for the startdate being within the range.]

\subsection*{Patterns in tropical cyclone exposures in eastern \ac{US} counties}

We used the storm exposure metrics to classify counties as exposed or unexposed
to specific tropical cyclones, and we explored patterns in these exposure
classifications over years with available data
(Table~\ref{tab:exposuresummaries}). Across the five exposure metrics, there
was wide variation in the average number of county exposures per year. For the
tornado metric, there were approximately~50 county exposures per
year, on average, within our study.  For flood and wind metrics, there were
substantially more county exposures ($>$200 / yr on average), and
even more when assessing exposure using the distance and rain metrics ($>$400 /
yr on average). 

We also measured the average number of county exposures per storm under each
exposure metric (limiting this analysis to the tropical cyclones that exposed
at least one county to that hazard; Table~\ref{tab:exposuresummaries}).
Tropical cyclones typically exposed only a few counties, if any, to tornadoes
(median counties exposed per storm:~7), more counties under the flood and wind
metrics (median:~26 and~34, respectively), and even more counties for the
metrics of rain (median:~68) and distance (median:~84).  For every metric
except the tornado-based metric, we identified at least one tropical cyclone
with over~300 counties exposed.  However, the largest-extent tropical cyclone
varied by metric (e.g., Beryl in~1994 exposed the most counties based on
distance, Frances in~2004 based on rain, and Ike in~2008 based on wind).

When we calculated and mapped the average number of exposures per decade in
each county for each exposure metric (Figure~\ref{fig:averageexposure}), strong
geographical patterns were clear. Wind-based exposure had a strong coastal
pattern, with almost all exposures in counties within about~200
\si{\kilo\metre} (124~mi) of the coastline. While exposures by rain- and
distance-based metrics were also more common in coastal areas compared to
inland areas, they also created inland exposures that were not identified by
the wind-based metric. Rain and, to some extent, flood exposures were
characterized by a pattern defined by the Appalachian Mountains, with much
lower exposures west of the mountain range than to the east. Almost all
tornado-based exposures were in coastal states, with most in Florida, Alabama,
South Carolina, and North Carolina and almost none north of Maryland.
Flood-based exposures were highest in North Carolina and along the South
Carolina coast, as well as further north in New Jersey, southeastern
Pennsylvania, and southeastern New York. Notably, some areas of the \ac{US}
that experienced regular exposures to tropical cyclone-related excessive
rainfall and floods were not identified as areas of regular risk based on the
more commonly used tropical cyclone exposure assessment metrics of wind or
distance (Figure~\ref{fig:averageexposure}). Patterns were similar when
analysis was restricted to years with all exposure data available
(1996\,--\,2011; Figure~S5). 

\subsection*{Agreement in county-specific exposure classification between exposure metrics}

Next, we measured the agreement between exposure classifications for each pair
of metrics. We found that these exposure classifications typically did not
strongly agree between pairs of metrics, indicating that the set of counties
identified as exposed based on one metric can have limited overlap with the
set of counties identified as exposed to the storm by another metric. 

Figure~\ref{fig:ivanexposure} shows Hurricane Ivan in 2004 as an example. For
the distance-based metric, the counties assessed as exposed followed the
tropical cyclone's track, including counties well inland. For the wind metric,
only counties near two of the tropical cyclone's landfalls were assessed as
exposed. For rain- and flood-based metrics, however, exposure extended to the
left of the track, including counties as far north as New York and Connecticut,
while for the tornado metric, exposed counties tended to be to the right of the
track and included several counties in central North Carolina, South Carolina,
and Georgia that were not identified as exposed to Ivan based on any other
metric. Figure~S3 provides similar maps for four other example tropical
cyclones (selected because they exposed many \ac{US}  counties based on at
least one metric), and Figure~S4 provides four examples of tropical cyclones in
which the disagreement in exposure assessment between rain- and wind-based
metrics was particularly notable.

We drew similar conclusions when we investigated all~46 tropical cyclones
between~1996 and~2011 (the period for which all five metrics were available)
for which~250 or more counties were exposed based on at least one metric
(Figure~\ref{fig:jaccard}). In this figure, each row provides results for a
specific tropical cyclone, and each box in that row shows the measured Jaccard
coefficient between two of the exposure metrics for that storm.  For most
tropical cyclones, distance-based county-level exposure assessment was at best
in moderate agreement with exposure assessments based on the four hazard-based
metrics. For almost all tropical cyclones, Jaccard index values between
distance and any of the four hazard-based metrics were~$\le0.6$ (i.e., out of
the counties assessed as exposed to the tropical cyclone by at least one of the
two metrics considered,~60\si{\percent} or fewer of the counties had the same
exposure assessment under the two metrics). For many tropical cyclones, the
Jaccard index values were~$\le0.2$, suggesting very poor agreement in exposure
assessment between the metrics.  Among the four hazard-based metrics, the
tornado-based metric showed universally poor agreement with other metrics in
county-level classification across all tropical cyclones considered.  For other
pairs of metrics, there were also generally large differences in which counties
were determined to be exposed, with the Jaccard index values~$<0.4$ for most
metric pairs in most tropical cyclones.

\begin{comment}
Storms are displayed within
clusters that have similar patterns in county-level exposure agreement for
metric pairs, based on hierarchical clustering using the complete link
method~\parencite{murtagh2012algorithms} (i.e., storms in the same cluster tend
to have similar patterns for the pairwise strength of agreement among metrics);
columns are also ordered based on hierarchical clustering. 
Maps are
available showing the counties identified as exposed under each of five metrics
for the widest-extent storm in each cluster: Hurricane Ivan in~2004
(Figure~\ref{fig:ivanexposure}) and Hurricanes Floyd in~1999, Lee in~2011, Cindy
in~2005, and Katrina in~2005 (Figure~S3).
\end{comment}

There were, however, a few exceptions\,---\,tropical cyclones in which similar
counties were determined to be exposed to the tropical cyclone for two or more
of the metrics considered.  For Floyd in~1999 (Figure~S3) and Irene in~2011,
for example, county-level classification agreed moderately to well (Jaccard
index of~$\sim 0.5\mbox{\,--\,}0.8$) for all pairs of exposure metrics except
those including the tornado-based metric. For another set of tropical cyclones
(e.g., Lee in~2011 [Figure~S3], Ernesto in~2006, and Bertha in~1996), there was
moderate to good agreement for all pairwise combinations of distance, rain, and
wind, but poor agreement for all other combinations of metrics. However, these
two sets of tropical cyclones represented the minority of all tropical cyclones
considered; for most tropical cyclones, there was low overlap in the counties
determined to be exposed to the tropical cyclones between most pairings of
metrics.

\subsection*{Case study}

Finally, we conducted a case study to investigate how these differences in
exposure assessment across the exposure metrics could affect estimates of
expected physical exposure among a vulnerable subpopulation,
electricity-dependent Medicare beneficiaries.
Figure~\ref{fig:topelecdependexposure} shows the study counties with the
highest expected physical exposures in this subpopulation based on each
exposure metric. A few counties (Miami-Dade County, FL; Harris County, TX) were
ranked in the top three counties for expected physical exposure in this
subpopulation for almost all exposure metrics
(Figure~\ref{fig:topelecdependexposure}). However, there were key differences
across exposure metrics. For example, most of the top counties based on wind-
and tornado-based metrics were in Florida, while over half of the counties with
highest expected physical exposures for the flood-based metric were in
Mid-Atlantic states (PA, NY).



