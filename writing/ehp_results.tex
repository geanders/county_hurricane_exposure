\section*{Results}

\subsection*{Patterns in tropical cyclone exposures in eastern \ac{US} counties}

We first calculated summary statistics for the exposure assessments
(Table~\ref{tab:exposuresummaries}). There was wide variation in the average
number of county exposures per year under the five different tropical cyclone
metrics considered. For the tornado metric, there were approximately~50 county
exposures identified per year, on average, within our study.  For flood and
wind metrics, there were substantially more county exposures identified ($>$200
per year on average), and even more when assessing exposure using the distance
and rain metrics ($>$400 per year on average). 

We also measured the average number of county exposures per tropical cyclone,
within those tropical cyclones in which at least one county was exposed to the
hazard (Table~\ref{tab:exposuresummaries}). The typical number of county
exposures per tropical cyclone was lowest when assessing exposure using the
tornado metric (median:~7 county exposures per tropical cyclone), higher under
the flood and wind metrics (median:~26 and~34, respectively), and even higher
for the metrics of rain (median:~68) and distance (median:~84).  For every
metric except the tornado-based metric, we identified at least one tropical
cyclone with over~300 counties exposed.  However, the largest-extent tropical
cyclone varied by metric (e.g., Beryl in~1994 exposed the most counties based
on distance, Frances in~2004 based on rain, and Ike in~2008 based on wind).

\subsection*{Agreement in county-specific exposure classification to specific
tropical cyclones}

The counties that were identified as exposed to a tropical cyclone differed
substantially depending on which metric was used in exposure assessment.  This
was true both when comparing exposure assessment based on distance versus the
four hazard-specific metrics considered (wind, rain, floods, and tornadoes)
and also when comparing exposure assessments among the four hazard-specific
metrics. 

Figure~\ref{fig:ivanexposure} gives an example of how sensitive county-level
tropical cyclone exposure assessment is to the choice of metric, using as an
example Hurricane Ivan in 2004, which was record-breaking for its duration as a
major hurricane~\parencite{franklin2006atlantic}. When county-level exposure
was determined based on distance, the counties assessed as exposed followed the
tropical cyclone's track, including counties well inland. For the wind metric,
only counties near two of the tropical cyclone's landfalls were assessed as
exposed. For rain- and flood-based metrics, however, exposure extended to the
left of the track, including counties as far north as New York and Connecticut,
while for the tornado metric, exposed counties tended to be to the right of the
track and included several counties in central North Carolina, South Carolina,
and Georgia that were not identified as exposed to Ivan based on any other
metric.  Figure~S3\todo{Check fig ref} provides similar maps for four other
example tropical cyclones (selected because they exposed many \ac{US}  counties
based on at least one metric), and Figure~S4\todo{Check fig ref} provides four
examples of tropical cyclones in which the disagreement in exposure assessment
between rain- and wind-based metrics was particularly notable.

We drew similar conclusions when we investigated all~46 tropical cyclones
between~1996 and~2011 (the period for which all five metrics were available)
for which~250 or more counties were exposed based on at least one metric
(Figure~\ref{fig:jaccard}). In this figure, each row provides results for a
specific tropical cyclone, and each box in that row shows a measure of
agreement between a specific pair of metrics in terms of the counties assessed
as exposed to that tropical cyclone, with darker colors representing greater
agreement in county-level exposure assessments.  

For most tropical cyclone, distance-based county-level exposure assessment was
at best in moderate agreement with exposure assessments based on the four
hazard-based metrics. For almost all tropical cyclones, Jaccard index values
between distance and any of the four hazard-based metrics were~0.6 or lower
(i.e., out of the counties assessed as exposed to the tropical cyclone by at
least one of the two metrics considered,~60\si{\percent} or fewer of the
counties had the same exposure assessment under the two metrics). For many
tropical cyclones, the Jaccard index values were~0.2 or lower, suggesting very
poor agreement in exposure assessment between the metrics.  Among the four
hazard-based metrics, the tornado-based metric showed universally poor
agreement with other metrics in county-level classification across all tropical
cyclones considered.  For other pairs of metrics, there were also generally
large differences in which counties were determined to be exposed to the
tropical cyclone, with the Jaccard index values below~0.4 for most metric pairs
in most tropical cyclones.

There were, however, a few exceptions\,---\,tropical cyclones in which similar
counties were determined to be exposed to the tropical cyclone for two or more
of the metrics considered.  For Floyd in~1999 (Figure~S3\todo{Check fig ref})
and Irene in~2011, for example, county-level classification agreed moderately
to well (Jaccard index of approximately~0.5\,--\,0.8) for all pairs of exposure
metrics except those including the tornado-based metric. These tropical
cyclones both made their first \ac{US}  landfall in North Carolina at minor
hurricane windspeeds (Category~2 and~1, respectively) and then skimmed the
eastern coast of the \ac{US} north through New England, bringing large rainfall
to much of the eastern coast from North Carolina north and causing extensive
inland flooding in North Carolina (Floyd) and New England
(Irene)~\parencite{avila2013atlantic, lawrence2000atlantic}. For another set of
tropical cyclones (e.g., Lee in~2011 [Figure~S3\todo{Check fig ref}], Ernesto
in~2006, and Bertha in~1996), there was moderate to good agreement for all
pairwise combinations of distance, rain, and wind, but poor agreement for all
other combinations of metrics. Most of these tropical cyclones either followed
the eastern coastline north of North Carolina (Bertha in~1996, Charley in~2004,
Gaston in~2004, Barry in~2007, and Hanna in~2008), in a pattern similar to
Hurricanes Floyd in~1999 and Irene in~2011, or made landfall in or near the
Florida panhandle and cut across to exit into the Atlantic around North
Carolina (Josephine in~1996, Earl in~1998, Alberto in~2006). However, these two
sets of tropical cyclones represented the minority of all tropical cyclones
considered; for most tropical cyclones, there was low overlap in the counties
determined to be exposed to the tropical cyclones between most pairings of
metrics.

\subsection*{Geographic patterns in tropical cyclone exposures}

When the average number of exposures per decade in each county were calculated,
strong geographical patterns were clear across tropical cyclone exposure
assessments based on these different metrics
(Figure~\ref{fig:averageexposure}). Wind-based exposure had a strong coastal
pattern, with almost all exposures in counties within about~200
\si{\kilo\metre} (124~mi) of the coastline. While exposures by rain- and
distance-based metrics were also more common in coastal areas compared to
inland areas, inland exposures were much more common based on these two metrics
compared to the wind-based metric. Rain and, to some extent, flood exposures
were characterized by a pattern defined by the Appalachian Mountains, with much
lower exposures west of the mountain range than to the east. Almost all
tornado-based exposures were in coastal states, with most in Florida, Alabama,
South Carolina, and North Carolina and almost none north of Maryland.
Flood-based exposures were highest in North Carolina and along the South
Carolina coast, as well as further north in New Jersey, southeastern
Pennsylvania, and southeastern New York. Notably, some areas of the \ac{US}
that experienced regular exposures to tropical cyclone-related excessive
rainfall and floods were not identified as areas of regular risk based on the
more commonly used tropical cyclone exposure assessment metrics of wind or
distance (Figure~\ref{fig:averageexposure}). 

We conducted a case study to investigate how these differences in exposure
assessment across the five metrics considered could strongly influence
health-related studies based on physical exposure to tropical cyclones. For
this example, we compared estimates of which \ac{US} counties have the largest
expected physical exposure to tropical cyclones among a susceptible
subpopulation, Medicare beneficiaries who are reliant on electricity for
medical equipment (Figure~S6\todo{Check fig ref})~\parencite{empower}.  We
found that a few counties (Miami-Dade County, FL; Harris County, TX) were
ranked in the top three counties of physical exposure among
electricity-dependent Medicare beneficiaries for almost all exposure metrics
(Figure~\ref{fig:topelecdependexposure}). However, there were key differences
when using different exposure metrics. For example, Florida counties dominated
the lists of top exposed counties based on wind- and tornado-based metrics,
while over half of the counties for the flood-based metric were in Mid-Atlantic
states (PA, NY).

\subsection*{Software}

To assist with future tropical cyclone impact studies, we created and published
open source software that includes the hazard-specific, county-level tropical
cyclone data in this paper, as well as tools to explore and map that 
data~\parencite{hurricaneexposure, hurricaneexposuredata}. This software includes continuous
data for the distance-, rain-, and wind-based metrics, allowing users to also explore
different thresholds of these metrics when assessing exposure.

We investigated how well these data correspond with data from other available
sources (Figures~S7\,--\,11\todo{Check fig refs}). Within the nine sample
counties, storm-related rainfall from the reanalysis precipitation dataset was
well-correlated with station-based rainfall measurements, although the
reanalysis data did tend to somewhat underestimate rainfall totals in storms
with extremely high rainfall (Figure~\ref{fig:raincomparison}). A distance
constraint (county population mean center within~500 \si{\kilo\metre} of the
storm's closest approach) was necessary for assessing storm exposure based on
this metric to ensure that rains unrelated and far from the storm track were
not misattributed to storm exposures. We found that this distance constraint
was typically adequately large to capture storm-related rain, although in rare
examples of exceptionally large storms (e.g., Hurricane Ike in 2008) or storms
for which tracking was stopped at extratropical transition (e.g., Tropical
Storm Lee in 2011), some storm-related rains were missed because of this
distance constraint (Figure~\ref{fig:rainexamples}).  The modeled wind speeds
generally agreed well with the \ac{HURDAT2} wind radii
(Figure~\ref{fig:windcomparison}), with disagreement mostly limited to large
storms that were characterized by high winds well inland (e.g., Hurricanes Ike
in 2008 and Sandy in 2012; Figure~\ref{fig:windexamples}).  Stream gage data
generally indicated higher discharge in these counties on dates identified as
flood events, particularly in counties with more reporting stream gages (e.g.,
Harris County, TX, with~10 reporting stream gages;
Figure~\ref{fig:floodcomparison}), although these two flooding data sources
were not completely consistent.

While generally in agreement with data from other sources, there are a few
limitations. The wind data are based on modeled, rather than observed, values,
and while the modeled wind data generally agree well with post-analysis maximum
wind radii from the revised Atlantic hurricane database
\ac{HURDAT2}~\parencite{landsea2013} (Figure~S7\todo{Check fig ref}), for a few
tropical cyclones it did not (e.g., Figure~S10\todo{Check fig ref}). The
rainfall data are from re-analysis data, which are generally well-correlated
with observed ground-based station data but sometimes oversmooth extreme
measurements (Figure~S8\todo{Check fig ref}). 
