\section*{Methods and Materials}

All counties in the eastern half of the \ac{US}
(Figure~\ref{fig:ivanexposure}), as of the~2010 \ac{US} Decennial Census, were
included in the study. The study covered all tracked storms in the \ac{HURDAT2}
\parencite{landsea2013} that came within~250 \si{\kilo\metre} of at least one
study county between~1988 and~2015 (Figure~S2\todo{Check fig ref}).
\ac{HURDAT2} is a post-storm assessment conducted by the \ac{US} \ac{NHC} and
incorporates data from a variety of sources, including satellite data and, when
available, aircraft reconnaissance data~\parencite{landsea2013, jarvinen1988}.
These data typically give measurements of the tropical cyclone center's
location at~6-\si{\hour} intervals at synoptic times (i.e., 6:00~am, 12:00~pm,
6:00~pm, and 12:00~am \ac{UTC}); some landfalling tropical cyclones have an
additional observation at the time of landfall~\parencite{landsea2013}.

\subsection*{Distance-based exposure metric}

For each storm, the distance of closest approach was measured for every study
county.  Storm center track location data from \ac{HURDAT2} were interpolated
using natural cubic splines to~15-\si{\minute} intervals~\parencite{hurricaneexposure}.
The distance between each county's population mean center, as assessed by the
2010 Decennial Census~\parencite{countycenters}, and each~15-\si{\minute} interpolated
storm center location was measured using Great Circle (WGS84 ellipsoid)
distance~\parencite{bivand2013applied}. The distance-based exposure metric was
based on the minimum value of this storm center-to-county distance throughout
the period the storm was tracked. The time of closest approach for each
county-storm pairing was also recorded for use assessing rain-, flood-, and
tornado-based metrics. To allow matching with data based on local time, times
were converted from \ac{UTC} to local time~\parencite{countytimezones}.

\subsection*{Rain-based exposure metric}

For the rain-based exposure metric, cumulative rainfall was estimated for the
period from two days before to one day after the date of a tropical cyclone's
closest approach to a county. Daily county-level rainfall was determined by
aggregating hourly~1/8\si{\degree} gridded reanalysis data from the \ac{NLDAS-2}
precipitation data files~\parencite{rui2013nldas}. The \ac{NLDAS-2} data integrate
satellite-based and land-based monitoring, applying a land-surface model to
create a reanalysis dataset that is spatially and temporally complete across
the continental \ac{US}~\parencite{rui2013nldas, alhamdan2014environmental}. To
aggregate to county-level, hourly data at each grid point were summed to create
a daily rainfall total, and these grid point rainfall totals were then averaged
across all grid points within a county's~1990 \ac{US} Census boundaries to
create a county-level daily total~\parencite{alhamdan2014environmental, cdcwonder}.
This county-level precipitation data are publicly available through the \ac{US}
\ac{CDC}'s \ac{WONDER} database~\parencite{cdcwonder, alhamdan2014environmental}.
The rainfall-based exposure metric was only available for tropical cyclones
through~2011, the period for which the county-aggregated reanalysis rainfall
data were available~\parencite{cdcwonder, alhamdan2014environmental}. 

To assess the rainfall estimates from the reanalysis data used in analysis in
the the main text, we compared cumulative rainfall estimates based on this
reanalysis rainfall data to rainfall observations from ground-based weather
stations in nine sample study counties, selected as counties with data
available from multiple stations. In these nine counties, observed rainfall was
collected for all storms,~1988\,--\,2011, for the same period of two days
before to one day after each storm's closest approach, using all stations with
available daily data in the Global Historical Climatology Network throughout
1988\,--\,2011~\parencite{menne2012overview, rnoaa, countyweather}. Agreement
between measured storm-related rainfall between these two data sources was
assessed by Spearman's rank correlation. 

\subsection*{Wind-based exposure metric}

Ground-based observations of wind speed are problematic during tropical
cyclones, as instruments often fail at high wind speed, while reanalysis data,
often available at hourly or higher resolution, can be too smooth to capture
wind extremes associated with a tropical cyclone. Therefore, to create a
dataset of county-level sustained winds during historical tropical cyclones, we
modeled maximum sustained wind speeds at each county's population mean
center~\parencite{countycenters}. \ac{HURDAT2} Best Track data were interpolated
from~6-\si{\hour} reported values to~15-\si{\minute} increments using natural cubic
splines~\parencite{stormwindmodel}, and then sustained wind speed was modeled for
each county center at each~15-\si{\minute} increment using a double exponential wind
speed model~\parencite{willoughby2006parametric} to estimate maximum ground-level
sustained wind speed at each county center~\parencite{stormwindmodel}. Asymmetry in
wind speeds around the tropical cyclone center was incorporated into the wind
speed model~\parencite{phadke2003modeling, stormwindmodel}. The maximum value of
this modeled wind speed was determined for each county as the maximum windspeed
across the~15-\si{\minute} incremented modeled values throughout the tropical
cyclone.

To validate the metric of county-level wind speed used in the main text, we
compared this wind metric to estimated wind radii from \ac{HURDAT2}, available
for storms in our study from~2004~\parencite{landsea2013}. \ac{HURDAT2} gives
estimates of the maximum radii to winds of~64,~50, and~34 \si{\knot} in each four
quadrants of a storm (i.e., distance from the storm's center to the furthest
point in a quadrant with winds at or above that speed) at each~6-\si{\hour} storm
observation. 

\subsection*{Flood- and tornado-based exposure metrics}

To identify flood- and tornado-based exposures to tropical cyclones in \ac{US}
counties, we used event listings from the National Oceanic and Atmospheric
Administration (NOAA)'s Storm Event Database~\parencite{stormevents}. For each
tropical cyclone, we identified all events with event types related to flooding
(``Flood", ``Flash Flood", ``Coastal Flood") and tornadoes (``Tornado") and
that occurred in a county within~500 \si{\kilo\metre} of the tropical cyclone's track
and with a start date within a five-day window centered on the date of the
tropical cyclone's closest approach to the
county~\parencite{hurricaneexposuredata}. ``Flood", ``Flash Flood" and ``Tornado"
events in this database were reported by county \ac{FIPS} code. ``Coastal
Flood" events were reported by forecast zone; for these, the event was matched
to the appropriate county if possible~\parencite{noaastormevents}. While this
database has recorded storm data since~1950, its coverage changed substantially
in~1996 to cover a wider variety of storm events~\parencite{stormevents}. We
therefore only considered these metrics of hurricane exposure for tropical
cyclones in~1996 and later.

As validation for the flood data, we compared streamflow data from \ac{US}
Geological Survey gages~\parencite{usgsgages, countyfloods, dataRetrieval} in nine
counties with tropical cyclone flood classifications based on the NOAA Storm
Events Database (Figure~\ref{fig:floodcomparison}).

\subsection*{Assessing agreement in exposure classification between
metrics}

Two of the exposure metrics (flood- and tornado-based) were inherently binary,
since these metrics were based on whether an event was listed in the NOAA Storm
Events database or not.  For the other exposure metrics, each county was
classified as exposed to a tropical cyclone based on whether the exposure
metric exceeded a certain threshold (Table~\ref{tab:exposuremetrics}). We
assessed county-level tropical cyclone exposure in the eastern \ac{US} based on
each exposure metric for all land-falling or near-land Atlantic basin tropical
cyclones. Depending on available exposure data, this assessment included some
or all of the period from~1988 to~2015, a period with~136 tropical cyclones
that made \ac{US} landfall or passed within~250 \si{\kilo\metre} (155 mi) of at
least one \ac{US} county (Figure~S2\todo{Check fig ref}).

We measured agreement between exposure metrics in the classification of a
county as exposed or unexposed to a specific tropical cyclone. For this
measurement, we calculated the within-storm Jaccard
index~\parencite{jaccard1901distribution, jaccard1908nouvelles} between every
pair-wise combination of hazard metrics.  The Jaccard index~($J_s$) measures
similarity between two metrics ($X_{1,s}$ and~$X_{2,s}$) for tropical
cyclone~$s$ as the proportion of counties in which both of the metrics classify
the county as exposed out of all counties classified as exposed by at least one
of the metrics:

\begin{equation} 
J_s = \frac{X_{1,s} \cap X_{2,s}}{X_{1,s} \cup X_{2,s}}
\end{equation}

\noindent This metric can range from~0, in the case of no overlap between the
counties classified as exposed based on the two metrics, to~1, in the case that
the two metrics classify exactly the same counties as exposed to the tropical
cyclone.

\subsection*{Case study: Physical exposure of electricity-dependent Medicare
beneficiaries to tropical cyclones}

The outcomes of disasters depend both on the geophysical forces of the disaster
as well as on the vulnerability of the society living in the geographical areas
affected by those forces.~\parencite{chakraborty2005population,
anderson2003community, cutter1996vulnerability} As a case study of how
differences in tropical cyclone exposure assessments across different hazard
metrics influence estimates of physical exposures of susceptible populations to
tropical cyclones, we investigated how the use of different metrics influenced
estimates of which eastern \ac{US} counties have the highest expected average
exposures of electricity-dependent Medicare beneficiaries to tropical cyclones.
This subpopulation was selected for this analysis since it may be particularly
susceptible to health impacts from tropical cyclone exposure, especially
through the pathway of storm-associated power outages and evacuations. We
collected data on the number of electricity-dependent Medicare beneficiaries in
each study county from the U.S. Department of Health \& Human Service's emPower
Map~2.0~\parencite{empower}. We then calculated the physical exposure of the
electricity-dependent Medicare population in each county, based on tropical
cyclone assessments using each hazard metric,
following~\parencite{peduzzi2009assessing}:

\begin{equation}
E_c = F_c * P_c
\end{equation}

\noindent where~$E_c$ is the average yearly physical exposure among
electricity-dependent Medicare beneficiaries in county~$c$ to tropical cyclone
exposures based on a given metric,~$F_c$ is the estimated yearly expected
frequency of tropical cyclone exposures in county~$c$ based on that metric,
and~$P_c$ is the size of the electricity-dependent Medicare population in
county~$c$, as of July~2017. When combined with estimates of vulnerability of a
population to a natural hazard, such measurements of physical exposure can be
used to calculate risk of human losses from the
hazard~\parencite{peduzzi2009assessing}.



