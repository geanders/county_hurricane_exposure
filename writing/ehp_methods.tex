\section*{Methods and Materials}

This study covered all counties in the eastern half of the \ac{US}
(Figure~\ref{fig:hurrtracks}), as of the~2010 \ac{US} Decennial Census. The
study covered all tracked land-falling or near-land Atlantic-basin tropical
cyclones between~1988 and~2015, considering all 136 storms in \ac{HURDAT2}
\parencite{landsea2013} that came within~250 \si{\kilo\metre} of at least one
eastern \ac{US} county in this period (Figure~\ref{fig:hurrtracks}). 

\begin{comment}
These data typically give measurements of
the tropical cyclone center's location [and central pressure / maximum
windspeed?] at~6-\si{\hour} intervals at synoptic times (i.e., 6:00~am,
12:00~pm, 6:00~pm, and 12:00~am \ac{UTC}); some landfalling tropical cyclones
have an additional observation at the time of landfall~\parencite{landsea2013}.
\end{comment}

\subsection*{Distance-based exposure metric}

We first measured how close each study storm came to each study county. We
started with tracking data from \ac{HURDAT2}, which records the storm's
position every~6 \si{\hour}, and interpolated to~15-\si{\minute} intervals
using natural cubic splines~\parencite{hurricaneexposure}. We then measured the
distance between the storm's position at each~15-\si{\minute} interval and each
county's population mean center, as assessed by the 2010 Decennial
Census~\parencite{countycenters}, using the Great Circle (WGS84 ellipsoid)
method~\parencite{bivand2013applied}. These measurements captured the distance
between the storm's central track and each study county throughout the period
when the storm was tracked. The closest distance between the storm and each
county was recorded as the distance of the storm's closest approach to the
county. We also recorded the time storm came closest to each county, for use in
assessing rain-, flood-, and tornado-based metrics where observed data on the
hazards needed to be matched by time to storm events.  To allow matching with
data based on local time, times were converted from \ac{UTC} to local
time~\parencite{countytimezones}.

\subsection*{Rain-based exposure metric}

To estimate storm-associated rainfall, we used precipitation data from a
re-analysis dataset, as this data was available for every county and every day
through a continuous period of the study
(1988\,--\,2011)~\parencite{alhamdan2014environmental, cdcwonder}.  By
comparison, observations from ground-based monitoring networks had missing
values spatially (i.e., for some study counties), temporally (for some days),
or both.  

We estimated daily rainfall for each study county by aggregating
hourly~1/8\si{\degree} gridded reanalysis data from the \ac{NLDAS-2}
precipitation data files~\parencite{rui2013nldas}. These data integrate
satellite-based and land-based monitoring, applying a land-surface model to
create a reanalysis dataset that is spatially and temporally complete across
the continental \ac{US}~\parencite{rui2013nldas, alhamdan2014environmental}. To
aggregate to county-level, the hourly data at each grid point were summed to
create a daily rainfall total, and these grid point rainfall totals were then
averaged across all grid points within a county's~1990 \ac{US} Census
boundaries~\parencite{alhamdan2014environmental, cdcwonder}. These daily
county-level estimates were then matched temporally with storm tracks, using
the date when each storm was closest to each county. The cumulative
storm-associated rainfall was then calculated as the sum of rainfall from two
days before the storm's closest approach to a county to one day after, as 
storm-related rains often precede the passing of the storm's center. 

To validate these rainfall estimates, we investigated a subset of counties that
also had available ground-based precipitation observations. We selected nine
sample counties geographically spread across storm-prone regions of the eastern
\ac{US} and for which precipitation data were available from multiple stations
throughout the study period: Miami-Dade, FL; Harris County, TX; Mobile County,
AL; Orleans Parish, LA; Fulton County, GA; Charleston County, SC; Wake County,
NC; Baltimore County, MD; and Philadelphia County, PA. We collected
ground-based observations of precipitation data from all stations in the county
with available daily data in the Global Historical Climatology Network
throughout 1988\,--\,2011~\parencite{menne2012overview, rnoaa, countyweather}.
We summed daily station-specific measurments for two days before to one day
after each storm's closest approach and then averaged these cumulative
station-based precipitation totals for each county as a county-wide estimate of
cumulative storm-related precipitation. We measured the rank correlation
(Spearman's~$\rho$~\parencite{spearman1904proof}) between storm-specific
cumulative precipitation estimates for the two data sources within each sample
county.

\subsection*{Wind-based exposure metric}

Ground-based observations of wind speed are problematic as a metric of exposure
to tropical cyclones, as instruments often fail at high wind speed, while
reanalysis data, often available at hourly or higher resolution, can lack a
fine enough temporal resolution to capture wind extremes associated with a
tropical cyclone. Therefore, to create a dataset of county-level sustained
winds during historical tropical cyclones, we modeled maximum sustained wind
speeds at each county's population mean center~\parencite{countycenters}, using
a wind speed model based on results from
Willoughby~\parencite{willoughby2006parametric}. For each storm, we used the
interpolated storm tracks generated for the distance metric and, for
each~15-\si{\minute} increment, we modeled maximum ground-level sustained wind
speed at each county center using a double exponential wind speed model with
inputs for the storm's forward speed, direction, and central wind
speed~\parencite{willoughby2006parametric, stormwindmodel}. This model
incorporated asymmetry in wind speeds around the tropical cyclone center that
results from the storm's forward movement~\parencite{phadke2003modeling,
stormwindmodel}. We then determined the maximum value of this estimate over the
storm tracking period for each study county to determine the maximum sustained
surface wind speed in each county during each storm.

To validate these modeled county-level wind speed estimates, we compared them
to county-level maximum sustained surface wind speeds based on the wind radii
estimates in \ac{HURDAT2}, which are based on \ldots. The wind radii values
have been included in \ac{HURDAT2} since 2004 [?] and estimate the distance
from the storm's center to the furthest point with winds at or above winds of a
certain speed in each four quadrants of the storm. They give estimates of these
maximum radii for three thresholds of maximum surface wind speeds: 64,~50,
and~34 \si{\knot}.  They therefore allow for the classification of counties
into four categories of maximum sustained wind speeds: $<$34~\si{\knot};
34\,--\,49.9~\si{\knot}; 50\,--\,63.9~\si{\knot}; $\ge$64~\si{\knot}. We
interpolated this data to~15-\si{\minute} increments and classified a county as
exposed to winds in a given wind speed category if its population mean center
was within~85\% of the maximum radius for that wind speed in its quadrant of
the storm [clarify why 85\%]. We compared these wind radii-based estimates with
the modeled wind speed estimates for the study storms since 2004 for which at
least one study county had a sustained wind of~$\ge$34 \si{\knot} (based on the
\ac{HURDAT2} wind radii). For each of these storms, we calculated the percent
of study counties that were classified in the same wind speed category.

\subsection*{Flood- and tornado-based exposure metrics}

To identify flood- and tornado-based tropical cyclone exposures in \ac{US}
counties, we used event listings from the National Oceanic and Atmospheric
Administration (NOAA)'s Storm Event Database~\parencite{stormevents}. For each
tropical cyclone, we identified all events with event types related to flooding
(``Flood", ``Flash Flood", ``Coastal Flood") and tornadoes (``Tornado") and
that occurred in a county within~500 \si{\kilo\metre} of the tropical cyclone's
track and with a start date within a five-day window centered on the date of
the tropical cyclone's closest approach to the
county~\parencite{hurricaneexposuredata}. ``Flood", ``Flash Flood" and
``Tornado" events in this database were reported by county \ac{FIPS} code and
so could be directly linked to counties.  ``Coastal Flood" events were reported
by forecast zone; for these, the event was matched to the appropriate county if
possible using regular expression matching of listed county
names~\parencite{noaastormevents}. While this database has recorded storm data
since~1950, its coverage changed substantially in~1996 to cover a wider variety
of storm events~\parencite{stormevents}. We therefore only considered these
metrics of hurricane exposure for tropical cyclones in~1996 and later.

The tornado observations from this dataset form the traditional tornado event
database for the \ac{US}, and so we did not conduct analyses to validate the
tornado event data. It is difficult to characterize flooding at the county
level because flooding can be very localized and can be triggered by a variety
of causes. To investigate the extent to which the NOAA flood event data
captures extremes that might be identified with other flooding data sources, we
investigated a sample of study counties, comparing the flood event data during
tropical storms with streamflow measurements at \ac{US} Geological Survey
county streamflow gages \parencite{usgsgages, countyfloods, dataRetrieval}.  

We considered nine study counties, selecting counties geographically spread
through storm-prone areas of the eastern \ac{US} and with multiple streamflow
gages reporting data during events (Baltimore County, MD; Bergen County, NJ;
Escambia County, FL; Fairfield County, CT; Fulton County, GA; Harris County,
TX; Mobile County, AL; Montgomery County, PL; and Wake County, NC). For each
county, we first identified all streamflow gages in the county that had
complete data for Jan.~1~1996\,--\,Dec.~31,~2015. If a storm did not come
within~500~\si{\kilo\metre} of a county, it was excluded from this analysis,
but all other study storms were considered. For each storm and county, we
summed daily total streamflow across all streamgages for each day in the
five-day window around the storm's closest approach. We took the maximum daily
streamflow total during those five days as a measure of the county's maximum
streamflow during that storm. We also calculated the percent of streamflow
gages in the county with a daily streamflow that exceeded a threshold of
flooding (the streamgage's median value for annual peak
flow~\parencite{countyfloods}) on any day during the five-day window. We
investigate how these measurements varied between storms with associated flood
events in the NOAA Storm Events data versus storms without an event listing, to
explore if storms with flood event listings tended to be  associated with
higher streamflows at gages within the county.

\subsection*{Binary storm exposure classifications based on different metrics}

In our open-source dataset, we provide the closest distance of each storm to
each county, as well as county-level storm-related winds and rainfall, as
continuous metrics. However, epidemiologic studies of tropical cyclones often
compare ``exposed'' versus ``unexposed'' communities to assess storm-related
health risks [refs], and so we also used these continuous metrics to classify
counties as ``exposed'' or ``unexposed'' and explored patterns in storm
exposure based on these binary clasifications. 

Two of the exposure metrics (flood- and tornado-based) were inherently binary,
since these metrics were based on whether an event was listed in the NOAA Storm
Events database.  For the other exposure metrics, each county was classified as
exposed to a tropical cyclone based on whether the exposure metric exceeded a
certain threshold (Table~\ref{tab:exposuremetrics}). For the rainfall metric, a
distance constraint was also necessary, to ensure that rains unrelated and far
from the storm track were not misattributed to a storm. Through exploratory
analysis, we set this distance metric at~500 \si{\kilo\metre} (i.e., for a
county to be classified as exposed based on rainfall, the cumulative rainfall
had to be over 75 \si{\milli\metre} and the storm must have passed within~500
\si{\kilo\metre} of the county; Table~\ref{tab:exposuremetrics}). We set this
distance constraint at a value that was typically large enough to capture
storm-related rain. However, data users should note that in rare examples of
exceptionally large storms (e.g., Hurricane Ike in 2008) or storms for which
storm tracking was stopped at extratropical transition (e.g., Tropical Storm
Lee in 2011), some storm-related rains may be missed because of this distance
constraint (Figure~S7). 

We assessed patterns in county-level tropical cyclone exposure in the eastern
\ac{US} based on each exposure metric. Depending on available exposure data,
this assessment included some or all of the period from~1988 to~2015 (Table 1).
For each binary metric of tropical cyclone exposure (Table 1), we first summed
the total number of county-level exposures over available years for each
exposure metric and mapped patterns in these exposures. We next investigated
agreement between storm exposure classifications based on different exposure
metrics. We calculated the within-storm Jaccard
index~\parencite{jaccard1901distribution, jaccard1908nouvelles} between each
pair of exposure metrics. The Jaccard index~($J_s$) measures similarity between
two metrics ($X_{1,s}$ and~$X_{2,s}$) for tropical cyclone~$s$ as the
proportion of counties in which both of the metrics classify the county as
exposed out of all counties classified as exposed by at least one of the
metrics:

\begin{equation} 
J_s = \frac{X_{1,s} \cap X_{2,s}}{X_{1,s} \cup X_{2,s}}
\end{equation}

\noindent This metric can range from~0, in the case of no overlap between the
counties classified as exposed based on the two metrics, to~1, in the case that
the two metrics classify exactly the same set of counties as exposed to the
tropical cyclone. We measured these values for all study storms that affected a
large number of eastern \ac{US} counties (250 or more of the study counties
classified as exposed by at least one of the exposure metrics) during the years
when all exposure data were available (1996\,--\,2011).

\subsection*{Case study: Physical exposure of electricity-dependent Medicare
beneficiaries to tropical cyclones}

Disasters' societal impacts depend both on the geophysical forces of the
disaster and on the vulnerability of those living in the affected geographical
areas~\parencite{chakraborty2005population, anderson2003community,
cutter1996vulnerability}. As a case study, we explored how differences in
tropical cyclone exposure assessments across different exposure metrics might
influence estimates of physical exposures of a susceptible population to
tropical cyclones. We calculated expected average physical exposures among
electricity-dependent Medicare beneficiaries in eastern \ac{US} counties to
tropical cyclones based on each exposure metric. This subpopulation was
selected since it may be particularly susceptible to health impacts from
tropical cyclone exposure, especially through the pathway of storm-associated
power outages and evacuations. 

We collected data on the number of electricity-dependent Medicare beneficiaries
in each study county from the \ac{US} Department of Health \& Human Service's
emPower Map~2.0~\parencite{empower}. We then calculated the physical exposure
of the electricity-dependent Medicare population in each county, based on
tropical cyclone assessments using each hazard metric,
following~\parencite{peduzzi2009assessing}:

\begin{equation}
E_c = F_c * P_c
\end{equation}

\noindent where~$E_c$ is the average yearly physical exposure among
electricity-dependent Medicare beneficiaries in county~$c$ to tropical cyclone
exposures based on a given metric,~$F_c$ is the estimated yearly expected
frequency of tropical cyclone exposures in county~$c$ based on that metric,
and~$P_c$ is the electricity-dependent Medicare population in
county~$c$, as of July~2017. 

\begin{comment}
When combined with estimates of vulnerability of a
population to a natural hazard, such measurements of physical exposure can be
used to calculate risk of human losses from the
hazard~\parencite{peduzzi2009assessing}.
\end{comment}


