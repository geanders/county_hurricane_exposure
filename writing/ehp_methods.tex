\section*{Methods and Materials}

This study covered all counties in the eastern half of the \ac{US}
(Figure~\ref{fig:hurrtracks}) as of the~2010 \ac{US} Decennial Census. The
study covered all tracked land-falling or near-land Atlantic-basin tropical
cyclones between~1988 and~2015 by considering all 136 storms in \ac{HURDAT2}
\parencite{landsea2013} that came within~250 \si{\kilo\metre} of at least one
eastern \ac{US} county in this period (Figure~\ref{fig:hurrtracks}). 

\subsection*{Distance-based exposure metric}

We first measured how close each storm came to each county. We used tracking
data from \ac{HURDAT2}, which records the storm center's position every~6
\si{\hour} (at the synoptic times of 6:00~am, 12:00~pm, 6:00~pm, and 12:00~am
\ac{UTC}), and interpolated this position to~15-\si{\minute} intervals using
natural cubic splines~\parencite{hurricaneexposure}. At each~15-\si{\minute}
interval, we measured the distance between the storm's center and each county's
population mean center~\parencite{bivand2013applied, countycenters}. From these
measurements, we determined how close each storm came to each county. 

We also recorded the time the storm came closest to each county. We later used
this time to link observed data on precipitation, flooding, and tornadoes to
the storm.  To allow matching with data recorded in local time (e.g., health
data), we converted these times from \ac{UTC} to local
time~\parencite{countytimezones}.

\subsection*{Rain-based exposure metric}

We based storm rainfall measurements on precipitation data files from the
\ac{NLDAS-2} re-analysis dataset, which is available for the continental
\ac{US}~\parencite{rui2013nldas}. We used data that were previously aggregated
from this re-analysis dataset to the county level for the Centers for Disease
Control and Prevention's \ac{WONDER} database \parencite{cdcwonder} by two of
the coauthors (Al-Hamdan and Crosson) \parencite{alhamdan2014environmental,
cdcwonder}. To create this aggregated data originally, these coauthors took
hourly precipitation measurements in the \ac{NLDAS-1} precipitation files,
which are given at~1/8\si{\degree} grid points across the continential
\ac{US}, and summed them by day for each grid point.  This created a daily
precipitation total for each grid point; these were then averaged for all grid
points within a county's~1990 \ac{US} Census
boundaries~\parencite{alhamdan2014environmental, cdcwonder}. This process
generated daily county-level precipitation estimates for each continental
\ac{US} county for 1988\,--\,2011 \parencite{cdcwonder}.  We matched these
county-level daily measurements by date with storm tracks, using for each
county and storm the date when the storm was closest to the county. Given the
location of storm-affected counties and the typical timing of tropical
cyclones, these precipitation measures primarily represent rainfall, although
occasionally they may represent snowfall or other types of precipitation.  

In the open-source data, we provide daily precipitation values for each county
for the period from five days before to three days after each storm's closest
approach to the county.  In the software that we published in association with
this data, we provide the functionality to aggregate these daily values to
create cumulative precipitation estimates for custom windows as needed. For
example, a user could determine precipitation only on the day the storm was
closest to each county, but could also determine the cumulative rainfall for a
more extended period, like from three days before to three days after the
storm's closest approach. For later analysis in this study, we calculated
storm-associated rainfall as the sum of precipitation from two days before the
storm's closest approach to the county to one day after.  

To validate these precipitation estimates, we compared them with ground-based
observations in a subset of study counties. We selected nine sample counties
geographically spread across storm-prone regions of the eastern \ac{US} and for
which precipitation data were available from multiple ground-based stations
within the Global Historical Climatology Network throughout the study
period~\parencite{menne2012overview, rnoaa, countyweather}. These sample
counties were: Miami-Dade, FL; Harris County, TX; Mobile County, AL; Orleans
Parish, LA; Fulton County, GA; Charleston County, SC; Wake County, NC;
Baltimore County, MD; and Philadelphia County, PA. We summed daily
station-specific measurements from two days before to one day after each
storm's closest approach and then averaged these cumulative station-based
precipitation totals for each county to create a county-wide estimate of
cumulative storm-related precipitation based on ground-based monitoring. We
measured the rank correlation (Spearman's~$\rho$~\parencite{spearman1904proof})
between storm-specific cumulative precipitation estimates for the two data
sources (\ac{NLDAS} re-analysis data versus ground-based monitoring) within
each sample county.

\subsection*{Wind-based exposure metric}

To create a dataset of county-level sustained winds during historical tropical
cyclones, we modeled maximum sustained wind speeds at each county's population
mean center~\parencite{countycenters}, using a wind speed model based on
results from Willoughby~\parencite{willoughby2006parametric}. For each storm,
we modeled the maximum ground-level sustained wind speed that the storm
generated at each county center for each~15-\si{\minute} increment throughout
the storm tracking period, using a double exponential wind speed model with
inputs for the storm's forward speed, direction, and central wind
speed~\parencite{willoughby2006parametric, stormwindmodel}. This model
incorporates the asymmetry in wind speeds around the tropical cyclone center
that results from the storm's forward movement~\parencite{phadke2003modeling,
stormwindmodel}. We then identified the maximum sustained surface wind speed in
each county during each storm.

To validate these modeled county-level wind speed estimates, we compared them
to the  post-season wind radii values, which have been routinely published
since 2004 \parencite{knaff2016using}. These wind radii characterize storm
size, estimating the maximum distance from the storm's center that winds of a
certain intensity extend. They give separate estimates for four quadrants of
the storm to capture asymmetry in the storm's wind field and are 
based on a post-season
re-analysis that incorporates available data (e.g., satellite data, aircraft
reconnaissance and ground-based data if available)
\parencite{knaff2016using}. Wind radii are estimated
for three thresholds of maximum surface wind speeds: 64,~50, and~34 \si{\knot}.
They therefore allow for the classification of counties into four categories of
maximum sustained wind speeds: $<$34~\si{\knot}; 34\,--\,49.9~\si{\knot};
50\,--\,63.9~\si{\knot}; $\ge$64~\si{\knot}. 

We interpolated the wind radii data to~15-\si{\minute} increments and
classified a county as exposed to winds in a given wind speed category if its
population mean center was within~85\% of the maximum radius for that wind
speed in that quadrant of the storm. The~85\% adjustment is based on previous
research that found that this ratio helps capture average wind extents within a
quadrant from these maximum wind radii \parencite{knaff2016using}.  We compared
these wind radii-based estimates of county-level maximum sustained surface
winds during a storm with the modeled wind speed estimates, comparing all study
storms since 2004 for which at least one study county experienced sustained
winds of~$\ge$34 \si{\knot} based on the \ac{HURDAT2} wind radii. For each of
these storms, we calculated the percent of study counties that were classified
in the same wind speed category based on both data sources.

\subsection*{Flood- and tornado-based exposure metrics}

To identify flood- and tornado-based tropical cyclone exposures in \ac{US}
counties, we used event listings from the National Oceanic and Atmospheric
Administration (NOAA)'s Storm Event Database~\parencite{stormevents}. While
this database has recorded storm data since~1950, its coverage changed
substantially in~1996 to cover a wider variety of storm
events~\parencite{stormevents}. We therefore only considered these metrics of
hurricane exposure for tropical cyclones in~1996 and later.

For each tropical cyclone, we identified all events with event types related to
flooding (``Flood,'' ``Flash Flood,'' ``Coastal Flood'') and tornadoes
(``Tornado'') with a start date within a five-day window centered on the date
of the tropical cyclone's closest approach to the
county~\parencite{hurricaneexposuredata}. To exclude events that started near
in time to the storm but far from the storm's track, and so were likely
unrelated to the storm, we exculed any events that occurred outside~500
\si{\kilo\metre} of the tropical cyclone's track. ``Flood,'' ``Flash Flood,''
and ``Tornado'' events in this database were reported by county \ac{FIPS} code
and so could be directly linked to counties.  ``Coastal Flood'' events were
reported by forecast zone; for these, the event was matched to the appropriate
county if possible using regular expression matching of listed county
names~\parencite{noaastormevents}. 

The tornado observations from this dataset form a traditional tornado event
database for the \ac{US}, and so we did not further validate
the tornado event data. It is difficult to characterize flooding at the county
level because flooding can be very localized and can be triggered by a variety
of causes. To investigate the extent to which the NOAA flood event listings
capture extremes that might be identified with other flooding data sources, we
investigated a sample of study counties, comparing the flood event data during
tropical storms with streamflow measurements at \ac{US} Geological Survey
county streamflow gages \parencite{usgsgages, countyfloods, dataRetrieval}.  

We considered nine study counties, selecting counties geographically spread
through storm-prone areas of the eastern \ac{US} and with multiple streamflow
gages reporting data during events. The sample counties were: Baltimore County,
MD; Bergen County, NJ; Escambia County, FL; Fairfield County, CT; Fulton
County, GA; Harris County, TX; Mobile County, AL; Montgomery County, PL; and
Wake County, NC. For each county, we first identified all streamflow gages in
the county with complete data for Jan.~1~1996\,--\,Dec.~31,~2015. If a storm
did not come within~500~\si{\kilo\metre} of a county, it was excluded from this
analysis, but all other study storms were considered. 

For each storm and county, we summed streamflow measurements across all county
gages to generate daily totals for the five-day window around the storm's
closest approach. We took the maximum of these daily streamflow totals as a
measure of the county's maximum daily streamflow during that storm. We also
calculated the percent of streamflow gages in the county with a daily
streamflow that exceeded a threshold of flooding (the streamgage's median value
for annual peak flow~\parencite{countyfloods}) on any day during the five-day
window. We investigated how these measurements varied between storms with
associated flood events in the NOAA Storm Events data versus storms without an
event listing for the county, to explore if storms with flood event listings
tended to be associated with higher streamflows at gages within the county.

\subsection*{Binary storm exposure classifications}

In our open-source data, we provide continuous measurements of three exposure
metrics: closest distance of each storm to each county, modeled storm winds,
and storm-related rainfall. However, epidemiologic studies of tropical cyclones
often use a binary exposure classification (``exposed'' versus ``unexposed'')
to assess storm-related health risks [refs], and so we also explored patterns
in storm exposure based on binary clasifications of these exposure metrics. 

Two of the exposure metrics (flood- and tornado-based) were inherently binary
in our data, since these metrics were based on whether an event was listed in
the NOAA Storm Events database.  For the other exposure metrics, each county
was classified as exposed to a tropical cyclone based on whether the exposure
metric exceeded a certain threshold (Table~\ref{tab:exposuremetrics}). We
picked reasonable thresholds, but others could be used with the open-source
data and its associated software. 

For the rainfall metric, a distance constraint was also necessary, to ensure
that rains unrelated and far from the storm track were not misattributed to a
storm. Through exploratory analysis, we set this distance metric at~500
\si{\kilo\metre} (i.e., for a county to be classified as exposed based on
rainfall, the cumulative rainfall in the county had to be over the
75~\si{\milli\metre} threshold and the storm must have passed within~500
\si{\kilo\metre} of the county; Table~\ref{tab:exposuremetrics}). This distance
constraint was typically large enough to capture storm-related rain.  However,
data users should note that in rare cases---for example, exceptionally large
storms (e.g., Hurricane Ike in 2008) or storms for which storm tracking was
stopped at extratropical transition (e.g., Tropical Storm Lee in 2011)---some
storm-related rains may be missed because of this distance constraint
(Figure~S1). This distance constraint can be customized using the software
published in association with the open-source
data~\parencite{hurricaneexposure}.

We characterized patterns in county-level exposure in the eastern \ac{US} for
each of the measured tropical cyclone hazards (extreme precipitation, maximum
sustained winds, flooding, and tornadoes).  Depending on available exposure
data, this assessment included some or all of the period from~1988 to~2015
(Table 1). For each binary metric of exposure to a tropical cyclone hazard, we
first summed the total number of county-level exposures over available years
for each exposure metric and mapped patterns in these exposures. We next
investigated agreement between storm exposure classifications based on these
different single-hazard exposures, as well as between each of these and a
distance-based proxy assessment of tropical cyclone exposure
(Table~\ref{tab:exposuremetrics}). We calculated the within-storm Jaccard
index~\parencite{jaccard1901distribution, jaccard1908nouvelles} between each
pair of exposure metrics. The Jaccard index~($J_s$) measures similarity between
two metrics ($X_{1,s}$ and~$X_{2,s}$) for tropical cyclone~$s$ as the
proportion of counties in which both of the metrics classify the county as
exposed out of all counties classified as exposed by at least one of the
metrics:

\begin{equation} 
J_s = \frac{X_{1,s} \cap X_{2,s}}{X_{1,s} \cup X_{2,s}}
\end{equation}

\noindent This metric can range from~0, in the case of no overlap between the
counties classified as exposed based on the two metrics, to~1, in the case that
the two metrics classify exactly the same set of counties as exposed to the
tropical cyclone. We measured these values for all study storms that affected a
large number of eastern \ac{US} counties (those storms with 250 or more of the
study counties classified as exposed by at least one of the exposure metrics)
during the years when all exposure data were available (1996\,--\,2011).

\subsection*{Case study}

The societal impact of a disaster depends both on its geophysical forces and on
the vulnerability of those living in the geographical areas it
affects~\parencite{chakraborty2005population, anderson2003community,
cutter1996vulnerability}. An estimate of ``physical exposure''---which
...---can help to jointly capture these facets of disaster impacts [ref, maybe
the peduzzi one given below?]. As a case study, we measured physical exposure
based on each metric of exposure and investigated whether conclusions from
these assessments varied substantially.

We collected data on the number of electricity-dependent Medicare beneficiaries
in each study county from the \ac{US} Department of Health \& Human Service's
emPower Map~2.0~\parencite{empower}. This subpopulation was selected since it
may be particularly susceptible to health impacts from tropical cyclone
exposure, especially through the pathway of storm-associated power outages and
evacuations. We calculated the physical exposure of the electricity-dependent
Medicare population in each county, based on tropical cyclone assessments using
each exposure metric (Table~\ref{tab:exposuremetrics}), following
\citeauthor*{peduzzi2009assessing}~(\citeyear{peduzzi2009assessing}):

\begin{equation}
E_c = F_c * P_c
\end{equation}

\noindent where~$E_c$ is the average yearly physical exposure among
electricity-dependent Medicare beneficiaries in county~$c$ to tropical cyclone
exposures based on a given metric,~$F_c$ is the estimated yearly expected
frequency of tropical cyclone exposures in county~$c$ based on that metric,
and~$P_c$ is the electricity-dependent Medicare population in
county~$c$, as of July~2017. 



