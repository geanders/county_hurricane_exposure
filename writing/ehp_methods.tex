\section*{Methods and Materials}

All counties in the eastern half of the \ac{US} (e.g., Figure~\ref{fig:ivanexposure}), 
as of the 2010 \ac{US} Decennial Census, were included in
the study. The study covered all tracked storms in the \ac{HURDAT2} \citep{landsea2013} 
that came within 250
kilometers of at least one study county between 1988 and 2015 (Figure~S2).
HURDAT2 is a post-storm assessment conducted by the \ac{US} National Hurricane
Center (NHC) and incorporates data from a variety of sources, including
satellite data and, when available, aircraft reconnaissance 
data~\citep{landsea2013, jarvinen1988}. These data typically give measurements of
the tropical cyclone center's location at 6-hour intervals at synoptic times
(i.e., 6:00 am, 12:00 pm, 6:00 pm, and 12:00 am UTC); some landfalling tropical
cyclones have an additional observation at the time of landfall~\citep{landsea2013}.

\subsection*{Distance-based exposure metric}

For each storm, the distance of closest approach was measured for every study county. 
Storm center track location data from \ac{HURDAT2} was 
interpolated using natural cubic splines to 15-minute intervals~\cite{hurricaneexposure}. 
The distance between each county's population mean center, as assessed by the 2010 Decennial
Census~\cite{countycenters}, and each 15-minute interpolated storm center location was
measured using Great Circle (WGS84 ellipsoid) distance~\cite{bivand2013applied}. The 
distance-based exposure metric was based on the minimum value of this storm 
center-to-county distance throughout the period the storm was tracked. The time of 
closest approach for each county-storm pairing was also recorded for use assessing 
rain-, flood-, and tornado-based metrics. To allow matching with data based on local 
time, times were converted from Coordinated Universal Time (UTC) to local 
time~\cite{countytimezones}.

\subsection*{Rain-based exposure metric}

For the rain-based exposure metric, cumulative rainfall was estimated for the
period from two days before to one day after the date of a tropical cyclone's
closest approach to a county. Daily county-level rainfall was determined by
aggregating hourly, 1/8 degree gridded reanalysis data from the North American
Land Data Assimilation System Phase 2 (NLDAS-2) precipitation data 
files~\citep{rui2013nldas}. The NLDAS-2 data integrate satellite-based and land-based
monitoring, applying a land-surface model to create a reanalysis dataset that
is spatially and temporally complete across the continental 
\ac{US}~\citep{rui2013nldas, alhamdan2014environmental}. To aggregate to county-level,
hourly data at each grid point were summed to create a daily rainfall total,
and these grid point rainfall totals were then averaged across all grid points
within a county's 1990 \ac{US} Census boundaries to create a county-level daily
total~\citep{alhamdan2014environmental, cdcwonder}. This county-level
precipitation data are publicly available through the \ac{US} Centers for Disease
Control's Wide-ranging Online Data for Epidemiological Research (WONDER)
database~\citep{cdcwonder, alhamdan2014environmental}. The rainfall-based
exposure metric was only available for tropical cyclones through 2011, the
period for which the county-aggregated reanalysis rainfall data were 
available~\citep{cdcwonder, alhamdan2014environmental}. Validation analysis of this and
other hazard-based exposure data is given in the Supplemental Material.

\subsection*{Wind-based exposure metric}

Ground-based observations of wind speed are problematic during tropical
cyclones, as instruments often fail at high wind speed, while reanalysis data,
often available at hourly or higher resolution, can be too smooth to capture
wind extremes associated with a tropical cyclone. Therefore, to create a
dataset of county-level sustained winds during historical tropical cyclones, we
modeled maximum sustained wind speeds at each county's population mean 
center~\citep{countycenters}. HURDAT2 Best Track data were interpolated from 6-hourly
reported values to 15-minute increments using natural cubic 
splines~\citep{stormwindmodel}, and then sustained wind speed was modeled for each
county center at each 15-minute increment using a double exponential wind speed
model~\citep{willoughby2006parametric} to estimate maximum ground-level
sustained wind speed at each county center~\citep{stormwindmodel}. Asymmetry in
wind speeds around the tropical cyclone center was incorporated into the wind
speed model~\citep{phadke2003modeling, stormwindmodel}. The maximum value of
this modeled wind speed was determined for each county as the maximum windspeed
across the 15-minute incremented modeled values throughout the tropical
cyclone.

\subsection*{Flood- and tornado-based exposure metrics}

To identify flood- and tornado-based exposures to tropical cyclones in \ac{US}
counties, we used event listings from the National Oceanic and Atmospheric
Administration (NOAA)'s Storm Event Database~\citep{stormevents}. For each
tropical cyclone, we identified all events with event types related to flooding
(``Flood", ``Flash Flood", ``Coastal Flood") and tornadoes (``Tornado") and
that occurred in a county within 500 kilometers of the tropical cyclone's track
and within a five-day window centered on the date of the tropical cyclone's
closest approach to the county~\citep{hurricaneexposuredata}. ``Flood", ``Flash
Flood" and ``Tornado" events in this database were reported by county Federal
Information Processing Standard (FIPS) code. ``Coastal Flood" events were
reported by forecast zone; for these, the event was matched to the appropriate
county if possible~\citep{noaastormevents}. While this database has recorded
storm data since 1950, its coverage changed substantially in 1996 to cover a
wider variety of storm events~\citep{stormevents}. We therefore only considered
these metrics of hurricane exposure for tropical cyclones in 1996 and later.

\subsection*{Assessment of tropical cyclone exposure data}

To assess the rainfall estimates from the reanalysis data used in analysis in the the 
main text, we compared cumulative rainfall estimates based on this reanalysis rainfall 
data to rainfall observations from ground-based weather stations in nine sample study 
counties, selected as counties with data available from multiple stations. In these nine
counties, observed rainfall was collected for all storms, 1988\,--\,2011, for the same period 
of two days before to one day after each storm's closest approach, using all stations with
available daily data in the Global Historical Climatology Network throughout 
1988\,--\,2011~\citep{menne2012overview, rnoaa, countyweather}. Agreement between 
measured storm-related rainfall between these two data sources was assessed by 
Spearman's rank correlation. 

To validate the metric of county-level wind speed used in the main text, we compared this 
wind metric to estimated wind radii from the revised Atlantic hurricane database (HURDAT2),
available for storms in our study from 2004~\citep{landsea2013}. HURDAT2 gives estimates of
the maximum radii to winds of 64, 50, and 34 knots in each four quadrants of a storm (i.e.,
distance from the storm's center to the furthest point in a quadrant with winds at or above
that speed) at each 6-hour storm observation. 

As validation for the flood data, we compared streamflow data from United States (\ac{US})
Geological Survey gages~\citep{usgsgages, countyfloods, dataRetrieval} in nine counties with
tropical cyclone flood classifications based on the NOAA Storm Events Database 
(Figure~\ref{fig:floodcomparison}).

\subsection*{Assessing agreement in exposure classification between
metrics}

Two of the exposure metrics (flood- and tornado-based) were inherently binary,
since these metrics were based on whether an event was listed in the NOAA Storm
Events database or not.  For the other exposure metrics, each county was
classified as exposed to a tropical cyclone based on whether the exposure
metric exceeded a certain threshold (Table~\ref{tab:exposuremetrics}). We
assessed county-level tropical cyclone exposure in the eastern \ac{US} based on
each exposure metric for all land-falling or near-land Atlantic basin tropical
cyclones. Depending on available exposure data, this assessment included some
or all of the period from 1988 to 2015, a period with 136 tropical cyclones
that made \ac{US} landfall or passed within 250 kilometers (155 miles) of at least
one \ac{US} county (Figure~S2).

We measured agreement between exposure metrics in the classification of a
county as exposed or unexposed to a specific tropical cyclone. For this
measurement, we calculated the within-storm Jaccard 
index~\citep{jaccard1901distribution, jaccard1908nouvelles} between every pair-wise
combination of hazard metrics.  The Jaccard index ($J_s$) measures similarity
between two metrics ($X_{1,s}$ and $X_{2,s}$) for tropical cyclone $s$ as the
proportion of counties in which both of the metrics classify the county as
exposed out of all counties classified as exposed by at least one of the
metrics:

\begin{equation} 
J_s = \frac{X_{1,s} \cap X_{2,s}}{X_{1,s} \cup X_{2,s}}
\end{equation}

\noindent This metric can range from 0, in the case of no overlap between the counties
classified as exposed based on the two metrics, to 1, in the case that the
two metrics classify exactly the same counties as exposed to the tropical cyclone.
