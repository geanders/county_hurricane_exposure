To assist with future tropical cyclone studies, we created and published open
source software with this dataset and accompanying
tools~\parencite{hurricaneexposuredata, hurricaneexposure}.  Many previous
studies have used geographical information system software (e.g., ArcGIS) to
assess exposure to tropical cyclones in the \ac{US}~\parencite{grabich2016,
zandbergen2009, czajkowski2011, kruk2010}.  Here, we offer methods to map and
output historic exposure to tropical cyclones that does not require the use of
proprietary software but instead uses a package written in the R statistical
programming language~\parencite{R}, which is free and open-source.  Further, by
including measurements of different hazard exposures in each county for each
tropical cyclone, this software allows for the development of more complex
exposure indices or models (e.g., random forests, multivariable generalized
linear models) that incorporate multiple tropical cyclone hazard measurements.  

[For discussion?] Based on these results, the two data sources were generally
in agreement, but the analysis highlights the difficulty of assessing
storm-related flooding at a multi-county, multi-year scale, as flooding can be
very localized within a county or can \ldots .   

However, these frequency maps, together with evidence from
specific tropical cyclones (Figures~\ref{fig:ivanexposure}
and~\ref{fig:jaccard}),  do illustrate the potential for strong differences in
spatial patterns in tropical cyclone exposures, depending on which tropical
cyclone hazards are considered.  A few previous studies have sought to
determine county-level exposure to tropical cyclones over multi-year periods,
including~\textcite{zandbergen2009}, which estimated exposure in \ac{US}
counties to all \ac{US} landfalling Atlantic-basin tropical cyclones
between~1851 and~2003, using both a distance-based metric and a metric that
combined distance and windspeed, and~\textcite{kruk2010}, which explored
exposure to hurricane-related winds in the \ac{US}, including inland areas,
for~1900\,--\,2008.  Our results suggest that such exposure assessments may
perform well in capturing some tropical cyclone hazards (e.g., wind), but
likely miss other potentially dangerous tropical cyclone exposures, especially
for hazards that repeatedly threaten northern or inland counties (e.g., rain,
flooding).

For example, distance from a tropical cyclone's track is relatively easy to
measure and has been used as an operational metric of exposure to tropical
cyclones in previous large-scale studies (examples include ...). Since distance
itself does not constitute a hazard, distance is meant in these cases as a
surrogate to capture exposure to hazards from the tropical cyclone. However,
here we found that in assessing \ac{US} county-level exposure to tropical
cyclones, distance is, at best, a moderate, and often a very poor, surrogate
for exposure to the specific tropical cyclone hazards of high wind, extreme
rainfall, flooding, and tornadoes (Figure~\ref{fig:jaccard}). Therefore, use of
distance to assess tropical cyclone exposure for impact studies could result in
problematic exposure misclassification, which could mask true associations,
even strong associations, between tropical cyclone exposure and outcomes of
interest in impact studies~\parencite{savitz2016interpreting,
armstrong1998effect}.  

The use of a single hazard-based metric (e.g., wind) could cause similar
problems if the impact is driven, at least in part, by a different hazard or by
multiple hazards of the tropical cyclone.

In creating this hurricane exposure dataset, we aimed for data that are
available for all eastern \ac{US} counties.

($<$90\% of counties classified into the same wind categories by the two
sources of wind estimates)

The wind radii, unlike the modeled wind speeds, are based on a storm-specific
re-analysis of available data. They therefore have the advantage that they can
capture unusual patterns in specific storms; their main disadvantage is that
they provide categorical, rather than continuous, estimates of wind speed, and
in many cases researchers might prefer more precise estimates of storm wind
speeds. However, since the wind radii data may be useful in some cases as
either a primary metric or for sensitivity analysis, we have included
county-level wind estimates for each storm based on these wind radii, in
addition to the continuous modeled windspeeds, in the open-source data
\parencite{hurricaneexposuredata}.

between county-level classifications of tropical cyclone exposure was typically
low to moderate across the  five exposure metrics we considered 

If measurements of one storm hazard are used as a proxy to capture exposure to
a different hazard, or a non-hazard proxy like distance to the storm track is
used to assess exposure to storm hazards, the resulting exposure
misclassification could be differential (i.e., associated with the outcome of
interest or with factors associated with risk of the outcome of interest). 


Therefore for most tropical cyclones, since exposure assessments often differ
across storm hazards, as well as between each storm hazard and a distance-based
proxy measurement, exposure misclassification is a potential risk.

\subsection*{Case study}

In the case study, we found differences in estimates of expected rates of
physical exposure to tropical cyclones among a population when exposure
assessment was based on different storm exposure metrics
(Figure~\ref{fig:topelecdependexposure}).  These results are informative
because previous assessments of physical exposure to tropical cyclones have
often concentrated on storm winds and storm proximity. For example, a global
study of exposure and vulnerability to natural hazards used a combination of
windspeed above a certain threshold and distance from the storm's track to
assess exposure~\parencite{peduzzi2009assessing}. \textbf{[Also check other
Peduzzi study]}  While many storm impacts might be most strongly linked to wind
hazards, there is a growing recognition of the potential risks of adverse
health outcomes and property damage from rain- and flood-related tropical
cyclone hazards in the \ac{US}~\parencite{smith2009}.  Assessments based solely
on wind and distance from the storm track therefore may underemphasize
potentially dangerous exposures to certain regions of the \ac{US}, particularly
inland locations.  In fact, based on our assessment of physical exposures to
tropical cyclones among electricity-dependent Medicare beneficiaries, the
highest expected physical exposure under any metric was measured for
Philadelphia County, PA, under the flood metric
(Figure~\ref{fig:topelecdependexposure}).  While this county has a
similar-sized population of electricity-dependent Medicare beneficiaries as
Miami-Dade County, FL (Figure~\ref{fig:topelecdependexposure}), the inland
county of Philadelphia has frequent storm-related flood events, which results
in a very high expected physical exposure among electricity-dependent Medicare
beneficiaries in the county. This exposure is poorly captured by most of the
other exposure metrics, under which Philadelphia County is not included in the
top ten counties for expected rates of physical
exposure~(Figure~\ref{fig:topelecdependexposure}).


\textit{Previous research has highlighted the range of impacts that tropical cyclones
can have in \ac{US} communities. However, studies of tropical cyclone exposures
and their impacts have varied widely in how they assess exposure, and there was
not previously a widely available public dataset that captured county-level
exposure to multiple storm hazards. Here we provide open-source tropical
cyclone hazard-related measurements for a large collection of historical
Atlantic-basin cyclones, aggregated at the county level to align with the
spatial scale at which much health-related data is collected.}

\textit{Further, we investigated agreement in exposure assessment across these metrics
when applied individually. We found large differences in which counties are
exposed to different hazards of tropical cyclones and that distance is, at
best, a moderate, and often a very poor, surrogate for exposure to the specific
tropical cyclone hazards of high wind, extreme rainfall, flooding, and
tornadoes. Use of distance as a surrogate for any of these hazards, or exposure
assessment based on one hazard when the pathway for health impacts is in part
or full through another storm hazard, could lead to exposure misclassification.
In the case of tropical cyclone risk and impact studies, including
epidemiological studies, this would result in biased estimates.  Our findings
highlight the importance of clarifying the potential pathway from tropical
cyclone hazards to health impacts when conducting tropical cyclone
epidemiological studies, and then basing exposure assessment on these hazards.}

