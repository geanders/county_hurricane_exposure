\section*{Discussion}

Epidemiological studies can help characterize which health risks are elevated
during disasters, to what degree, and to
whom~\parencite{ibrahim2005unfortunate, noji2005disasters}.  As a result, these
studies help improve disaster preparedness and
response~\parencite{noji2005disasters}.  However, tropical cyclones are
multi-hazard events, making it complicated to measure exposure and so to
conduct multi-community, multi-year studies leveraging large administrative
datasets.  Here, we provide open-source county-level data on multiple tropical
cyclone exposure metrics and explore limitations in that data.  
Further, we explore patterns in storm exposure classifications based on
different metrics, and we find that county-level tropical cyclone exposure
assessments vary substantially when using different metrics, based either on
storm hazards or on how close the storm's central track came to the community.
Our results can inform exposure assessment for future county-level studies of
the health risks and impacts associated with tropical cyclones exposures as
well as insights to help in designing studies.  

\subsection*{Exposure assessment data and software}

We investigated how well the open-source data correspond with data from other
potential sources
(Figures~\ref{fig:raincomparison}\,--\,\ref{fig:floodcomparison} and~S2). While
generally in agreement, there are some caveats. The rainfall data are generally
well-correlated with ground-based observations, but may sometimes oversmooth
very high rainfall values compared to ground-based observations
(Figure~\ref{fig:raincomparison}). When rainfall data is used to create binary
exposure classifications, this disagreement is unlikely to influence results,
as both data sources agree in identifying these as storms with high rainfall,
but would be important to consider for cases that include rainfall as a
continuous measurement. 

The wind data are based on modeled, rather than observed, values, and while the
modeled wind data generally agree well with post-analysis maximum wind radii
from \ac{HURDAT2}~\parencite{landsea2013} (Figure~\ref{fig:windcomparison}),
there were a few storms with some discrepancies. These storms---for example, Hurricane Sandy
in 2012 and Hurricane Ike in 2008---were
unusually large systems for which high winds persisted well inland from
landfall (Figures~\ref{fig:windcomparison}
and~S2). 

For the flooding data, we found that flood event status determined based on the
NOAA Storm Events listings typically agreed with measurements from \ac{USGS}
streamgages, with a flood event more likely to be listed if a storm elevated
streamflow across the county (Figure~\ref{fig:floodcomparison}).  However,
there are differences between the two flooding datasets, and these highlight
both the difficulty of measuring flood exposure at the county level and
inherent challenges in using data from a storm event database for epidemiologic
exposure assessment.  For example, there was one storm in Fulton County, GA,
for which there was high streamflow but not an associated flood event listing
(Hurricane Ida, 2009).  This storm occurred in November 2009, following a month
with historic rainfall and flooding in
Georgia~\parencite{shepherd2011overview}.  In this case, the flooding
associated with Ida was incorporated into an ongoing flood event listing, with
a start date well before the five-day window we used to temporally match storm
event listings with tropical cyclone tracks. 

This disagreement highlights the difficulty of large-scale pairing of storm
tracks with storm event listings---without criteria for temporally matching
event start dates to storm dates, many false positives would be
captured, for which the occurrance of a storm in the midst of an ongoing storm
event might be improperly attributed to the storm.  However, distance
and time restrictions, like those we used in matching NOAA Storm Event listings
with tropical storm locations and dates, do create the opportunity for
occasional false negatives, as for Hurricane Ida in Fulton County, GA, where a
storm contributes meaningfully to an ongoing event, but the event is not
captured for the storm in the exposure data.  

There are further limitations for the flood and tornado data---these data came
from the \ac{NOAA} Storm Events database, which, while a widely-used database
of events maintained by \ac{NOAA}, is based on reports, and so may be prone to
underreporting~\parencite{Ashley2008flood, Curran2000}, especially in
sparsely-populated areas~\parencite{Witt1998, Ashley2007}, as well as to other
reporting errors. 

While these are important caveats for the data, we selected these data sources
as among the best currently available for measuring each of these hazards
consistently and comprehensively at a multi-county, multi-year scale. In
addition to providing tropical cyclone exposure metrics for individual hazards,
this dataset and its associated software allow users not only to access
measurements for single hazards, but also to create tropical cyclone exposure
profiles based on multiple hazards or to craft exposure indices that combine
hazard metrics~\parencite{chakraborty2005population, peduzzi2009assessing}.
This ability can be critical, as different hazards of tropical cyclones can
act synergistically in causing impacts~\parencite{smith2009}.  


\subsection*{Patterns in tropical cyclone exposures}

We found average exposures to different tropical cyclone hazards differed
geographically (Figure~\ref{fig:averageexposure}). These patterns were not
unexpected based on what is known about tropical cyclone hazards, but still
highlight variations that are critical to consider in designing studies and
statistical analysis for tropical cyclone epidemiology. Further, they
demonstrate the need for multi-hazard exposure datasets for tropical cyclone
epidemiology, especially in capturing inland risks. 

Tropical cyclone wind exposures had a strong coastal pattern, which is
consistent with the dramatic decrease in wind intensity that typically
characterizes the landfall of tropical cyclones.  Tropical cyclone rain
exposures tended to extended further inland compared to wind exposures, up to
the Appalachian mountains. This agrees with previous research indicating that
the Appalachian mountains' topography both enhances precipitation during
tropical cyclones and provides hydrological conditions for severe
flooding~\parencite{rees2001}.  Almost all tropical cyclone tornado exposures
were in southern coastal states, consistent with previous evidence that most
noteworthy tropical cyclone-related tornadoes occur on the right-side of the
tropical cyclone track in Atlantic-basin \ac{US} storms~\parencite{moore2012}.
The exposure averages we calculated may be limited as estimates of
long-term frequencies, as tropical cyclones follow decadal
patterns~\parencite{kossin2007more} likely not adequately captured in the
available data, which covered less than 20 years. 

\subsection*{Agreement between exposure metrics}

We found that tropical cyclones tended to bring different hazards to different
counties and also that agreement was typically low between distance-based
tropical cyclone exposure assessment and each of the single hazard-specific
exposures, as well as between pairs of hazard-specific metrics
(Figures~\ref{fig:ivanexposure}\,--\,\ref{fig:jaccard} and S4\,--\,S5).  These
findings align with previous results from atmospheric science and related
fields on the characteristics of tropical cyclones. While tropical cyclone
rainfall and windspeed can be well-correlated when the tropical cyclone is over
water~\parencite{cerveny2000}, this relationship often weakens once the
hurricane has made landfall~\parencite{jiang2008}.  Fast-moving tropical
cyclones bring higher risks of dangerous winds inland~\parencite{kruk2010},
while slow-moving tropical cyclones are likely to bring more
rain~\parencite{rappaport2000} and cause more damage because of sustained
hazardous conditions~\parencite{rezapour2014}. Further, while the likelihood
and extent of flooding during a tropical cyclone is related to the tropical
cyclone's rainfall, it is also driven by factors like top soil saturation and
the structure of the water basin's drainage network~\parencite{chen2015,
rees2001}. 

Therefore, the use of a distance-based metric to assess exposure to any of
these hazards, or the use of measurements from one hazard as a proxy for
exposure to any of the other hazards considered, would often introduce exposure
misclassification. This conclusion reinforces similar findings from a study of
Florida's 2004 storm season~\parencite{grabich2015measuring}.  For some
studies, such exposure misclassification might plausibly be differential.  For
example, tropical cyclone wind exposures tend to be concentrated in counties
near the coast, while tropical cyclone wind exposures sometimes extend well
inland.  If the etiologically-relevant exposure for a health outcome is extreme
rainfall but exposure is classified based on measurements of wind, the
probability of being misclassified as unexposed would be higher in inland
counties, while the probability of being misclassified as exposed would be
higher in coastal counties. If coastal counties differ from inland counties in
either the outcome of interest or in factors associated with risk of that
outcome, exposure misclassification would be
differential~\parencite{savitz2016interpreting}.  Such differential exposure
misclassification could bias estimates of tropical cyclone effects either
towards the null (estimating a lower or null association compared to the true
association that exists) or away from the null (estimating a larger association
than actually exists)~\parencite{savitz2016interpreting, armstrong1998effect}.  

We did, however, find a small set of tropical cyclones for which for
which agreement was high across several single-hazard exposure assessments
(e.g., Floyd in~1999, Irene in 2011, Hannah in~2008, Bertha in~1996; Ernesto
in~2006 (Figure~\ref{fig:jaccard})).  Hurricanes Floyd in~1999 and
Irene in~2011 both made their first \ac{US} landfall in North Carolina at minor
hurricane windspeeds (Category~2 and~1, respectively) and then skimmed the
eastern coast of the \ac{US} north through New England, bringing substantial
rainfall to much of the eastern coast from North Carolina north and causing
extensive inland flooding in North Carolina (Floyd) and New England
(Irene)~\parencite{avila2013atlantic, lawrence2000atlantic}.  Hurricanes Hannah
in~2008, Bertha in~1996, and Ernesto in~2006 also followed the eastern
coastline. For these storms, the tropical cyclones' persistent proximity
to water may have helped maintain wind speeds in similar patterns to rain and
distance-based exposure assessments, resulting in more similarities across
exposure assessments compared to the other tropical cyclones we considered.
For these storms, it may be possible to assess exposure to multiple hazards of
the storm using a single metric, in some cases even a proxy like the distance
between a county and the storm's track.  However, for these storms it may be
difficult to untangle the contribution of each hazard to the overall
effect of the storm, given that several hazards have similar geographical
patterns. 

\subsection*{Limitations}

The dataset presented here does have several limitations, in addition to the
caveats previously discussed. First, the dataset is not comprehensive of all
possible tropical cyclone hazards. For example, coastal counties can experience
dangerous storm surge, which is not specifically covered in this dataset
(although some resulting coastal flooding is captured). We are exploring ways to
include this in future versions of the dataset, although to date we have
focused on exposures that could affect any county, whether inland or coastal.
Second, this data is aggregated to the county level. Many health outcome
datasets are aggregated at this level, but some may be aggregated at a finer
spatial resolution (e.g., Census tract or ZIP code) or unaggregated (i.e., point locations
for each outcome). We have published the wind
model we used to create this dataset as its own open-source R package
\parencite{stormwindmodel}, and it can be used to model
storm-associated winds at a finer spatial resolution; however, measurements of
other hazards cannot similarly be re-scaled through tools we provide. Next,
we provide these data and associated software tools through R packages, and so
some experience in the R programming language is required to make full use of
them. However, R is currently a popular programming language for environmental
epidemiology, allowing the data to reach a large audience, and we are exploring
options to create a web application using the Shiny platform to allow more
user-friendly web-based access of the data \parencite{shiny2019}. 
Finally, to assess patterns and agreement for binary exposure classifications, 
we have chosen one set of sensible thresholds for binary classifications based
on continuous metrics (rainfall, maximum sustained winds, and distance from 
the storm's track). Results and conclusions would differ somewhat with other
threshold choices. We have published all our code for this analysis online
[web address], in case other researchers would like to explore other 
threshold choices for these analyses.

\subsection*{Conclusions}

To conduct tropical cyclone epidemiological studies that span multiple
communities and storms, it is critical to have consistent and comprehensive
measurements of exposure to storm hazards. Here we have created and shared a
dataset that provides these data for counties in the United States over
multiple years. Despite some limitations in these data, they provide a powerful
tool for expanding tropical cyclone epidemiology studies to more extensively
leverage existing administrative health data, allowing researchers to
investigate how these storms affect county-wide health risk, including for
outcomes also common outside of storm periods.  Further, this dataset provides
hazard measurements that are comparable across communities and storms, which
allows epidemiological researchers to design studies to explore how health
risks are modified by characteristics of both the storms and the communities
that are hit. The data are given in an open-source format, along with
associated software tools, which allows them to be freely used and for others
to explore all associated code and to contribute additions through platforms
like GitHub.

Based on our analysis in this paper, these data are typically in agreement with
measurements from other sources of data available to characterize
storm-associated hazards (e.g., ground-based monitors, streamgages, post-storm
wind radii estimates).  However, researchers who are planning to use the data
should explore the analyses presented in this paper to understand the strengths
and weaknesses of the data.  Our results suggest that county-level storm
exposure is not well-characterized by the closest distance that a storm's
central track came to a county, and that exposure to one storm hazard within a
county (e.g., severe winds) does not imply exposure to other hazards (e.g.,
excessive rainfall, flooding). As a result, it is critical that researchers
consider which storm hazards are likely on the causal pathway for the outcomes
they are studying, and to characterize storm exposure in a way that captures
those specific hazards. 
