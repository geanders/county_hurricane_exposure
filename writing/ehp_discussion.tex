\section*{Discussion}

Epidemiological studies can help characterize which health risks are elevated
during disasters, to what degree, and to whom [Ibrahim editorial; Noji paper].
As a result, these studies help improve disaster preparedness and response
[Noji paper].  However, tropical cyclones are multi-hazard events, making it
complicated to measure exposure.  Here, we provide open-source county-level
data on multiple tropical cyclone exposure metrics and explore limitations in
that data.  

Further, we find that county-level tropical cyclone exposure assessments vary
substantially when using different metrics, based either on storm hazards or on
how close the storm's central track came to the community. Our results can
inform exposure assessment for future county-level studies of the health risks
and impacts associated with tropical cyclones exposures, and the open-source
data and software we have created provides multi-hazard, county-level exposure
data for such studies.

\subsection*{Exposure assessment data and software}

We investigated how well the open-source data correspond with
data from other available sources
(Figures~\ref{fig:raincomparison}\,--\,\ref{fig:floodcomparison} and~S2). While
generally in agreement, there are a few caveats that should be considered when
using these data. The rainfall data are generally
well-correlated with ground-based observations, but may sometimes
oversmooth very high rainfall values compared to ground-based observations
(Figure~\ref{fig:raincomparison}). When rainfall data is used to create binary
exposure classifications, this disagreement is unlikely to influence results,
as both data sources agree in identifying these as storms with high rainfall,
but would be important to consider for cases that include rainfall as a
continuous measurement. 

The wind data are based on modeled, rather than observed, values, and while the
modeled wind data generally agree well with post-analysis maximum wind radii
from \ac{HURDAT2}~\parencite{landsea2013} (Figure~\ref{fig:windcomparison}),
there were a few storms with some discrepancies. These storms---for example, Hurricane Sandy
in 2012 and Hurricane Ike in 2008---were
unusually large systems for which high winds persisted well inland from
landfall (Figures~\ref{fig:windcomparison}
and~S2). 

For the flooding events data, we found that flood event status determined based
on the NOAA Storm Events listings often agreed with measurements from \ac{USGS}
streamgages, with a flood event more likely to be listed if a storm elevated
stream flows in the county (Figure~\ref{fig:floodcomparison}).  However, there
are differences between the two flooding datasets, and these highlight both the
difficulty of measuring flood exposure at the county level and inherent
challenges in using data from a storm event database for epidemiologic exposure
assessment. 

For example, there was one storm in Fulton County, GA, for which there was high
streamflow but not an associated flood event listing (Hurricane Ida, 2009).
This storm occurred in November 2009, following a month with historic rainfall
and flooding in Georgia \parencite{shepherd2011overview}.  Therefore, in this
case the flooding associated with Ida was incorporated into an ongoing flood
event listing, with a start date well before the five-day window used to
temporally match storm event listings with tropical cyclone tracks used here. 

This disagreement highlights the difficulty of large-scale pairing of storm
tracks with storm event listings---without criteria for temporally matching
event dates to storm tracks, it is likely that many false positives would be
captured, for which the occurrance of a storm in the midst of an ongoing storm
event listing would be improperly attributed to the storm.  However, distance
and time restrictions, like those we used in matching NOAA Storm Event listings
with tropical storm locations and dates, do create the opportunity for
occasional false negatives, as for Hurricane Ida in Fulton County, GA, where a
storm contributes meaningfully to an ongoing event, but the event is not
captured for the storm.  

There are further limitations for the flood and tornado data---these data came
from the \ac{NOAA} Storm Events database, which, while a widely-used database
of events maintained by \ac{NOAA}, is based on reports, and so may be prone to
underreporting~\parencite{Ashley2008flood, Curran2000}, especially in
sparsely-populated areas~\parencite{Witt1998, Ashley2007} [Simmons and Sutter
ref], as well as to other reporting errors. 

While these are important caveats for the data, we selected these data sources
as among the best currently available for measuring each of these hazards
consistently and comprehensively at a multi-county, multi-year scale. In
addition to providing tropical cyclone exposure metrics for individual hazards,
this dataset and its associated software allow users not only to access
measurements for single hazards, but also to create tropical cyclone exposure
profiles based on multiple hazards or craft exposure indices that combine
hazard metrics~\parencite{chakraborty2005population, peduzzi2009assessing}.
This ability can be critical, as different hazards of tropical cyclones often
act synergistically in causing impacts~\parencite{smith2009}.  


\subsection*{Patterns in tropical cyclone exposures}

We found differences in geographic patterns in average exposures to different
tropical cyclone hazards (Figure~\ref{fig:averageexposure}). These patterns
were not unexpected based on what is known about tropical cyclone hazards, but
still highlight variations that are critical to consider in designing studies
and statistical analysis for tropical cyclone epidemiology. Further, they
demonstrate the need for multi-hazard exposure datasets for tropical cyclone
epidemiology. 

Tropical cyclone wind exposures had a strong coastal pattern, which is
consistent with the dramatic decrease in wind intensity that typically
characterizes the landfall of tropical cyclones.  Tropical cyclone rain
exposures tended to extended further inland compared to wind exposures, up to
the Appalachian mountains. This agrees with previous research indicating that
the Appalachian mountains' topography both enhances precipitation during
tropical cyclones and provides hydrological conditions for severe
flooding~\parencite{rees2001}.  Almost all tropical cyclone tornado exposures
were in southern coastal states, consistent with previous evidence that most noteworthy
tropical cyclone-related tornadoes occur on the right-side of the tropical
cyclone track in Atlantic-basin \ac{US} storms~\parencite{moore2012}.  The
exposure averages we calculated are somewhat limited as estimates of long-term
frequencies, as tropical cyclones follow decadal
patterns~\parencite{kossin2007more} likely not adequately captured in the
available data, which covered less than 20 years. However, exposure assessments
spanning over a century have found similar patterns for distance- and
wind-based metrics~\textcite{zandbergen2009, kruk2010} [double-check that refs
found similar patterns].

\subsection*{Agreement between exposure metrics}

We found that agreement was typically low across hazard-specific tropical
exposure classifications, and agreement was also typically low between
distance-based tropical cyclone exposure assessment and each of the single
hazard-specific exposures (Figures~\ref{fig:ivanexposure}\,--\,
\ref{fig:jaccard} and S4\,--\,S5). This suggests that the use of a
distance-based metric to assess exposure to any of these hazards, or the use of
measurements from one hazard as a proxy for exposure to any of the other
hazards considered, would often be problematic, introducing exposure
misclassification.

These findings align with previous results from atmospheric science and related
fields on the characteristics of tropical cyclones. While tropical cyclone
rainfall and windspeed can be well-correlated when the tropical cyclone is
over water~\parencite{cerveny2000}, this relationship often weakens once the
hurricane has made landfall~\parencite{jiang2008}.  Fast-moving tropical
cyclones bring higher risks of dangerous winds inland~\parencite{kruk2010},
while slow-moving tropical cyclones are likely to bring more
rain~\parencite{rappaport2000} and cause more damage because of sustained
hazardous conditions~\parencite{rezapour2014}. Further, while the likelihood
and extent of flooding during a tropical cyclone is related to the tropical
cyclone's rainfall, it is also driven by factors like top soil saturation and
the structure of the water basin's drainage network~\parencite{chen2015,
rees2001}. 

However, we did find a small set of tropical cyclones for which for which
agreement was high across single-hazard exposure assessments (e.g., Floyd
in~1999, Irene in 2011, Hannah in~2008, Bertha in~1996;
Figure~\ref{fig:jaccard}).  Hurricanes Floyd in~1999 (Figure~S4) and Irene
in~2011 both made their first \ac{US} landfall in North Carolina at minor
hurricane windspeeds (Category~2 and~1, respectively) and then skimmed the
eastern coast of the \ac{US} north through New England, bringing substantial
rainfall to much of the eastern coast from North Carolina north and causing
extensive inland flooding in North Carolina (Floyd) and New England
(Irene)~\parencite{avila2013atlantic, lawrence2000atlantic}.  Hurricanes Hannah
in~2008 and Bertha in~1996 followed the eastern coastline north of North
Carolina in a pattern similar to Floyd and Irene. For this set of storms, the
tropical cyclones' persistent proximity to water may have helped maintain wind
speeds in similar patterns to rain and distance-based exposure assessments,
resulting in more similarities across exposure assessments compared to the
other tropical cyclones we considered. 

Since exposure assessments often differ across storm hazards, as well as
between each storm hazard and a distance-based proxy measurement, exposure
misclassification is a potential risk. For some studies, such exposure
misclassification might plausibly be differential (i.e., associated with the
outcome of interest or with factors associated with risk of the outcome of
interest).  For example, tropical cyclone wind exposures tend to be
concentrated in counties near the coast, while tropical cyclone wind exposures
sometimes extend well inland.  If the etiologically-relevant exposure for a
health outcome is extreme rainfall but exposure is classified based on
measurements of wind, the probability of being misclassified as unexposed would
be higher in inland counties, while the probability of being misclassified as
exposed would be higher in coastal counties. If coastal counties differ from
inland counties in either the outcome of interest or in factors associated with
risk of that outcome, exposure misclassification would be
differential~\parencite{savitz2016interpreting}.  Such differential exposure
misclassification could bias estimates of tropical cyclone effects either
towards the null (estimating a lower or null association compared to the true
association that exists) or away from the null (estimating a larger association
than actually exists)~\parencite{savitz2016interpreting, armstrong1998effect}.
[Variation across studies in how exposure is assessed?]

\subsection*{Case study}

The case study illustrates how differences in exposure assessment
across storm exposure metrics can result in differing conclusions. In this 
case, we found differences in estimates of expected
physical exposure to tropical cyclones among a population when exposure
assessment was based on different storm exposure metrics
(Figure~\ref{fig:topelecdependexposure}).  

These results are informative because previous assessments of physical exposure
to tropical cyclones have often concentrated on storm winds and storm
proximity. For example, a global study of exposure and vulnerability to natural
hazards used a combination of windspeed above a certain threshold and distance
from the storm's track to assess exposure~\parencite{peduzzi2009assessing}.
While many storm impacts might be most strongly linked to wind hazards, there
is a growing recognition of the potential risks of adverse health outcomes and
property damage from rain- and flood-related tropical cyclone hazards in the
\ac{US}~\parencite{smith2009}.  Assessments based solely on wind and distance
from the storm track therefore may underemphasize potentially dangerous
exposures to certain regions of the \ac{US}, particularly inland locations. 

In fact, based on our assessment of physical exposures to tropical cyclones
among electricity-dependent Medicare beneficiaries, the highest expected
physical exposure under any metric was measured for Philadelphia County, PA,
under the flood metric (Figure~\ref{fig:topelecdependexposure}).  While this
county has a similar-sized population of electricity-dependent Medicare
beneficiaries as Miami-Dade County, FL
(Figure~\ref{fig:topelecdependexposure}), the inland county of Philadelphia has
frequent storm-related flood events, which results in a very high expected
physical exposure among electricity-dependent Medicare beneficiaries in the
county. This exposure is poorly captured by most of the other exposure metrics,
under which Philadelphia County is not included in the top ten counties for
physical exposure~(Figure~\ref{fig:topelecdependexposure}).

\subsection*{Conclusions}

Previous research has highlighted the range of impacts that tropical cyclones
can have in \ac{US} communities. However, studies of tropical cyclone exposures
and their impacts have varied widely in how they assess exposure, and there was
not previously a widely available public dataset that captured county-level
exposure to multiple storm hazards. Here we provide open-source tropical
cyclone hazard-related measurements for a large collection of historical
Atlantic-basin cyclones, aggregated at the county level to align with the
spatial scale at which much health-related data is collected.

Further, we investigated agreement in exposure assessment across these metrics
when applied individually. We found large differences in which counties are
exposed to different hazards of tropical cyclones and that distance is, at
best, a moderate, and often a very poor, surrogate for exposure to the specific
tropical cyclone hazards of high wind, extreme rainfall, flooding, and
tornadoes. Use of distance as a surrogate for any of these hazards, or exposure
assessment based on one hazard when the pathway for health impacts is in part
or full through another storm hazard, could lead to exposure misclassification.
In the case of tropical cyclone risk and impact studies, including
epidemiological studies, this would result in biased estimates.  Our findings
highlight the importance of clarifying the potential pathway from tropical
cyclone hazards to health impacts when conducting tropical cyclone
epidemiological studies, and then basing exposure assessment on these hazards.

