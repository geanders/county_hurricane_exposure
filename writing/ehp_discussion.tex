\section*{Discussion}

Tropical cyclones impact public health in many ways, and epidemiological
studies of the health risk and impacts associated with tropical cyclone
exposures could help improve preparedness for and response to future tropical
cyclones.  However, tropical cyclones are multi-hazard events, making it
complicated to measure exposure. Here, we provide open-source county-level data
on multiple storm exposure metrics, and we find that county-level tropical
cyclone exposure assessments vary substantially when using a distance-based
metric versus four hazard-based metrics, as well as among different
hazard-based metrics. Our results can inform exposure assessment for future
county-level studies of the health risks and impacts associated with tropical
cyclones exposures, and the open-source data and software we have created
provides multi-hazard, county-level exposure data for such studies.

\subsection*{Exposure assessment data and software}

In creating this hurricane exposure dataset, we aimed for data that are
available for all eastern \ac{US} counties. In addition to providing tropical
cyclone exposure metrics for individual hazards, this data and its associated
software allow users not only to access measurements for single hazards, but
also to create tropical cyclone exposure profiles based on multiple hazards or
craft exposure indices that combine hazard
metrics~\parencite{chakraborty2005population, peduzzi2009assessing}. This
ability can be critical, as different hazards of tropical cyclones often act
synergistically in causing impacts~\parencite{smith2009}.  

\begin{comment}
To assist with future tropical cyclone studies, we created and published open
source software with this dataset and accompanying
tools~\parencite{hurricaneexposuredata, hurricaneexposure}.  Many previous
studies have used geographical information system software (e.g., ArcGIS) to
assess exposure to tropical cyclones in the \ac{US}~\parencite{grabich2016,
zandbergen2009, czajkowski2011, kruk2010}.  Here, we offer methods to map and
output historic exposure to tropical cyclones that does not require the use of
proprietary software but instead uses a package written in the R statistical
programming language~\parencite{R}, which is free and open-source.  Further, by
including measurements of different hazard exposures in each county for each
tropical cyclone, this software allows for the development of more complex
exposure indices or models (e.g., random forests, multivariable generalized
linear models) that incorporate multiple tropical cyclone hazard measurements.  
\end{comment}

We investigated how well these data correspond with data from other available
sources (Figures~\ref{fig:raincomparison}\,--\,\ref{fig:floodcomparison}
and~S2). While generally in agreement with data from other sources,
there are a few caveats that should be considered when applying these data in
future studies. The rainfall data are from re-analysis data are generally
well-correlated with observed ground-based station data, but may sometimes
oversmooth very high rainfall values compared to ground-based observations
(Figure~\ref{fig:raincomparison}). When rainfall data is used to create binary
exposure classifications, this disagreement is unlikely to influence results,
as both data sources agree in identifying these as storms with high rainfall,
but it may be important to consider for cases that include rainfall as a
continuous measurement. 

The wind data are based on modeled, rather than observed, values, and while the
modeled wind data generally agree well with post-analysis maximum wind radii
from \ac{HURDAT2}~\parencite{landsea2013} (Figure~\ref{fig:windcomparison}),
there were a few storms where the two datasets did not agree as well ($<$90\%
of counties classified into the same wind categories by the two sources of wind
estimates). These storms were exceptionally large systems for which high winds
persisted well inland from landfall, like Hurricanes Sandy and Ike
(Figures~\ref{fig:windcomparison} and~S2). Since the wind radii are based on
\ldots, they have [advantages]; their main disadvantages is that they provide
categorical, rather than continuous, estimates of wind speed, and in many cases
researchers might prefer more precise estimates of storm wind speeds. However,
since the wind radii data may be useful in some cases as either a primary
metric or for sensitivity analysis, we have include county-level wind estimates
for each storm based on these wind radii, in addition to the continuous modeled
windspeeds, in the open-source data \parencite{hurricaneexposuredata}.

For the flooding events data, we found it often corresponded with measurements
from \ac{US} Geographical Survey streamgages, with flood event listings more
likely during storms which elevated stream flows
(Figure~\ref{fig:floodcomparison}. However, there are differences between the
two flooding datasets, and these highlight both the difficulty of measuring
flood exposure at the county level and inherent challenges in using the storm
event database for epidemiologic exposure assessment.  The flood and tornado
data came from the \ac{NOAA} Storm Events database, which, while a widely-used
database of events maintained by \ac{NOAA}, is based on reports, and so may be
prone to underreporting~\parencite{Ashley2008flood, Curran2000}, especially in
less populated areas~\parencite{Witt1998, Ashley2007}, as well as to other
reporting errors. 

\begin{comment}
[For discussion?] Based on these results, the two data sources were generally
in agreement, but the analysis highlights the difficulty of assessing
storm-related flooding at a multi-county, multi-year scale, as flooding can be
very localized within a county or can \ldots .   
\end{comment}

While these are important caveats for the data, we selected these data sources
as among the best currently available for measuring each of these hazards at a
multi-county, multi-year scale.

\subsection*{Patterns in tropical cyclone exposures}

We found geographic patterns in tropical cyclones based on the different
exposure metrics (Figure~\ref{fig:averageexposures}). These patterns were not
unexpected based on what is known about tropical cyclone hazards, but still
highlight variations across metrics that are critical to consider in designing
studies for tropical cyclone epidemiology, and they demonstrate the need for
multi-hazard exposure datasets for tropical cyclone epidemiology. Wind-based
exposures had a strong coastal pattern, which is consistent with the dramatic
decrease in wind intensity that typically characterizes the landfall of
tropical cyclones. Exposures by rain- and distance-based metrics often extended
further inland compared to the wind-based metric, up to the Appalachian
mountains. This agrees with previous research indicating that the Appalachian
mountains' topography both enhances precipitation during tropical cyclones and
provides hydrological conditions for severe flooding~\parencite{rees2001}.
Almost all tornado-based exposures were in southern coastal states, echoing
previous findings that most noteworthy tropical cyclone-related tornadoes occur
on the right-side of the tropical cyclone track in Atlantic-basin \ac{US}
storms~\parencite{moore2012}.  These exposure averages are somewhat limited as
estimates of long-term frequencies, as tropical cyclones follow decadal
patterns~\parencite{kossin2007more} likely not adequately captured in the
available data. However, exposure assessments spanning over a century have
found similar patterns for distance- and wind-based
metrics~\textcite{zandbergen2009, kruk2010} [double-check that refs found
similar patterns].

\begin{comment}
However, these frequency maps, together with evidence from
specific tropical cyclones (Figures~\ref{fig:ivanexposure}
and~\ref{fig:jaccard}),  do illustrate the potential for strong differences in
spatial patterns in tropical cyclone exposures, depending on which tropical
cyclone hazards are considered.  A few previous studies have sought to
determine county-level exposure to tropical cyclones over multi-year periods,
including~\textcite{zandbergen2009}, which estimated exposure in \ac{US}
counties to all \ac{US} landfalling Atlantic-basin tropical cyclones
between~1851 and~2003, using both a distance-based metric and a metric that
combined distance and windspeed, and~\textcite{kruk2010}, which explored
exposure to hurricane-related winds in the \ac{US}, including inland areas,
for~1900\,--\,2008.  Our results suggest that such exposure assessments may
perform well in capturing some tropical cyclone hazards (e.g., wind), but
likely miss other potentially dangerous tropical cyclone exposures, especially
for hazards that repeatedly threaten northern or inland counties (e.g., rain,
flooding).
\end{comment}

\subsection*{Agreement between exposure metrics}

We found that there was typically only moderate or low agreement between
county-level exposure classifications for a tropical cyclone when comparing
different exposure metrics
(Figures~\ref{fig:ivanexposure}\,--\,\ref{fig:jaccard}  and S4\,--\,S5). This
suggests that the use of one of these metrics as a proxy for all storm-based
hazards in epidemiological research would often be problematic, introducing
exposure misclassification.

\begin{comment}
For example, distance from a tropical cyclone's track is relatively easy to
measure and has been used as an operational metric of exposure to tropical
cyclones in previous large-scale studies (examples include ...). Since distance
itself does not constitute a hazard, distance is meant in these cases as a
surrogate to capture exposure to hazards from the tropical cyclone. However,
here we found that in assessing \ac{US} county-level exposure to tropical
cyclones, distance is, at best, a moderate, and often a very poor, surrogate
for exposure to the specific tropical cyclone hazards of high wind, extreme
rainfall, flooding, and tornadoes (Figure~\ref{fig:jaccard}). Therefore, use of
distance to assess tropical cyclone exposure for impact studies could result in
problematic exposure misclassification, which could mask true associations,
even strong associations, between tropical cyclone exposure and outcomes of
interest in impact studies~\parencite{savitz2016interpreting,
armstrong1998effect}.  
\end{comment}

These findings align with previous results from atmospheric science and related
fields on the characteristics of tropical cyclones. While tropical cyclone
rainfall and windspeed are often well-correlated when the tropical cyclone is
over water~\parencite{cerveny2000}, this relationship often does not remain as
strong once the hurricane has made landfall~\parencite{jiang2008}.  Fast-moving
tropical cyclones bring higher risks of dangerous winds
inland~\parencite{kruk2010}, while slow-moving tropical cyclone are likely to
bring more rain~\parencite{rappaport2000} and cause more damage because of
sustained hazardous conditions~\parencite{rezapour2014}. Further, while the
likelihood and extent of flooding during a tropical cyclone is related to the
tropical cyclone's rainfall, it is also driven by factors like top soil
saturation and the structure of the water basin's drainage
network~\parencite{chen2015, rees2001}. 

However, we did find a small set of tropical cyclones for which for which
agreement between metrics were unusually high (e.g., Floyd in~1999, Irene in
2011, Hannah in~2008, Bertha in~1996; Figure~\ref{fig:jaccard}).  Hurricanes
Floyd in~1999 (Figure~S4) and Irene in~2011 both made their first \ac{US}
landfall in North Carolina at minor hurricane windspeeds (Category~2 and~1,
respectively) and then skimmed the eastern coast of the \ac{US} north through
New England, bringing large rainfall to much of the eastern coast from North
Carolina north and causing extensive inland flooding in North Carolina (Floyd)
and New England (Irene)~\parencite{avila2013atlantic, lawrence2000atlantic}.
Hurricanes Hannah in~2008 and Bertha in~1996 followed the eastern coastline
north of North Carolina in a pattern similar to Floyd and Irene. For this set
of storms, the tropical cyclones' persistent proximity to water may have helped
maintain wind speeds in similar patterns to rain and distance exposures,
resulting in this moderate to good agreement among exposure assessments based
on different metrics. 

In some cases, exposure misclassification from using one of these exposure
metrics to capture one or more of the other storm hazards may be differential
(i.e., associated with the outcome of interest or with factors associated with
risk of the outcome of interest). For example, tropical cyclone wind exposures
tend to be concentrated in counties near the coast, since most tropical
cyclones rapidly decrease in their sustained windspeed following landfall.
However, tropical cyclone exposures based on distance can extend well inland,
following the tropical cyclone's tracks, but may not adequately capture all
wind-exposed counties near the coast. In this case, if the
etiologically-relevant exposure is high wind but exposure is classified based
on distance, the probability of being misclassified as unexposed would be
higher in coastal counties, while the probability of being misclassified as
exposed would be higher in inland counties. If coastal counties differ from
inland counties in either the outcome of interest or in factors associated with
risk of that outcome, differential exposure misclassification would
exist~\parencite{savitz2016interpreting}. Such differential exposure
misclassification can bias estimates of tropical cyclone effects either towards
the null (estimating a lower or null association compared to the true
association that exists) or away from the null (estimating a larger association
than actually exists)~\parencite{savitz2016interpreting, armstrong1998effect}.

\begin{comment}
The use of a single hazard-based metric (e.g., wind) could cause similar
problems if the impact is driven, at least in part, by a different hazard or by
multiple hazards of the tropical cyclone.
\end{comment}

\subsection*{Case study}

The conducted case study demonstrates how differences in exposure assessment
across storm exposure metrics can result in differing estimates of expected
physical exposure to tropical cyclones among a population
(Figure~\ref{fig:topelecdependexposure}).  Previous assessments of physical
exposure to tropical cyclones have often focused on solely storm winds and
storm proximity. For example, a global study of exposure and vulnerability to
natural hazards used a combination of windspeed above a certain threshold and
distance from the storm's track to assess
exposure~\parencite{peduzzi2009assessing}. While many storm impacts might be
most strongly linked to wind hazards, there is a growing recognition of the
potential risks of adverse health outcomes and property damage from rain- and
flood-related tropical cyclone hazards in the \ac{US}~\parencite{smith2009}.
Such assessments may underemphasize potentially dangerous exposures to certain
regions of the \ac{US}, particularly inland locations. In fact, based on our
assessment of physical exposures to tropical cyclones among
electricity-dependent Medicare beneficiaries, the highest expected physical
exposure under any metric was measured for Philadelphia County, PA, under the
flood metric (Figure~\ref{fig:topelecdependexposure}).  While this county has a
similar-sized population of electricity-dependent Medicare beneficiaries as
Miami-Dade County, FL (Figure~\ref{fig:topelecdependexposure}), the frequency
of storm-related flood events in this inland county results in a very high
expected physical exposure among this subpopulation, while this county is not
included in the top ten counties for physical exposure based on most of the
other exposure metrics.

\subsection*{Conclusions}

Previous research has highlighted the range of impacts that tropical cyclones
can have in \ac{US} communities. However, studies of tropical cyclone exposures
and their impacts have varied widely in how they assess exposure. Here we found
large differences in which counties are exposed to different hazards of
tropical cyclones and that distance is, at best, a moderate, and often a very
poor, surrogate for exposure to the specific tropical cyclone hazards of high
wind, extreme rainfall, flooding, and tornadoes. Use of distance as a surrogate
for any of these hazards could lead to exposure misclassification and, in the
case of tropical cyclone risk and impact studies, including epidemiological
studies, result in biased estimates.  Similarly, if such studies disagree on
their findings, it could result from the often poor agreement between exposure
classifications based on these different metrics, making studies that use
different metrics hard to compare or combine in meta-analyses. Our findings
highlight the importance of clarifying the potential pathway from tropical
cyclone hazards to health impacts when conducting tropical cyclone
epidemiological studies, and then basing exposure assessment on these hazards.
To help, we provide open-source tropical cyclone hazard-related measurements
for a large collection of historical Atlantic-basin cyclones, aggregated at the
county level to align with the spatial scale at which much health-related data
is collected.
