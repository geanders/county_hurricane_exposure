\begin{abstract}

\noindent \textbf{Background:} Tropical cyclones bring severe impacts to
	\ac{US} communities, including substantial human health risks, from
	hazards that include wind, rain, flooding, and tornadoes. However, 
	it is difficult to conduct multi-year, multi-storm epidemologic 
	studies without storm hazard datasets that can easily be linked to 
	human health data. Further, while some storm exposure metrics are
	straightforward to measure (e.g., distance from a community to a 
	storm's track), they may poorly capture exposure to storm hazards. \\ 
\textbf{Objectives:} (1)~Provide an open-source dataset to improve tropical
	cyclone exposure assessment for health-related research and
	(2)~investigate agreement between county-level assessments of tropical cyclone
	exposure when based on different storm hazards, to clarify 
	risk of exposure misclassification in epidemiologic studies. \\ 
\textbf{Methods:} We created an open-source dataset with county-level data on
	exposure to tropical cyclones in the \ac{US}. The dataset includes
	measures of distance from the storm track, maximum sustained wind,
	rainfall, flooding, and tornadoes for all land-falling or near-land
	Atlantic basin storms over multiple years (1996\,--\,2011 for all
	metrics, up to~1988\,--\,2015 for specific metrics). [Validation] We measured
	agreement in county-level exposure assessment for all pairwise
	combinations of these metrics. \\ 
\textbf{Results:} We created an open-source dataset ... 
	[Most data agreed well. Flooding is hard to measure.]
	In most of the storms investigated, agreement in county-level
	exposure assessment was at best moderate, and often poor, when using
	distance compared with four hazard-based measurements. There was also
	substantial disagreement in exposure assessment based on all pairwise
	combinations of the four hazard-based measurements. There were
	geographic patterns in exposures based on these metrics, with most
	wind-based exposures near the coast, flood- and rain-based exposures
	extending to inland and northern counties, and tornado-based exposures
	in southern counties. \\ 
\textbf{Discussion:} Our results provide a multi-hazard dataset that can be
	leveraged for epidemiologic research on tropical cyclones.  Further,
	our results suggest that when epidemiologic studies use distance as a
	surrogate for tropical cyclone exposure, or use one hazard-based metric
	when the impact is partly or fully caused by a different hazard, the
	study will be prone to exposure misclassification.  
\end{abstract}
