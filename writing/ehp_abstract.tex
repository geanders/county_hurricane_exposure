\begin{abstract}

\noindent \textbf{Background:} Tropical cyclones bring severe impacts to
\ac{US} communities, including substantial human health risks, from hazards
that include wind, rain, flooding, and tornadoes. However, studies quantifying
tropical cyclone risks and impacts vary widely in how they classify exposure to
tropical cyclones, using various hazard-based metrics and, in some cases, using
distance from the storm track as a surrogate. \\ 
\textbf{Objectives:} (1)~Investigate how well county-level assessments of
tropical cyclone exposure agree when based on distance versus hazard-based
measurements, as well as among assessments based on different hazard
measurements, and (2)~provide software to improve tropical cyclone exposure
assessment for health-related research. \\ 
\textbf{Methods:} We measured county-level exposure to tropical cyclones in the
United States based on distance from the storm track, maximum sustained wind,
rainfall, flooding, and tornadoes for all land-falling or near-land Atlantic
basin storms, for~1996\,--\,2011 for all metrics and up to~1988\,--\,2015 for
specific metrics. We quantified how well county-level exposure assessment
agreed for all pairwise combinations of metrics.\\ 
\textbf{Results:} In most of the storms investigated, agreement in county-level
exposure assessment was at best moderate, and often poor, when using distance
compared with four hazard-based measurements. There was also substantial
disagreement in exposure assessment based on all pairwise combinations of the
four hazard-based measurements. There were geographic patterns in exposures
based on these metrics, with most wind-based exposures near the coast, flood-
and rain-based exposures extending to inland and northern counties, and
tornado-based exposures in southern counties. \\ 
\textbf{Discussion:} Our results suggest that when impact studies use distance
as a surrogate for tropical cyclone exposure, or use one hazard-based metric
when the impact is partly or fully caused by a different hazard, the study will
be prone to exposure misclassification.  To facilitate future research, we make
this multi-hazard storm exposure data available through open-source software. 
\end{abstract}
