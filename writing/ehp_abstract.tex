\begin{abstract}

\noindent \textbf{Background:} 
        Tropical cyclone epidemiology can be 
	advanced through county-level exposure assessment methods that are comprehensive and 
	consistent across space and time, facilitating multi-year, multi-storm studies. 
	Further, an understanding of 
	patterns in and between exposure metrics that are based on specific hazards of the storm can help in 
	designing tropical cyclone epidemiological studies.\\ 
\textbf{Objectives:} (1)~Provide an open-source dataset for tropical
	cyclone exposure assessment for epidemiological research and
	(2)~investigate patterns and agreement between county-level assessments of tropical cyclone
	exposure based on different storm hazards. \\ 
\textbf{Methods:} We created an open-source dataset with county-level data on
	exposure to four tropical cyclone hazards: peak sustained wind,
	rainfall, flooding, and tornadoes. The data cover all eastern \ac{US} counties 
	for all land-falling or near-land
	Atlantic basin storms, covering~1996\,--\,2011 for all
	metrics and up to~1988\,--\,2015 for specific metrics. 	
	We validated 
	measurements against other data sources and investigated patterns
	and agreement among binary exposure classifications based on these
	metrics, as well as compared them to use of distance from the storm's track as a proxy for
	exposure.\\ 
\textbf{Results:} We describe how we developed this dataset and explore how its
	measurements compare to other data that could be used to characterize
	tropical cyclone hazards for epidemiological research. Measurements in
	the dataset created here typically agreed with data from other sources,
	and we present and discuss areas of disagreement and other caveats for
	the data. Tropical cyclones typically brought different hazards to different
	counties, and so agreement was usually low between distance-based tropical
	cyclone exposure assessment and each hazard-specific metric, as well as 
	between pairs of hazard-specific metrics. \\
\textbf{Discussion:} Our results provide a multi-hazard dataset that can be
	leveraged for epidemiologic research on tropical cyclones, as well as 
	insights that can inform the design of tropical cyclone epidemiological 
	studies.
\end{abstract}
