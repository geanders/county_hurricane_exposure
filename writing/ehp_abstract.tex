\begin{abstract}

\noindent \textbf{Background:} Tropical cyclones bring severe impacts to
	\ac{US} communities, including substantial human health risks, from
	hazards that include wind, rain, flooding, and tornadoes. However, 
	it is difficult to conduct multi-year, multi-storm epidemologic 
	studies without storm hazard datasets that can easily be linked to 
	human health data. Further, while some storm exposure metrics are
	straightforward to measure (e.g., distance from a community to a 
	storm's track), they may poorly capture exposure to storm hazards. \\ 
\textbf{Objectives:} (1)~Provide an open-source dataset to improve tropical
	cyclone exposure assessment for health-related research and
	(2)~investigate patterns and agreement between county-level assessments of tropical cyclone
	exposure when based on different storm hazards, to inform the design of tropical cyclone epidemiological studies. \\ 
\textbf{Methods:} We created an open-source dataset with county-level data on
	exposure to tropical cyclones in the \ac{US}. The dataset includes
	measures of distance from the storm track, maximum sustained wind,
	rainfall, flooding, and tornadoes for all land-falling or near-land
	Atlantic basin storms over multiple years (1996\,--\,2011 for all
	metrics, up to~1988\,--\,2015 for specific metrics). We compared these
	measurements to other potential data sources and investigated patterns
	and agreement among binary exposure classifications based on these
	metrics across multiple years and counties. \\ 
\textbf{Results:} Here, we created and shared a large-scale dataset with 
	consistent and comprehensive measurements of county-level exposure
	to storm hazards. These data typically 
	agreed with data from other sources, although there are some caveats
	that should be considered by potential users. Long-term geographic patterns in binary 
	classifications based on these data agree with previous results from
	atmospheric science. County-level exposure classification for a storm differs substantially 
	depending on the hazard metric used. \\ 
\textbf{Discussion:} Our results provide a multi-hazard dataset that can be
	leveraged for epidemiologic research on tropical cyclones.  Further,	
	our results suggest that epidemiologic studies that use distance as a
	surrogate for tropical cyclone exposure, or use one hazard-based metric
	when the impact is partly or fully caused by a different hazard, will 
	likely be prone to exposure misclassification.  
\end{abstract}
