\begin{abstract}

\noindent \textbf{Background:} 
        Tropical cyclone epidemiology can be 
	advanced through methods to assess 
	county-level exposure to storm hazards, particularly 
	methods that are comprehensive and 
	consistent across space and time, allowing multi-year, multi-storm studies. 
	Further, an understanding of 
	patterns in and between hazard-specific exposure metrics can help in 
	designing tropical cyclone epidemiological studies.\\ 
\textbf{Objectives:} (1)~Provide an open-source dataset to improve tropical
	cyclone exposure assessment for epidemiological research and
	(2)~investigate patterns and agreement between county-level assessments of tropical cyclone
	exposure based on different storm hazards. \\ 
\textbf{Methods:} We created an open-source dataset with county-level data on
	exposure to four tropical cyclone hazards: maximum sustained winds,
	rainfall, flooding, and tornadoes. The data cover all eastern \ac{US} counties 
	for all land-falling or near-land
	Atlantic basin storms for~1996\,--\,2011 for all
	metrics (up to~1988\,--\,2015 for specific metrics). 	
	We validated 
	measurements against other data sources and investigated patterns
	and agreement among binary exposure classifications based on these
	metrics. We also compared exposure assessment
	based on using distance between a county and the storm's track as a proxy for
	storm exposure.\\ 
\textbf{Results:} Measurements in the open-source dataset created here typically
        agreed with data from other sources, and we present and discuss areas of
	disagreement and other caveats for using the data.
        Long-term geographic patterns in binary 
	classifications based on these data agree with previous results from
	atmospheric science. County-level exposure classification for a storm differs substantially 
	depending on the hazard metric used. \\ 
\textbf{Discussion:} Our results provide a multi-hazard dataset that can be
	leveraged for epidemiologic research on tropical cyclones.  Further,	
	our results suggest that the use of distance as a
	surrogate for tropical cyclone exposure, or use one hazard-based metric
	when the impact is partly or fully caused by a different hazard, would often 
	result in exposure misclassification in epidemiological studies.  
\end{abstract}
