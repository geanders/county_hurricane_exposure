\begin{abstract}

\noindent \textbf{Background:} 
        Tropical cyclone epidemiology can be 
	advanced through exposure assessment methods that are comprehensive and 
	consistent across space and time, as these facilitate multi-year, multi-storm studies. 
	Further, an understanding of 
	patterns in and between exposure metrics that are based on specific hazards of the storm can help in 
	designing tropical cyclone epidemiologic research.\\ 
\textbf{Objectives:} (1)~Provide an open-source dataset for tropical
	cyclone exposure assessment for epidemiologic research and
	(2)~investigate patterns and agreement between county-level assessments of tropical cyclone
	exposure based on different storm hazards. \\ 
\textbf{Methods:} We created an open-source dataset with county-level data on
	exposure to four tropical cyclone hazards: peak sustained wind,
	rainfall, flooding, and tornadoes. The data cover all eastern \ac{US} counties 
	for all land-falling or near-land
	Atlantic basin storms, covering~1996\,--\,2011 for all
	metrics and up to~1988\,--\,2018 for specific metrics. 	
	We validated 
	measurements against other data sources and investigated patterns
	and agreement among binary exposure classifications based on these
	metrics, as well as compared them to use of distance from the storm's track as a proxy for
	exposure.\\ 
\textbf{Results:} Our open-source dataset was typically consistent with data from other sources,
	and we present and discuss areas of disagreement and other caveats. Over the 
	study period and area, tropical cyclones typically brought different hazards to different
	counties. Therefore, when comparing exposure assessment between
	different hazard-specific metrics, agreement was usually low, as it also was when 
	comparing exposure assessment based on a distance-based proxy measurement and 
	any of the hazard-specific metrics.\\
\textbf{Discussion:} Our results provide a multi-hazard dataset that can be
	leveraged for epidemiologic research on tropical cyclones, as well as 
	insights that can inform the design and analysis for tropical cyclone epidemiologic 
	research.
\end{abstract}
