\section*{Introduction}

Tropical cyclones, including hurricanes, can have disastrous impacts in the
\ac{US}. A tropical cyclone's high winds can bring health risks
and property damage through structural damage of houses and other buildings,
falling trees, and wind-borne debris~\citep{rappaport2000} and cause power
outages~\citep{liu2005, han2009}, which introduce a number of threats to human
and ecological health, including water quality risks if the outage affects
wastewater treatment plants~\citep{mallin2006}.  Other risks can exist without
severe wind; for example, one study found that most of the direct
hurricane-related deaths in the \ac{US} between~1970 and~1999 occurred in cases
when wind was below hurricane strength, including for Tropical Storms Charley
in~1998 and Alberto in~1994~\citep{rappaport2000}.  Tropical cyclones can
produce excessive rain, especially in certain topographies (e.g., near
mountains), so counties well inland sometimes experience more extreme rain than
coastal counties. Flood risks from tropical cyclones can result from this rain,
although the two risks are not perfectly correlated~\citep{chen2015}. In the
\ac{US}, over half of hurricane-related direct deaths from~1970 to~1999 from
Atlantic basin storms were a result of freshwater flooding~\citep{rappaport2000}. 
Flooding can also degrade water quality~\citep{mallin2006}, which can threaten both 
human and ecological health.

To measure tropical cyclone exposure for either investigations of exposure
patterns or for estimation of tropical cyclone risks and impacts, including
those related to human health, studies have varied widely in how they determine
tropical cyclone exposure (some examples shown in Figure~S1). While some
studies have measured exposure based on specific hazards of the tropical
cyclone (e.g., wind, rain), many have used distance from the tropical
cyclone's central track as a surrogate in identifying exposure to the hazards
of a tropical cyclone (e.g.,~\citet{czajkowski2011, tansel2010, kinney2008,
caillouet2008increase}).  

Distance from the tropical cyclone's central track can be easily measured using
widely-available hurricane tracking data. There are, however, a number of
limitations to using distance to assess exposure to tropical cyclones, and
there are reasons to suspect that exposure classifications based on other
tropical cyclone hazards may disagree with distance-based classifications.
Tropical cyclones vary dramatically in size: \ac{US} tropical cyclones have
been observed with radii to maximum winds as small as~20~\si{\kilo\metre} and as
large as~200~\si{\kilo\metre}~\citep{mallin2006, quiring2011variations}. While a
number of tropical cyclone hazards are strongly associated with distance from
the tropical cyclone's center (e.g., wind and, at the coast, storm surge and
waves~\citep{rappaport2000, kruk2010}), other hazards like dangerous rain,
floods, and tornadoes can occur well away from the tropical cyclone's central
track~\citep{rappaport2000, atallah2007, moore2012}.  For example, fatal
tropical cyclone tornadoes, which were linked to over~300 deaths in the \ac{US}
between~1995 and~2009, most often occur~200\,--\,500~\si{\kilo\metre} from the
tropical cyclone's center~\citep{moore2012}. 

Further, distance-based exposure metrics that use
buffers tend to use an equal buffer distance on each side of the tropical
cyclone track (e.g.,~\cite{czajkowski2011, grabich2015, grabich2016,
zandbergen2009, tansel2010}), but the forces of a tropical cyclone tend to be
distributed around the center in a non-symmetrical way. For example, extreme
winds are more common to the right of the track, where counter-clockwise
cyclonic winds move in concert with the tropical cyclone's forward 
motion~\citep{halverson2015}, and the fatal tornadoes associated with \ac{US} tropical
cyclones between~1995 and~2009 occurred almost exclusively to the right of the
tropical cyclone's track, mostly in the right front quadrant of the tropical
cyclone~\citep{moore2012}. Rain, conversely, is often heaviest to the left of
the tropical cyclone's track, especially when the tropical cyclone interacts
with other weather systems~\citep{atallah2003, atallah2007, zhu2013variations}
or undergoes an extratropical transition~\citep{elsberry2002}.

[Something about differences in other metrics]

The use of different exposure metrics across studies hampers the development of
a scientific consensus on tropical cyclone-related risks, as differences
observed between studies could result from differences in exposure assessment.
Further, if a study misclassifies exposure to the hazard or hazards that cause
an impact, estimates of tropical cyclone risks or impacts will be biased,
making it hard to identify true associations. These concerns are particularly
important if there are large differences in the locations that are determined
to be exposed based on different metrics.  

To investigate this, we measured county-level exposure to tropical cyclones in
all eastern \ac{US} counties using five metrics
(Table~\ref{tab:exposuremetrics}): (1)~distance to tropical cyclone track;
(2)~maximum sustained wind speed; (3)~cumulative rainfall; (4)~flood events;
and (5)~tornado events.  We assessed exposure at the county level since data on
many potential impacts are available at county-level aggregations (e.g., direct
hurricane-related deaths~\citep{czajkowski2011}; birth
outcomes~\citep{grabich2015, grabich2016}; autism
prevalence~\citep{kinney2008}) and since decisions and policies to prepare for,
and respond to, tropical cyclones are often undertaken at the county
level~\citep{zandbergen2009, rappaport2000}.  We explored patterns in these
exposure assessments and measured how well exposure classification agreed
across these five metrics in terms of classifying specific counties as exposed
to a tropical cyclone.  Finally, to make this hazard-specific tropical cyclone
exposure data available to other researchers, we published open source software
for county-based hurricane exposure assessment in \ac{US}
counties~\citep{hurricaneexposure}.
