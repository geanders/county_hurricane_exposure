\section*{Introduction}

\acresetall

Hurricanes and other tropical cyclones can severely impact human health in
\ac{US} communities, as shown in several long-term national
studies~\parencite{rappaport2000, rappaport2014fatalities,
rappaport2016fatalities, czajkowski2010fatal, czajkowski2011, moore2012}.
These studies characterize tropical cyclone health impacts using fatality data
aggregated in large part from disaster-related mortality ascertainment and
surveillance conducted by agencies like the \ac{US} Centers for Disease Control
and Prevention, the \ac{US} National Weather Service, and local vital
statistics departments. These fatality data are based on case-by-case
ascertainment: identifying and characterizing deaths for which there is a clear
link with a disaster, through death certificate coding or other indicators
\parencite{rocha2017medicolegal}.  

For several types of climate-related disasters---including heat waves, floods,
and wildfires---such research has been richly supplemented by studies that
estimate community-wide excess mortality and morbidity associated with the
disaster (e.g.,~\cite{anderson2010heat, son2012impact, haikerwal2015impact,
liu2017wildfire, milojevic2017mental}).  These studies, which typically use time
series or case-crossover designs, compare the community-wide rate of a health
outcome during a disaster to rates during comparable
non-disaster periods.  While such studies cannot attribute specific cases
(e.g., a specific death) to a disaster, they can quantify the community-wide
change in health risk during disasters and capture impacts that might be missed
or underestimated with traditional disaster-related mortality ascertainment and
surveillance.  In many cases, these studies analyze multi-year, multi-community
data, allowing them to estimate average associations over many disasters and to
explore how a disaster's characteristics, or the characteristics of the
community it hits, modify associated health risks
(e.g.,~\cite{anderson2010heat, son2012impact, liu2017wildfire}).  Some studies
have begun to use this approach to study the health impacts of tropical
cyclones, including several studies of Hurricane Maria
(e.g.,~\cite{santos2018use, santos2018differential}), Hurricane Sandy
(e.g.,~\cite{kim2016, mongin2017, swerdel2014}), and the 2004 hurricane season
in Florida~\parencite{mckinney2011}.  However, to expand this approach to
longer time periods and larger sets of communities, researchers must be able to
assess exposure to tropical cyclones consistently and comparably across storms,
years, and communities.  

The \ac{NHC} publishes a ``Best Tracks'' dataset that is considered the gold
standard for Atlantic-basin tropical cyclone tracks.  It records the central
position of a tropical cyclone every six hours, as well as the storm's minimum
central pressure and maximum sustained surface wind.  These data are openly
available through the \ac{HURDAT2}, a post-storm assessment that incorporates
data from several sources, including satellite data and, when available,
aircraft reconnaissance data~\parencite{landsea2013, jarvinen1988}.  With these
data, it is straightforward to measure a storm's direct path, and so to measure
whether a community was on or within a certain distance of that central track.
Some previous epidemiological studies have done this to assess exposure to a
tropical storm (e.g.,~\cite{currie2013, kinney2008, caillouet2008increase}), 
and this method can be applied consistently across years and communities 
for large-scale studies.

However, when epidemiological studies assess exposure to a tropical cyclone
using, as a proxy, how close the storm came to a community, they may
misclassify exposure~\parencite{grabich2015measuring}. While a number of
tropical cyclone hazards are strongly associated with distance from the
tropical cyclone's center (e.g., wind and, at the coast, storm surge and
waves~\parencite{rappaport2000, kruk2010}), other hazards like heavy rainfall,
floods, and tornadoes can occur well away from the tropical cyclone's central
track~\parencite{rappaport2000, atallah2007, moore2012}.  For example,
tornadoes generated by tropical cyclones, which were linked to over~300 deaths
in the \ac{US} between~1995 and~2009, most often
occur~200\,--\,500~\si{\kilo\metre} from the tropical cyclone's
center~\parencite{moore2012}.  Further, when studies use distance from the
storm's track to assess exposure, they often use the same buffer constraints on
each side of the storm track (e.g.,~\cite{kinney2008, currie2013}).  However,
the forces of a tropical cyclone tend to be distributed around its center
non-symmetrically.  Extreme winds are more common to the track's right, where
counter-clockwise cyclonic winds move in concert with the tropical cyclone's
forward motion~\parencite{halverson2015}, and almost all of the fatal tornadoes
associated with \ac{US} tropical cyclones between~1995 and~2009 occurred to the
right of the tropical cyclone's track~\parencite{moore2012}. Rain, conversely,
is often heaviest to the left of the track, especially when the tropical
cyclone interacts with other weather systems~\parencite{atallah2003,
atallah2007, zhu2013variations} or undergoes an extratropical
transition~\parencite{elsberry2002}.  The multi-hazard nature of tropical
cyclones therefore makes it hard to assess exposure based on how close the
storm's central track came to the community.  While other approaches have been
developed to incorporate storm hazards, particulary wind, into exposure
assessment (e.g., \cite{grabich2015measuring, zandbergen2009, czajkowski2011}),
there is not yet a standard approach, and when different studies use
different datasets or storm hazards when assessing storm exposure, it becomes
difficult to compare and aggregate findings. 

To help epidemiological researchers assess exposure to tropical cyclones, we
developed an open-source dataset~\parencite{hurricaneexposure}, which we
present here.  These data cover all counties in states in the eastern half of
the \ac{US} over multiple years and include five metrics characterizing
exposure to tropical cyclones (Table~\ref{tab:exposuremetrics}): (1)~closest
distance the storm's central track came to the county's center (a proxy
measurement of storm exposure used in some previous studies); (2)~peak
sustained surface wind at the county's center over the course of the storm;
(3)~cumulative rainfall in the county over the course of the storm;
(4)~flooding in the county concurrent with the storm; and (5)~tornadoes in the
county concurrent with the storm.  We aggregated these data at the county level
since data for epidemiological studies are often available at this level (e.g.,
direct hurricane-related deaths~\parencite{czajkowski2011}; birth
outcomes~\parencite{grabich2015, grabich2015measuring}; autism
prevalence~\parencite{kinney2008}) and since decisions and policies to prepare
for, and respond to, tropical cyclones are often undertaken at the county
level~\parencite{zandbergen2009, rappaport2000}.  We also wrote functions to
explore and map these data and to link them with county-level health
data~\parencite{hurricaneexposuredata}. 

Here, we describe how we developed this dataset and explore how its
measurements compare to other data that could be used to characterize tropical
cyclone hazards for epidemiologic research. Further, we expand on previous
research on methods to measure exposure to tropical cyclones for epidemiologic
research~\parencite{grabich2015measuring}. This previous study investigated
differences in which communities were assessed as exposed to tropical cyclones
during the 2004 hurricane season in Florida, comparing exposure assessment
based on distance to the storm's track or versus assessment based on a metric
that incorporated storm-generated winds within the county.  They found
important differences across methods of assessing storm exposure, concluding
that a study may be prone to bias from exposure misclassification if distance
to the storm track is used as a proxy measurement for exposure
assessment~\parencite{grabich2015measuring}.  Here we expand to investigate
this question across a larger set of counties and storm seasons.  Further, we
investigate patterns in exposure classification based on other metrics of
storm-related hazards---rainfall, flooding, and tornadoes---all of which are
important for inland health impacts of tropical
cyclones~\parencite{czajkowski2011, moore2012}. These results can help
epidemiologists design studies and plan statistical analysis for multi-year,
multi-community studies that estimate excess mortality and morbidity associated
with exposure to the hazards brought by tropical cyclones. 
