\section*{Introduction}

\acresetall

Recently, several major tropical cyclones have hit \ac{US} cities, including
Hurricanes Harvey, Irma, and Maria in 2017 and Hurricanes Florence and Michael
in 2018. Researchers are currently exploring how these storms impacted human
health (e.g., [Maria refs, others?]), including through projects funded through
NIH Rapid Response grants [NIH RePorter ref].  Evidence from this research will
add to evidence from previous hurricanes, including Sandy in 2012 [refs] and
Katrina in 2005 [refs], which found these storms can create substantial risk
not only for accidental health outcomes (e.g., drowning, crushing deaths and
injuries, electrocution, carbon monoxide poisoning) but also for
non-accidental, non-communicable diseases [refs]. However, much of the
epidemiological research to date on the health impacts of tropical cyclones has
focused on a single storm (e.g., [refs]), or has relied on death and injury
counts for which a link to the storm was identifiable through mention on a
death certificate or other information about the death [refs], an approach that
may miss many of the health impacts of storms, particularly from non-accidental
causes, as evidenced with Hurricane Maria [ref for this?]. 

A recent review highlights the need to move beyond single-storm and descriptive
analyses to larger-scale studies with consistently [defined?] exposure and
outcome assessment [ref].  Key to expanding tropical cyclone epidemiology are
exposure assessment methods that appropriately capture those hazards of the
storm that are relevant to human health, methods that can be applied
consistently across multiple storms, locations, and years. 

The \ac{NHC}'s ``Best Tracks'' dataset is considered the gold standard in
recording the location and central intensity of Atlantic-basin tropical
cyclones. This data is provided through the \ac{HURDAT2} dataset, a post-storm
assessment conducted by the \ac{US} \ac{NHC} that incorporates data from a
variety of sources, including satellite data and, when available, aircraft
reconnaissance data~\parencite{landsea2013, jarvinen1988}. This dataset records
the central position of a storm every six hours (at the synoptic times of
6:00~am, 12:00~pm, 6:00~pm, and 12:00~am \ac{UTC}) as well as measures of the
minimum pressure [at the storm's center?] and the storm's central maximum
sustained winds (typically located in the eye wall [?] if the storm has
developed an eye).  From the ``Best Tracks'' data, it is straightforward to
measure whether a study location was in the direct path, or within a certain
distance of the path, of a storm using geographical software. Some studies have
used distance to a storm's track either in assessing average exposure patterns
to tropical storms [refs] or for assigning exposure for health studies [refs].

However, the hazards that tropical cyclones bring that threaten human health
are not perfectly aligned with distance from the storm's track.  Tropical
cyclones are multi-hazard events. Recent storms have shown how much storms can
differ in intensity of these hazards; Hurricanes Michael and Maria demonstrated
the threats of extreme storm-related winds, for example, while Hurricanes
Florence and Harvey were more notable for extreme rain and flooding [refs].
Depending on the health outcome considered, some storm hazards might be more
plausible as a risk factor than others. Respiratory outcomes in the months
after a storm, for example, might be increased through a pathway of heightened
mold exposure from flooding, while heat stroke and other heat-related outcomes
in the days immediately after a storm might follow from a pathway of extreme
winds causing power outages, resulting in loss of air conditioning [refs].
Geographical patterns can differ for these different storm hazards, and so
exposure assessment based on one hazard could create exposure misclassification
if the pathway for health risks is through a different storm hazard.   

% Tropical cyclones, including hurricanes, can have disastrous impacts in the
% \ac{US}. A tropical cyclone's high winds can bring health risks
% and property damage through structural damage of houses and other buildings,
% falling trees, and wind-borne debris~\parencite{rappaport2000} and cause power
% outages~\parencite{liu2005, han2009}, which introduce a number of threats to human
% and ecological health, including water quality risks if the outage affects
% wastewater treatment plants~\parencite{mallin2006}.  Other risks can exist without
% severe wind; for example, one study found that most of the direct
% hurricane-related deaths in the \ac{US} between~1970 and~1999 occurred in cases
% when wind was below hurricane strength, including for Tropical Storms Charley
% in~1998 and Alberto in~1994~\parencite{rappaport2000}.  Tropical cyclones can
% produce excessive rain, especially in certain topographies (e.g., near
% mountains), so counties well inland sometimes experience more extreme rain than
% coastal counties. Flood risks from tropical cyclones can result from this rain,
% although the two risks are not perfectly correlated~\parencite{chen2015}. In the
% \ac{US}, over half of hurricane-related direct deaths from~1970 to~1999 from
% Atlantic basin storms were a result of freshwater flooding~\parencite{rappaport2000}. 
% Flooding can also degrade water quality~\parencite{mallin2006}, which can threaten both 
% human and ecological health.

% To measure tropical cyclone exposure for either investigations of exposure
% patterns or for estimation of tropical cyclone risks and impacts, including
% those related to human health, studies have varied widely in how they determine
% tropical cyclone exposure (some examples shown in Figure~S1). While some
% studies have measured exposure based on specific hazards of the tropical
% cyclone (e.g., wind, rain), many have used distance from the tropical cyclone's
% central track as a surrogate in identifying exposure to the hazards of a
% tropical cyclone (e.g.,~\textcite{czajkowski2011, tansel2010, kinney2008,
% caillouet2008increase}).  

% Distance from the tropical cyclone's central track can be easily measured using
% widely-available hurricane tracking data. There are, however, a number of
% limitations to using distance to assess exposure to tropical cyclones, and

% Geographical patterns likely differ for these different storm hazards, and so
% exposure assessment based on one hazard could create exposure misclassification
% if the pathway for health risks is through a different storm hazard.  
 
There are reasons to suspect that exposure classifications based on tropical
cyclone hazards like wind and rain may disagree with distance-based
classifications, based only on distance from the storm's path. Further, an
assessment based on one storm hazard (e.g., wind) might disagree with assessments based on
other hazards (e.g., rain or flooding).
Tropical cyclones vary dramatically in size: \ac{US} tropical cyclones have
been observed with radii to maximum winds as small as~20~\si{\kilo\metre} and
as large as~200~\si{\kilo\metre}~\parencite{mallin2006, quiring2011variations}.
While a number of tropical cyclone hazards are strongly associated with
distance from the tropical cyclone's center (e.g., wind and, at the coast,
storm surge and waves~\parencite{rappaport2000, kruk2010}), other hazards like
dangerous rain, floods, and tornadoes can occur well away from the tropical
cyclone's central track~\parencite{rappaport2000, atallah2007, moore2012}.  For
example, fatal tropical cyclone tornadoes, which were linked to over~300 deaths
in the \ac{US} between~1995 and~2009, most often
occur~200\,--\,500~\si{\kilo\metre} from the tropical cyclone's
center~\parencite{moore2012}. 

Further, studies that have used distance-based exposure metrics have tended to
use an equal buffer distance on each side of the tropical cyclone track
(e.g.,~\cite{czajkowski2011, grabich2015, grabich2016, zandbergen2009,
tansel2010}), but the forces of a tropical cyclone tend to be distributed
around the center in a non-symmetrical way. For example, extreme winds are more
common to the right of the track, where counter-clockwise cyclonic winds move
in concert with the tropical cyclone's forward
motion~\parencite{halverson2015}, and the fatal tornadoes associated with
\ac{US} tropical cyclones between~1995 and~2009 occurred almost exclusively to
the right of the tropical cyclone's track, mostly in the right front quadrant
of the tropical cyclone~\parencite{moore2012}. Rain, conversely, is often
heaviest to the left of the tropical cyclone's track, especially when the
tropical cyclone interacts with other weather systems~\parencite{atallah2003,
atallah2007, zhu2013variations} or undergoes an extratropical
transition~\parencite{elsberry2002}.

If a study misclassifies exposure to the hazard or hazards that cause an
impact, estimates of tropical cyclone risks or impacts will be biased, making
it hard to identify true associations. The use of different exposure metrics
across studies hampers the development of a scientific consensus on tropical
cyclone-related risks, as differences observed between studies could result
from differences in exposure assessment.  These concerns are particularly
important if there are large differences in the locations that are determined
to be exposed based on different metrics.  

To investigate this, and to provide open-source data on tropical cyclone exposures
to other researchers, we developed a dataset of county-level exposure to
tropical cyclones in all eastern \ac{US} counties for five different metrics
(Table~\ref{tab:exposuremetrics}): (1)~distance to tropical cyclone track;
(2)~maximum sustained wind speed; (3)~cumulative rainfall; (4)~flood events;
and (5)~tornado events.  We assessed exposure at the county level since data on
many potential impacts are available at county-level aggregations (e.g., direct
hurricane-related deaths~\parencite{czajkowski2011}; birth
outcomes~\parencite{grabich2015, grabich2016}; autism
prevalence~\parencite{kinney2008}) and since decisions and policies to prepare
for, and respond to, tropical cyclones are often undertaken at the county
level~\parencite{zandbergen2009, rappaport2000}.  We explored patterns in these
exposure assessments and measured how well exposure classification agreed
across these five metrics in terms of classifying specific counties as exposed
to a tropical cyclone.  Finally, to make this hazard-specific tropical cyclone
exposure data available to other researchers, we published open source software
for county-based hurricane exposure assessment in \ac{US}
counties~\parencite{hurricaneexposure}.
