\section*{Introduction}

\acresetall

Recently, several major tropical cyclones have hit \ac{US} cities, including
Hurricanes Harvey, Irma, and Maria in 2017 \parencite{blake20182017} and
Hurricanes Florence and Michael in 2018 \parencite{avila20192018}. Researchers
are exploring how these storms impacted human health (e.g.,
\parencite{santos2018use, rivera2018estimating, santos2018differential,
grineski2019impact, issa2018deaths, tanz2019notes, paul2019brief}), including
through projects funded by \ac{NIH} Rapid Response grants
\parencite{nihreporter}. These projects will add to evidence characterizing how
hurricanes impact human health, including evidence from multi-year studies
based on fatality data from post-disaster surveillance
\parencite{rappaport2000, rappaport2014fatalities, rappaport2016fatalities,
czajkowski2010fatal, czajkowski2011, moore2012}.

For several types of climate-related disasters---including heat waves, floods,
and wildfires---research based on surveillance data has been richly
supplemented by studies analyzing trends in community-wide administrative or
vital data, including time series or case-crossover studies (e.g.,
\cite{anderson2010heat, son2012impact, haikerwal2015impact, liu2017wildfire,
milojevic2017mental}).  These studies typically compare the daily rate of a
health outcome in a community during disasters to rates during comparable
non-disaster periods.  While these studies cannot attribute specific cases
(e.g., a specific person's death) to a disaster, they can help characterize the
community-wide change in health risk during disasters, and can help capture
impacts that might be missed or underestimated in post-disaster surveillance.
In many cases, these studies analyze multi-year, multi-community data, allowing
them to estimate average associations over many disasters and to explore how a
disaster's characteristics, or the characteristics of the community it hits,
modify associated health risk (e.g., \cite{anderson2010heat, son2012impact,
liu2017wildfire}).  Some studies have begun to use this approach to study the
health impacts of tropical cyclones, including several studies of Hurricane
Maria (e.g., \cite{santos2018use, santos2018differential}) and a study of the
2004 hurricane season in Florida \parencite{mckinney2011}.  However, to expand
this approach to study longer time periods or a larger set of communities, it
is critical that researchers be able to assess exposure to tropical cyclones in
a way that is consistent and comparable across storms, years, and communities.  

The \ac{NHC} publishes a ``Best Tracks'' dataset that is considered the gold
standard for Atlantic-basin tropical cyclone tracks.  It records the central
position of a tropical cyclone every six hours, as well as the storm's minimum
central pressure and maximum wind speed.  This data is openly available through
the \ac{HURDAT2}, a post-storm assessment conducted by the \ac{US} \ac{NHC}
that incorporates data from a variety of sources, including satellite data and,
when available, aircraft reconnaissance data~\parencite{landsea2013,
jarvinen1988}. With this data, it is straightforward to measure whether a
community was in the direct path, or within a certain distance of the path, of
a storm. Some studies have indeed used this data to assess tropical cyclone
exposure based on how near the storm's central track came to the county, either
to assess average exposure patterns for tropical cyclones [refs] or to
characterize exposure for health studies (e.g., \cite{currie2013}).

When epidemiological studies assess exposure to a tropical cyclone based on 
how close the storm came to a community, however, they may misclassify exposure. 
Tropical cyclones vary dramatically in size: \ac{US} tropical
cyclones have been observed with radii to maximum winds as small
as~20~\si{\kilo\metre} and as large
as~200~\si{\kilo\metre}~\parencite{mallin2006, quiring2011variations}.  While a
number of tropical cyclone hazards are strongly associated with distance from
the tropical cyclone's center (e.g., wind and, at the coast, storm surge and
waves~\parencite{rappaport2000, kruk2010}), other hazards like heavy rainfall,
floods, and tornadoes can occur well away from the tropical cyclone's central
track~\parencite{rappaport2000, atallah2007, moore2012}.  For example, fatal
tropical cyclone tornadoes, which were linked to over~300 deaths in the \ac{US}
between~1995 and~2009, most often occur~200\,--\,500~\si{\kilo\metre} from the
tropical cyclone's center~\parencite{moore2012}.  

Further, when studies use distance from the storm's track to assess exposure,
they often use the same buffer constraints on each side of the tropical cyclone
track (e.g.,~\cite{czajkowski2011, zandbergen2009, grabich2015measuring,
tansel2010, currie2013}). However, the forces of a tropical cyclone tend to be
distributed around its center non-symmetrically. Extreme winds are more common
to the track's right, where counter-clockwise cyclonic winds move in concert
with the tropical cyclone's forward motion~\parencite{halverson2015}, and the
fatal tornadoes associated with \ac{US} tropical cyclones between~1995 and~2009
occurred almost exclusively to the right of the tropical cyclone's track,
mostly in the storm's right front quadrant~\parencite{moore2012}. Rain,
conversely, is often heaviest to the left of the track, especially when the
tropical cyclone interacts with other weather systems~\parencite{atallah2003,
atallah2007, zhu2013variations} or undergoes an extratropical
transition~\parencite{elsberry2002}.

The multi-hazard nature of tropical cyclones therefore makes it hard to assess
exposure based only on a proxy measurement, like how close the storm's central
track came to the community. While other approaches have been developed to
incorporate wind hazards into exposure assessment (e.g.,
\cite{grabich2015measuring, zandbergen2009}), but these do not incorporate other
storm hazards. To help epidemiological researchers with exposure assessment for
tropical cyclones, we here develop a dataset of county-level exposure to
tropical cyclones in all eastern \ac{US} counties, which we have packaged as an
open-source dataset~\parencite{hurricaneexposure}, as well as an associated
package of open-source functions that can be used to explore and map this data
and to link it with county-level health data~\parencite{hurricaneexposuredata}.
These data cover five metrics characterizing exposure to tropical cyclone
hazards (Table~\ref{tab:exposuremetrics}): (1)~closest distance the storm's
central track came to the county's center (a proxy measurement of storm
exposure used in some previous studies); (2)~maximum sustained wind speed at
the county's center over the course of the storm; (3)~cumulative rainfall in
the county over the course of the storm; (4)~flood events in the county
associated with the storm; and (5)~tornado events in the county associated with
the storm.  We aggregated these data at the county level since data on many
potential impacts are available at county-level aggregations (e.g., direct
hurricane-related deaths~\parencite{czajkowski2011}; birth
outcomes~\parencite{grabich2015, grabich2016}; autism
prevalence~\parencite{kinney2008}) and since decisions and policies to prepare
for, and respond to, tropical cyclones are often undertaken at the county
level~\parencite{zandbergen2009, rappaport2000}. 

In this paper, we describe how we developed this dataset and explore how the
measurements in it compare to other data that could be used to characterize
tropical cyclone hazards in the context of epidemiological studies.  Further,
we explore and present patterns in binary exposure assessments based on these
measurements, both for each individual metric and in terms of agreement between
metrics. These analyses can help epidemiologists as they develop study designs
and plan statistical analysis for multi-year, multi-community studies that
investigate how community-wide rates of health outcomes change during exposure
to the hazards brought by tropical cyclones. These analyses can also help
researchers compare and interpret results from previous epidemiological studies
that have assessed tropical cyclone exposure in different ways, helping to
[synthesize] previous epidemiologic results to form a better picture of how
these storms can affect human health. 
