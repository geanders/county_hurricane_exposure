\section*{Introduction}

\acresetall

Recently, several major tropical cyclones have hit \ac{US} cities, including
Hurricanes Harvey, Irma, and Maria in 2017 \parencite{blake20182017} and
Hurricanes Florence and Michael in 2018 \parencite{avila20192018}. Researchers
are exploring how these storms impacted human health (e.g.,
\parencite{santos2018use, rivera2018estimating, santos2018differential,
grineski2019impact, issa2018deaths, tanz2019notes, paul2019brief}), including
through projects funded through \ac{NIH} Rapid Response grants
\parencite{nihreporter}. These projects will add to evidence from previous
hurricanes, including Sandy in 2012 (e.g., \parencite{swerdel2014}) and Katrina
in 2005 (e.g., \parencite{burton2009health}), characterizing the health impacts
of hurricanes. 

A recent review highlights the need to supplement these single-storm studies
with multi-year, multi-storm studies, to identify patterns that are common
across multiple storms [ref]. While multi-year studies of hurriance mortality
impacts have been conducted using disaster surveillance data, studies based on
administrative data could identify risks for non-accidental mortality and
morbidity that those studies likely undercount [ref]. Multi-storm studies with
administrative data require exposure assessment that is consistent across time
and space, allowing the administrative data to be linked to disaster exposure
[ref]. Key to expanding tropical cyclone epidemiology are therefore exposure
assessment methods that appropriately capture hazards of the storm that are
relevant to human health, methods that can be applied consistently across
multiple storms, locations, and years. 

The \ac{NHC} publishes a ``Best Tracks'' dataset that is considered the gold
standard for Atlantic-basin tropical cyclone tracks.  It records the central
position of a tropical cyclone every six hours, as well as the storm's
\underline{minimum pressure} and \underline{central maximum sustained winds}. This data is openly
available through the \ac{HURDAT2}, a post-storm assessment conducted
by the \ac{US} \ac{NHC} that incorporates data from a variety of sources,
including satellite data and, when available, aircraft reconnaissance
data~\parencite{landsea2013, jarvinen1988}. With this data, it is
straightforward to measure whether a community was in the direct path, or
within a certain distance of the path, of a storm. Some studies have indeed
used this data to assess tropical cyclone exposure based on how near the
storm's central track came to the county, either to assess average exposure
patterns to tropical storms [refs] or to characterize exposure for health
studies [refs].

When epidemiological studies assess exposure to a tropical cyclone based on 
how close the storm came to a community, however, they may misclassify exposure. 
Tropical cyclones vary dramatically in size: \ac{US} tropical
cyclones have been observed with radii to maximum winds as small
as~20~\si{\kilo\metre} and as large
as~200~\si{\kilo\metre}~\parencite{mallin2006, quiring2011variations}.  While a
number of tropical cyclone hazards are strongly associated with distance from
the tropical cyclone's center (e.g., wind and, at the coast, storm surge and
waves~\parencite{rappaport2000, kruk2010}), other hazards like heavy rainfall,
floods, and tornadoes can occur well away from the tropical cyclone's central
track~\parencite{rappaport2000, atallah2007, moore2012}.  For example, fatal
tropical cyclone tornadoes, which were linked to over~300 deaths in the \ac{US}
between~1995 and~2009, most often occur~200\,--\,500~\si{\kilo\metre} from the
tropical cyclone's center~\parencite{moore2012}.  

Further, when studies use distance from the storm's track to assess exposure,
they often use an equal buffer distance on each side of the tropical cyclone
track (e.g.,~\cite{czajkowski2011, grabich2015, grabich2016, zandbergen2009,
tansel2010}). However, the forces of a tropical cyclone tend to be distributed
around the center in a non-symmetrical way. Extreme winds are more common to
the right of the track, where counter-clockwise cyclonic winds move in concert
with the tropical cyclone's forward motion~\parencite{halverson2015}, and the
fatal tornadoes associated with \ac{US} tropical cyclones between~1995 and~2009
occurred almost exclusively to the right of the tropical cyclone's track,
mostly in the right front quadrant of the tropical
cyclone~\parencite{moore2012}. Rain, conversely, is often heaviest to the left
of the tropical cyclone's track, especially when the tropical cyclone interacts
with other weather systems~\parencite{atallah2003, atallah2007,
zhu2013variations} or undergoes an extratropical
transition~\parencite{elsberry2002}.

If a study misclassifies exposure to the hazard or hazards that cause the
health risk being studied, the study will generate biased estimates of tropical
cyclone risks and impacts. Further, when different studies use different
methods to assess exposure to tropical cyclones, their results are difficult to
meaningfully compare and aggregate. If different methods identify
similar sets of communities as ``exposed'',  these concerns are less serious.
However, if different methods differ substantially in which communities they
identify as ``exposed'', it makes it very important that epidemiological
studies are thoughtful in how they assess exposure.

To provide open-source data on tropical cyclone exposures to other researchers,
we developed a dataset of county-level exposure to tropical cyclones in all
eastern \ac{US} counties for five different metrics
(Table~\ref{tab:exposuremetrics}): (1)~distance to tropical cyclone track;
(2)~maximum sustained wind speed; (3)~cumulative rainfall; (4)~flood events;
and (5)~tornado events~\parencite{hurricaneexposure}. This data is provided at
the county level since data on many potential impacts are available at
county-level aggregations (e.g., direct hurricane-related
deaths~\parencite{czajkowski2011}; birth outcomes~\parencite{grabich2015,
grabich2016}; autism prevalence~\parencite{kinney2008}) and since decisions and
policies to prepare for, and respond to, tropical cyclones are often undertaken
at the county level~\parencite{zandbergen2009, rappaport2000}. Further, we
explored patterns in these exposure assessments across storm hazards and
measured how well exposure classification agreed across these five metrics in
terms of classifying specific counties as exposed to a tropical cyclone.  
