These data typically give measurements of
the tropical cyclone center's location [and central pressure / maximum
windspeed?] at~6-\si{\hour} intervals at synoptic times (i.e., 6:00~am,
12:00~pm, 6:00~pm, and 12:00~am \ac{UTC}); some landfalling tropical cyclones
have an additional observation at the time of landfall~\parencite{landsea2013}.

, using the Great Circle %
(WGS84 ellipsoid) method

We selected re-analysis data because of
this completeness. 

We selected re-analysis data over other data
sources as it included data for every county and every day through a continuous
period of the study (1988\,--\,2011)~\parencite{alhamdan2014environmental,
cdcwonder}. By comparison, observations from ground-based monitoring networks
had missing values spatially (i.e., for some study counties), temporally (for
some days), or both.  

We estimated daily rainfall for each study county by aggregating
hourly~1/8\si{\degree} gridded data from the \ac{NLDAS-2} precipitation data
files~\parencite{rui2013nldas}. These data integrate satellite-based and
land-based monitoring, applying a land-surface model to create a reanalysis
dataset that is spatially and temporally complete across the continental
\ac{US}~\parencite{rui2013nldas, alhamdan2014environmental}. To aggregate to
county-level, the hourly data at each grid point were summed to create a daily
rainfall total, and these grid point rainfall totals were then averaged across
all grid points within a county's~1990 \ac{US} Census
boundaries~\parencite{alhamdan2014environmental, cdcwonder}. These aggregated
county-level daily precipitation data are also available through the \ac{US}
Centers for Disease Control's WONDER database [ref] and are available for all
study counties for 1988\,--\,2011.

When combined with estimates of vulnerability of a
population to a natural hazard, such measurements of physical exposure can be
used to calculate risk of human losses from the
hazard~\parencite{peduzzi2009assessing}.

These measurements
captured the distance between the storm's central track and each study county
throughout the storm tracking period. 

To measure storm-associated rainfall in each county, we used re-analysis data.
Re-analysis data integrate data from multiple sources, including
satellite-based and ground-based monitoring, using a land-surface model to
combine data in a dataset that is spatially and temporally complete over a
certain period [ref]. By comparison, observations from ground-based monitoring
networks typically have missing values spatially (i.e., for some study
counties), temporally (for some days), or both. 

We included more days prior to the storm because
storm-related rains often precede the passing of the storm's center [ref]. 

Ground-based observations of wind speed are problematic for measuring exposure
to tropical cyclones, as instruments often fail at high wind speed, while
reanalysis data---often available at hourly or higher resolution---can lack a
fine enough temporal resolution to capture wind extremes associated with a
tropical cyclone. 
A tropical cyclone's rainfall can can begin a day or more before a
storm and last a day or more after, so this functionality makes it possible to
quantify the cumulative precipitation throughout several days surrounding the
storm's approach. 

\subsection*{Case study}

\textit{The societal impact of a disaster depends both on its geophysical
forcesi, on the population exposed to the hazard, and on the vulnerability of
those living in the geographical areas it
affects~\parencite{chakraborty2005population, anderson2003community,
cutter1996vulnerability}. An estimate of ``physical exposure''---which
...---can help to jointly capture these facets of disaster impacts [ref, maybe
the peduzzi one given below?]. As a case study, we measured physical exposure
based on each metric of exposure and investigated whether conclusions from
these assessments varied substantially.}

We collected data on the number of electricity-dependent Medicare beneficiaries
in each study county from the \ac{US} Department of Health \& Human Service's
emPower Map~2.0~\parencite{empower}. This subpopulation was selected because it
may be particularly susceptible to health impacts from tropical cyclone
exposure, especially through the pathway of storm-associated power outages and
evacuations. We calculated the physical exposure of the electricity-dependent
Medicare population in each county, based on tropical cyclone assessments using
each exposure metric (Table~\ref{tab:exposuremetrics}), following
\citeauthor*{peduzzi2009assessing}~(\citeyear{peduzzi2009assessing}):

\begin{equation}
E_c = F_c * P_c
\end{equation}

\textit{\noindent where~$E_c$ is the average yearly physical exposure among
electricity-dependent Medicare beneficiaries in county~$c$ to tropical cyclone
exposures based on a given metric,~$F_c$ is the estimated yearly expected
frequency of tropical cyclone exposures in county~$c$ based on that metric,
and~$P_c$ is the electricity-dependent Medicare population in
county~$c$, as of July~2017.}



